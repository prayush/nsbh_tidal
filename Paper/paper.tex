\documentclass[aps,prd,amsmath,floats,floatfix, twocolumn,
superscriptaddress,nofootinbib,showpacs]{revtex4-1}

\usepackage[colorlinks, pdfborder={0 0 0}, plainpages=false]{hyperref}
\usepackage{graphicx}
\usepackage{xspace}
\usepackage[usenames,dvipsnames]{color}
\usepackage{amssymb}
\usepackage[dvipsnames]{xcolor}
\usepackage{placeins}
\usepackage[utf8]{inputenc}
% \usepackage[switch,columnwise]{lineno} 
\usepackage{lineno}
\interfootnotelinepenalty=10000

\newcommand{\roughly}{\mathchar"5218\relax} % Different from \sim in spacing

% Macros for text changes
\newcommand{\red}{\textcolor{red}}
\newcommand{\dan}[1]{\textcolor{WildStrawberry}{#1}}
\newcommand{\Mark}[1]{\textcolor{Cerulean}{#1}}
\newcommand{\larry}[1]{\textcolor{OliveGreen}{#1}}
\newcommand{\bela}[1]{\textcolor{Blue}{#1}}
\newcommand{\saul}[1]{\textcolor{Orange}{#1}}
\newcommand{\prayush}{\textcolor{red!40!black}}

% Macros for text notes and comments
\newcommand{\Note}[1]{\textcolor{blue}{\textbf{[#1]}}}
\newcommand{\D}{\mathrm{d}}
\newcommand{\lambdans}{\Lambda_\mathrm{NS}}
\newcommand{\dlambda}{\delta\lambdans^{90\%}}

\newcommand{\ii}{\mathrm{i}}
\newcommand{\chibh}{\chi_\mathrm{BH}}
\newcommand{\chins}{\chi_\mathrm{NS}}
\newcommand{\mbh}{m_\mathrm{BH}}
\newcommand{\mns}{m_\mathrm{NS}}
\newcommand{\mchirp}{\mathcal{M}_c}
\newcommand{\LL}{\mathcal{L}}
\newcommand{\deff}{D_\mathrm{eff}}
\newcommand{\arr}{\mathtt{R}}

\newcommand{\Caltech}{\affiliation{Theoretical Astrophysics 350-17,
    California Institute of Technology, Pasadena, CA 91125, USA}}
\newcommand{\Cornell}{\affiliation{Center for Radiophysics and Space
    Research, Cornell University, Ithaca, New York 14853, USA}}
\newcommand{\CITA}{\affiliation{Canadian Institute for Theoretical
    Astrophysics, 60 St.~George Street, University of Toronto,
    Toronto, ON M5S 3H8, Canada}} %
\newcommand{\GWPAC}{\affiliation{Gravitational Wave Physics and
    Astronomy Center, California State University Fullerton,
    Fullerton, California 92834, USA}} %
\newcommand{\AEI}{\affiliation{Albert Einstein Institute,
Am M\"uhlenberg, Golm, Germany}} %
\newcommand{\CIFAR}{\affiliation{Canadian Institute for Advanced Research, 
    180 Dundas St.~West, Toronto, ON M5G 1Z8, Canada}} %



    
\newcommand{\NS}{\mathrm{NS}}
%%%%%%%%%%%%%%%%%%%%%%%%%%%%%%%%%%%%%%%%%%%%%%%%%%%%%%%%%%%%%%%

\begin{document}
%\linenumbers

\title{
Measuring neutron star tidal deformability with Advanced LIGO: a Bayesian analysis
of neutron star - black hole binary observations
}

\author{Prayush Kumar}\CITA\email{prkumar@cita.utoronto.ca}
\author{Michael P\"urrer}\AEI
\author{Harald P. Pfeiffer}\CITA\CIFAR\AEI

\date{\today}

\begin{abstract}
The pioneering discovery of gravitational waves (GW) by Advanced LIGO has ushered
us into an era of observational GW astrophysics. Compact binaries remain the 
primary target sources for GW observation, of which neutron star - black hole 
(NSBH) binaries form an important subset.
% 
GWs from NSBH sources carry signatures of (a) the star's tidal distortion by
the companion black hole (BH) during inspiral, and (b) its potential tidal
disruption near merger.
%
In this paper, we present a Bayesian study of the measurability of
neutron star (NS) tidal deformability $\lambdans\propto (R/M)_\mathrm{NS}^{5}$
from GW observation(s) of disruptive NSBH coalescences, taking into account the
crucial effect of BH spins.
% 
First, we find that if BHBH templates are used to estimate source 
parameters for an NSBH signal, a significant bias is introduced in the 
inference if the signal-to-noise ratio (SNR) is greater than $\simeq30$.
% 
Second, for (loud) signals with SNR $\gtrsim 23$ we can bound  $\lambdans$
by a factor of two.
% 
Finally, we study the improvement in our measurement of $\lambdans$ from 
multiple observations. Using an approximately astrophysical population, 
we find that (i) the median $\lambdans$ converges within $10\%$ of the true
value in approximately $20$ observations; (ii) in the same
number of observations, the associated $90\%$ credible intervals shrink to
$\pm 50\%$ of the true value.
% 
We find these results encouraging, and recommend that an effort to 
measure $\lambdans$ be taken within the LIGO-Virgo Collaboration once
we begin to detect NSBH coalescences.
\end{abstract}

\pacs{}
% 04.25.D- Numerical relativity
% 04.25.dg Numerical studies of black holes and black-hole binaries
% 04.25.Nx Post-Newtonian approximation; perturbation theory; related approximations 
% 04.30.-w Gravitational waves (see also 04.80.Nn Gravitational wave detectors and experiments)
% 04.30.Db Wave generation and sources 
% 02.70.Hm Spectral methods

\maketitle

%%%%%%%%%%%%%%%%%%%%%%%%%%%%%%%%%%%%%%%%%%%%%%%%%%%%%%%%%%%%%%%%%%%%%%%%%%%%%%%
\section{Introduction}\label{s1:introduction}
%%%%%%%%%%%%%%%%%%%%%%%%%%%%%%%%%%%%%%%%%%%%%%%%%%%%%%%%%%%%%%%%%%%%%%%%%%%%%%%

The Advanced LIGO (aLIGO) observatories completed their first observing run ``O1''
early-2016, operating at a factor of $3-4$ higher gravitational-wave
(GW) strain sensitivity than their first-generation 
counterparts~\cite{Shoemaker2009}.
% 
During O1, they made the first terrestrial observation of gravitational 
waves~\cite{LIGOVirgo2016a}. Emitted by a pair of coalescing black holes, these
waves heralded an era of observational GW astrophysics as they traveled 
through Earth.
% 
Towards the end of this decade, we expect aLIGO to reach its design sensitivity.
In addition to the US-based efforts, we also expect the French-Italian detector Advanced
Virgo~\cite{aVIRGO,aVirgo2}, Japanese detctor KAGRA~\cite{kagra,Somiya:2011np},
and LIGO-India~\cite{2013IJMPD..2241010U} to begin observing at comparable
sensitivities within a few years. With a global network of sensitive GW
 observatories, we can expect GW astronomy to face significant developments over the
coming years.
% 
% At design sensitivity, we expect aLIGO to observe $\sim 70$ compact binary
% mergers a year~\cite{Abadie:2010cf}.

% \red{
% Text about CBCs. How was each type observed so far?
% }
Coalescing compact binaries of stellar-mass black holes (BH) and/or
neutron stars (NS) are the primary targets for the second
generation GW detectors~\cite{Timmes:1995kp,Fryer:1999mi,RevModPhys.74.1015,
2010ApJ...714.1217B,2010ApJ...715L.138B,Dominik:2014yma,Belczynski:2006zi,
2012ApJ...749...91F,
Wex:1998wt,1991ApJ...379L..17N,Mandel:2015spa,Abbott:2016nhf}.
A binary system of black holes was recently observed by 
aLIGO~\cite{LIGOVirgo2016a}. Previously, stellar-mass black holes had only 
been observed by inference in mixed binaries with stellar companion (through
electromagnetic observations of the companion)~\cite{Lewin2010,
Remillard:2006fc,Fragos:2010tm}.
Neutron stars, on the other hand, have had numerous sightings. Thousands of
electromagnetically emitting neutron stars, or pulsars, have been 
documented~\cite{Manchester:2004bp},
in varied situations: as radio pulsars~\cite{Lattimer:2012nd,Manchester:2004bp},
in binary systems with a stellar companion~\cite{1971ApJ...169L..23M,
Bond:2002eh,Lattimer:2012nd,Manchester:2004bp},
and in binary neutron stars (BNS)~\cite{Hulse:1975uf,Taylor:1982wi,
Weisberg:2010zz,Lattimer:2012nd,Manchester:2004bp}.
Mixed binaries of black holes and neutron stars, is an astrophysically
interesting class of systems~\cite{Wex:1998wt,
1991ApJ...379L..17N,Janka1999,Fryer:2015jpa}, that has not yet been detected.
We expect to observe $\mathcal{O}(10)$ mixed binaries per year with
aLIGO~\cite{Abadie:2010cf}.


% \prayush{
% The actual masses of
% astrophysical black holes are uncertain, but observations and
% population synthesis studies suggest that BHs formed from stellar
% core-collapse can have masses up to and higher than
% $34M_\odot$~\cite{2008ApJ...678L..17S,2010ApJ...714.1217B}.  Also,
% recent measurements using continuum fitting and X-ray reflection
% fitting suggest that black holes can have high spin, with the BH
% angular momentum in dimensionless units exceeding
% $0.8$~\cite{2011ApJ...742...85G,2012MNRAS.424..217F,Gou:2013dna,
%   2009ApJ...697..900M,McClintock:2006xd,Miller:2009cw,McClintock:2013vwa}.
% }
% % 

% \red{
% Text describing BHNS disruption and accretion dynamics.
% }

% Most of the matter falls into the BH in ~1ms. But the disk itself has an
% accretion timescale of ~100ms-1s. The disk-BH system is the expected engine
% for a SGRB, through the production of a relativistic jet. The unbound material,
% on the other hand, is responsible for an EM signal in the infrared/optical, 
% powered by radioactive decays in the ejecta. And possibly a radio afterglow
% through interactions with the ISM. But not gamma-rays (unless you are referring
% to the very small amount of material ejected in the jet, but that's not what
% people usually mean when talking about material unbound by the merger).
% 


NSBH binaries are of interest for multiple reasons. For instance,
they have been long associated with (as possible progenitors of) short
Gamma-ray Bursts (SGRBs)~\cite{eichler:89,1992ApJ...395L..83N,moch:93,Barthelmy:2005bx,
2005Natur.437..845F,2005Natur.437..851G,Shibata:2005mz,Paschalidis2014,
Tanvir:2013}. Depending on their equation of state (EoS), NSs can get disrupted by
the tidal field of their companion BHs. Once disrupted, most of the NS
material falls into the hole over an $\mathcal{O}(1$ms$)$ time-scale,
with the rest partly getting ejected as unbound material
% (kilonovae, r-process material),
and partly forming an accretion disk around the BH.
% 
This short lived ($0.1-1s$) disk-BH system is hypothesized to drive SGRBs
through the production of relativistic jets~\cite{Foucart:2015a,
Lovelace:2013vma,Deaton2013,Foucart2012,Shibata:2005mz,Paschalidis2014}.
% 
However, whether or not such a system forms depends also on the nature of
the BH. Massive BHs (with $\mbh\gtrsim 10M_\odot$), as well as BHs with
large retrograde spins, tend to swallow the NS whole without forming a
disk~\cite{Foucart:2013psa}. 
% 
On the other hand, {\it low-mass} BHs with $\mbh\in[3M_\odot, 8M_\odot]$, 
can disrupt their companion NSs much before merger, forming long-sustained disks
that are required to sustain 
SGRBs~\cite{Shibata:2007zm,2010PhRvD..81f4026F,Lovelace:2013vma,Kawaguchi:2015}.
% 
% Such systems additionally leave a strong imprint on the emitted GW signal.
A {\it coincident} detection of both GWs and gamma-rays from an NSBH merger,
will
provide us with a unique opportunity to confirm this hypothesized link between
NSBH mergers and GRBs~\cite{Abbott:2016wya}.


% \red{
% Text focusing on tides
% }

Another question that compact object mergers can help answer is `what is the 
nature of matter at nuclear densities', i.e. `what is the EoS of NS material'?
A large fraction of past work aimed at measuring NS matter effects from GW
signals has consisted of inquiries about BNSs~\cite{Lee1999a,Lee1999b,Lee2000,
oechslin:07,Read:2008iy,Markakis:2010mp,Markakis:2011vd,stergioulas:11,
East:2011xa,Lackey2014,Wade:2014vqa,Bauswein:2014qla}. In this paper, we will
instead focus on NSBHs.
% 
During the course of early inspiral, the tidal field of the BH produces a
deformation in its companion NS. The quadrupolar moment of the star associated
with this deformation also depends on its material properties, through an EoS-dependent
tidal deformability parameter $\lambdans$. This induced quadrupolar moment
changes over the orbital time-scale, resulting in the emission of GWs in {\it
coherence} with the orbital waves.
These waves draw more energy from the orbit and increase the inspiral rate (as
compared to an equivalent BBH)~\cite{Flanagan2008}.
% The coherence of GW emission between stellar and orbital waves drains energy 
% more rapidly from the orbit and increases the inspiral rate (compared to a
% BBH)~\cite{Flanagan2008}.
% 
% We know the leading and next-to-leading order terms in post-Newtonian (PN)
% theory that capture this effect~\cite{Vines2011}, entering binary phasing at
% $5$PN order.
% 
Closer to merger, the strong tidal field of the BH can disrupt the NS. The
quadrupolar moment of the disrupted binary system falls monotonically over a
millisecond time-scale~\cite{Kyutoku:2010zd,Lackey:2013axa,Lovelace:2013vma,
Foucart:2015a,Pannarale:2015jia}, resulting in the damping of GW amplitude.
% 
This penultimate stage also depends strongly on the internal structure and energy
transport mechanism of the NS, and carries the strongest tidal signature in the
GW spectrum~\cite{Foucart:2014nda,Deaton2013}.

% This penultimate stage carries the strongest signature in the GW spectrum, and
% can possibly be accompanied by SGRBs. 
% The final stage is a BH, whose quasi-normal
% modes may (or not) be strongly excited~\cite{FoucartEtAl:2011,Lackey:2013axa}.



% During early inspiral, the (weak) tidal field of the BH deforms its companion
% NS.
% % , thereby exciting its resonant oscillation modes.
% The resulting induced quadrupolar moment of the star is proportional to the
% tidal field of its companion,
% % through the relation $Q_{ij} = -\lambda\mathcal{E}_{ij}$,
% with $\lambdans$ the EoS-dependent proportionality constant. $\lambdans$ is the
% NS's tidal deformability parameter, related to its dimensionless Love number
% $k_2$ and radius $R_\mathrm{NS}$ as $\lambdans:=\frac{2}{3G}k_2 R_\mathrm{NS}^5$.
% % 
% Over the course of inspiral, the change in NS's quadrupole moment results in 
% the emission of GWs that are coherent with the orbital waves.
% % 
% 



% \red{
% Insert paragraph describing impact on GWs
% }
% 
% \red{
% Make clear the distinction between Fischer matrix studies and Bayesian ones. 
% Explain the differences. this is important as it sets the stage for for what is
% new in this paper.
% } 
% - Describe how waves are affected in each stage for a NSBH. 
% - Describe what past work has studied and how?
% - Do we need to compare with BNS.. Maybe in passing, say `` In comparison, for BNS we expect to measure EoSs within 20 detections.

Gravitational waves emitted by coalescing NSBH binaries carry subtle hints of
the NS EoS from inspiral through to merger. During early inspiral, the tidal
dephasing is relatively weak and appears as a $5^{th}$ Post-Newtonian (PN)
order effect~\cite{Vines2011}. Closer to merger, a disruptive fate of the NS
can result in a strong suppression of GW emission above a cut-off 
frequency~\cite{Pannarale:2015jia}. Some past studies of tidal measurements
with NSBH binaries have used PN inspiral-only waveforms~\cite{Maselli:2013rza}.
In doing so, however, they ignore (i) the merger signal which could contain significant
information for NSBHs, and (ii) the errors due to unknown vaccum terms in PN 
waveforms, which could dominate over the tidal terms themselves~\cite{Barkett2015,
Yagi:2014}.
% 
Some other studies that account for merger effects via the use of complete
numerical simulations~\cite{Foucart:2013psa}, are limited in the binary
parameter space they sample.
% 
Others, that do the same through the use of phenomenological waveform
models~\cite{Lackey2011,Lackey:2013axa} use the Fisher matrix to estimate
$\lambdans$ measurement errors. Fisher matrix estimates are known to be
unreliable at realistic signal-to-noise ratios (SNR)~\cite{Vallisneri:2007ev},
such as those as we might expect in the upcoming observing runs of GW
detectors~\cite{Abadie:2010cf}.




% A large fraction of past work aimed at measuring NS matter effects from GW
% signals has consisted of inquiries about binary neutron stars (BNS)~\cite{
% Lee1999a,Lee1999b,Lee2000,oechslin:07,Read:2008iy,Markakis:2010mp,Markakis:2011vd,
% stergioulas:11,East:2011xa,Lackey2014,Wade:2014vqa,Bauswein:2014qla}.
% % 
% % It has shown~\cite{DelPozzo:13,Chatziioannou:2015uea,Agathos:2015a} that
% % with $20-30$ BNS observations, aLIGO would begin to distinguish between hard,
% % moderate and soft candidate equations of state.
% % 
% % These studies rely on the PN 
% % description of BNS dynamics, which is reasonable since BNSs merge at very high
% % frequencies and most of their signal power resides in inspiral.
% % 
% During early inspiral the tidally induced waves are weak in strength,
% as appropriate for an effect that enters binary phasing at $5^{th}$
% Post-Newtonian (PN) order~\cite{Vines2011}. During late-inspiral and merger,
% a disruptive fate of the NS could result in significant suppression of the
% emitted GWs~\cite{}.
% % 
% % The difficulty in discerning NS matter effects in an NSBH coalescence from its
% % GW signal lies in distinguishing it reliably from a BBH signal, since the 
% % difference between the two is either subtle (during inspiral) or short-lived
% % (disruption and merger).
% % 
% The earlier in inspiral the NS disrupts, the more orbits there are over which
% its effects are manifest in the GW signal.
% % 
% In addition, NSBHs merge at relatively lower frequencies than BNS, and their
% disruption during late-inspiral occurs at frequencies where aLIGO has
% relatively more resolving power. This, combined with a shorter inspiral,
% means that the merger portion of NSBH signals is relatively more valuable for
% measuring $\lambdans$. 


In this paper we study the measurability of neutron star's tidal deformability
from realistic binaries of {\it low}-mass BHs and NSs by aLIGO. We also probe
how tidal effects affect the estimation of other binary parameters for the same
class of systems. This study improves upon previous work in the following ways.
% 
First, we include tidal effects during inspiral and merger in a consistent
way, by using the waveform model of Lackey {\it et al.}~\cite{Lackey:2013axa}
(abbreviated henceforth to ``LEA'').
% 
Second, we include the effect of black hole spin on tidal GW signals, in
addition to the effect of BH mass, tidal deformability of the NS, and the SNR.
% 
Third, we perform a complete Bayesian analysis, instead of using the Fisher matrix
approximation.
% 
And fourth, we explore how our measurement errors decrease as we gain information
from multiple (realistic) events.
% 


We now outline the main questions and results discussed in this paper.
First, we probe the effect of ignoring tidal effects in
the recovery of non-tidal binary parameters, such as 
component masses and spins. This is the case for current and planned aLIGO
efforts.
To do so, we first use the enhanced-LEA (or ``LEA+'', see Sec.~\ref{s2:waveforms})
model to generate a set of realistic signals;
and then use non-tidal (BBH) waveform filters to estimate the underlying
binary masses and spins with a Markov-chain Monte Carlo. Here and throughout,
we use the zero-detuning high-power design sensitivity curve~\cite{Shoemaker2009}
to characterize the expected detector noise.
% 
% \red{
% MOVE the following to SUMMARY. Here, keep 1-2 sentences:
We find that, for individual events, ignoring tidal effects will affect mass
and spin-estimation only marginally; only for very loud signals (SNRs $\gtrsim 30$)
will the systematic biases be large enough to exceed the underlying
statistical uncertainty. Furthermore, detection searches can ignore tidal
effects without loss of sensitivity.
% }


% \red{
% Somewhat reduce discussion of results. Discuss in order of probability, starting
% with low SNR, ending with exceptional events.
% }

% \prayush{
Second, we study the ability of aLIGO to constrain neutron star tidal 
deformability with a single observation of an NSBH merger. For this, we
use the same setup for signal waveforms as before, but replace the filter
template model with one that includes tidal effects from inspiral
through to merger (i.e. LEA+)~\cite{Lackey:2013axa}. For most binaries with
BH masses outside of the mass-gap $(2-5M_\odot)$~\cite{Bailyn:1997xt,
Kalogera:1996ci,Kreidberg:2012,Littenberg:2015tpa} and/or realistic signal-to-noise
ratios (SNR), we find it difficult to put better than a factor of $2$ bound
on $\lambdans$ with a single observation. As we can see from
Fig.~\ref{fig:TT_LambdaCIWidths90_0_Lambda_SNR}, it is only at SNRs 
$\rho\gtrsim 20-30$ (under otherwise favorable circumstances, such as a stiff
equation of state) that we are able
to bring this down to a $\pm 75\%$ bound on $\lambdans$. For signals louder
than $\rho =30$, we can constrain $\lambdans$ to a much more meaningful degree
(within $\pm 50\%$ of its true value).
% For more realistic SNRs and BH parameters, a single measurement is 
% for $\lambdans$ is accompanied with a $\sim\pm150+\%$. uncertainty.
% }
While this is discouraging at first, we turn to ask: what if we combine
information from a population of low-SNR observations?



% \red{
% BEgin this paragraph by pointing out that EOS is universal among all NSs. 
% Therefore, one universal $\lambdans$ vs $\mns$ curve. Therefore, by combining
% multiple BHNS events, expect to measure $\lambdans$ better than with any 
% single BHNS event.
% }
% % \prayush{
The EoS of matter at nuclear densities is believed to be universal among all
neutron stars. The Tolman-Oppenheimer-Volkoff equation~\cite{Tolman:1939jz,
Oppenheimer:1939ne,1934PNAS...20..169T}
would then predict that NS properties satisfy a universal relationship
between $\lambdans$ and $\mns$. As the final part of
this paper, we combine information from multiple observations of realistic NSBH
systems and perform a fully-Bayesian analysis of how our estimation of
$\lambdans$ changes as we accumulate detections. This is similar to an earlier
study~\cite{DelPozzo:13} aimed at binary neutron stars.
% % }
% % 
% \red{
% MOVE into later sections. Keep here only 1-2 informational sentences.
% }
% % 
We restrict ourselves to a population of NSs with masses clustered very tightly
around $1.35M_\odot$ (with a negligible variance), and negligible spins. We
sample different nuclear EoSs by sampling entire populations fixing different
values for the NS tidal deformability.
% These values correspond to a hard EoS at the lower end, and a soft EoS at the upper.
For all populations, we take source locations to be uniformly distributed in
spatial volume, and source orientations to be uniform on the $2-$sphere. To
summarize, we find the following:
% 
(a) Our median estimate for $\lambdans$ starts out prior dominated, but 
converges to within $10\%$ of the true value within $10-20$ detections.
(b) Measurement uncertainties for $\lambdans$, on the other hand, depend on
$\lambdans$ itself. We find that for hard equations of state (with 
$\lambdans\geq 1000$), $10-20$ observations are sufficient to constrain
$\lambdans$ within $\pm 50\%$. For softer equations of state, the same level
of certainty would require substantially more ($25-40$) observations.
% 
(c) Further, if the astrophysical ``mass-gap''~\cite{Bailyn:1997xt,
Kalogera:1996ci,Kreidberg:2012,Littenberg:2015tpa} is real, we find that $20-50\%$
additional observations would be required to attain the same measurement
accuracy as above. And, (d) putting tighter constraints on the $\lambdans$ of a
population would require $50+$ NSBH observations, in any scenario.
% 
All of the above is possible within a few years of design
aLIGO operation~\cite{Abadie:2010cfa}.

% \red{
% MOVE to SUMMARY, only keeping 2-4 sentences here. Also, paradigm A/B barely
% matters. Therefore, first describe the common results, then point out how the
% presence/absence of mass gap modifies them.
% }
% 
% % \prayush{
% In addition, we probe two paradigms, one
% that allows for BH masses to lie within the astrophysical mass-gap (paradigm
% A), and one that does not (paradigm B).
% % 
% For paradigm A, we find the following: (i) for the softer equations of state
% that result in less deformable neutron stars, $15-20$ detections are sufficient
% for our median value of $\lambdans$ to lie within $10\%$ of the true
% value. (ii) For NSBH populations with more deformable NSs
% ($\lambdans\geq 1000$), the same is achievable within as few as $10$ realistic
% observations. (iii) The statistical uncertainty associated with $\lambdans$
% measurement can be restricted to be within $\pm50\%$ using $10-20$ observations
% when $\lambdans\geq 1000$), and using $25-40$ observations for softer equations
% of state. All of the above is possible within a few years of design
% aLIGO operation~\cite{Abadie:2010cfa}, if astrophysical BHs do indeed have
% masses between $\leq 5M_\odot$ (i.e. in the mass-gap). However, further
% restricting $\lambdans$ will require $50+$ NSBH observations.
% % 
% For paradigm B, we find the information accumulation to be somewhat slower.
% While the quantitative inferences for $\lambdans\geq1000$ populations are
% not affected significantly, we find $\mathcal{O}(10-20)\%$ more events are
% required to attain the same measurement accuracy as under paradigm A.
% % 
% We conclude that within as few as $20-30$ observations of disruptive NSBH
% mergers, aLIGO will begin to place interesting bounds on NS deformability.
% This, amongst other things, will allow us to rank different equations of 
% state for neutron star matter from most to least likely, with a few years'
% time-scale.
% % }




% \red{
% The following numerical details disrupt the flow of the text. MOVE after
% description of analyses.
% }

% \prayush{
In this paper, we restrict our parameter space to span mass-ratios
$q:=\mbh/\mns\in[2,5]$, BH spin (aligned with orbit)
$\chibh\in[-0.5, +0.75]$, and dimensionless NS tidal deformability 
$\lambdans:= G\left(\frac{c^2}{G \mns}\right)^5\lambda \in[500, 2000]$.
The mass of NS for signals (and not templates) is fixed to $1.35M_\odot$ 
throughout~\cite{stellarcollapsemass}. Similarly, their spin is taken to be
negligible~\cite{Miller:2014aaa}. Signal strength is sampled over a range of
SNR $\rho\in[10, 70]$.
% }
% 
Further, we note that our results are applicable for the design sensitivity
LIGO instruments ($2018-19$). 
We also note that the parameter space probed is somewhat restricted by the
domain of applicability of the LEA+ model which we use as filters. Finally,
the accuracy of our quantitative results depends on the reliability of the same
waveform model, which is the only one available of its kind in current
literature. A more recent work~\cite{Pannarale:2015jka} improves upon the 
amplitude description of LEA+, but needs to be augmented with a compatible
phase model. However, we expect our broad conclusions here to hold despite
the intrinsic inaccuracies of LEA+ (with errors not exceeding
$\mathcal{O}(10\%)$~\cite{Pannarale:2015jka}).
% 
% we expect the combined effect of modeling errors to not
% change our broad qualitative conclusions, which are based on statistical trends
% with large errors themselves.



The remainder of the paper is organized as follows. 
Sec.~\ref{s1:techniques} discusses data analysis techniques and resources 
used in this paper, such as the waveform model, and parameter estimation 
algorithm.
Sec.~\ref{s1:PEwithnoNS} discusses the consequences of ignoring tidal 
effects in parameter estimation waveform models.
Sec.~\ref{s1:PEwithNS} discusses the measurability for the leading order
tidal parameter $\lambdans$ at plausible SNR values.
Sec.~\ref{s1:multiple_observations} discusses the improvement in our
measurement of $\lambdans$ with successive (multiple) observations of
NSBH mergers.
Finally, in Sec.~\ref{s1:discussion} we summarize our results and discuss
future prospects with Advanced LIGO.





%%%%%%%%%%%%%%%%%%%%%%%%%%%%%%%%%%%%%%%%%%%%%%%%%%%%%%%%%%%%%%%%%%%%%%%%%%%%%%%
\section{Techniques}\label{s1:techniques}
%%%%%%%%%%%%%%%%%%%%%%%%%%%%%%%%%%%%%%%%%%%%%%%%%%%%%%%%%%%%%%%%%%%%%%%%%%%%%%%
% #################
% THESE FIGURES ARE FOR THE NEXT SECTION
\begin{figure*}
\centering 
\textbf{Illustrative posterior probability distributions for mass-ratio $\eta$,}\par
\textbf{at different SNR values (shown in legend).}\par\medskip
\includegraphics[width=1.8\columnwidth]{plots/SingleSystemEta_q4_0_mc2_25_chi0_50}\\
\caption{We show here probability distributions for mass ratio $\eta$ as measured
for the same signal at different SNRs. The intrinsic parameters of the source
are: $q = \mbh/\mns = 5.4M_\odot/1.35M_\odot = 4$, $\chibh=+0.5$, and $\lambdans=2000$;
and the signal is injected at SNRs $\rho=\{20,30,50\}$ (left to right). The templates
{\it ignore} tidal effects.
% 
In each panel: the dashed red line marks the median value
$\eta^\mathrm{Median}$, while the dashed green line show the true value
$\eta^\mathrm{Injected}$. The darker shading shows
the recovered $90\%$ credible interval for $\eta$, $(\Delta\eta)^{90\%}$.
% 
Comparing systematic and statistical errors, we find that:
at $\rho=20$, $\eta$ measurement is dominated by statistical
errors; at $\rho=30$, the two become comparable; and 
for louder signals ($\rho\simeq50$), the systematic errors dominate.
}
\label{fig:SingleSystemEtaPDFvsSNR}
\end{figure*}
% 
% ###############################
% In this section we summarize various technical aspects of this paper. The model
% used to obtain tidal waveforms is described in Sec.~\ref{s2:waveforms}. The 
% process of inferring source parameters from a GW signal is summarized in 
% Sec.~\ref{s2:bayesian}. The process of generating and combining multiple events
% is instead detailed in the corresponding results section,
% Sec.~\ref{s2:astro_multiple}.


\subsection{Waveform Models}\label{s2:waveforms}

% \red{
% Begin with introducing sentence:
% }

Lackey {\it et al.}~(LEA)~\cite{Lackey:2013axa} developed a complete inspiral-merger
waveform model for disrupting NSBHs. Theirs is a frequency-domain
phenomenological model that includes the effect of BH and NS masses and spins
$\{\mbh, \chibh, \mns\}\equiv\vec{\theta}$ and NS tidal deformability
$\lambdans$. It was calibrated to a suite of $134$ numerical relativity (NR)
simulations of NSs inspiraling into spinning BHs. The parameter space these
simulations span, which we assume to be the domain of validity for the LEA
model, includes NS masses $1.2M_\odot\leq\mns\leq 1.45M_\odot$,
mass-ratios $2\leq q\leq 5$, and BH spins $-0.5\leq\chibh\leq+0.75$.
They also sample a total of $21$ two-parameter nuclear EoSs to cover the
spectrum of NS deformability.
% 
% Amongst the varied quantities were (a) the
% equation of state for the NS, (b) NS mass, (c) mass-ratio and (d) black hole
% spins, taken as aligned with the orbit.
% A total of $21$ two-parameter EoSs were used for these simulations, which also 
% had $q\in[2, 5]$, $\chibh\in[-0.5, +0.75]$, and 
% $m_\mathrm{NS}\in[1.20, 1.45]M_\odot$. Based on this suite of simulations, the
% authors calibrated a frequency-domain waveform model for the inspiral and merger phasing of
% NSBH coalescences.
% 
The GW strain $\tilde{h}(f)$ per the LEA model can be written as
\begin{equation}
 \tilde{h}_\mathrm{NSBH}(f, \vec{\theta}, \lambdans) = \tilde{h}_\mathrm{BBH}(f, \vec{\theta})\,A(f, \vec{\theta}, \lambdans)\,e^{\ii \Delta\Phi(f, \vec{\theta}, \lambdans)},
\end{equation}
with NS spin $\chins=0$ identically. Here, $\tilde{h}_\mathrm{BBH}$ is
an underlying BBH waveform model. In the original LEA model,
this was taken to be the SEOBNRv1
model~\cite{Taracchini:2012} of the Effective-one-body (EOB)
family~\cite{Buonanno99}. The factor $A(\cdot)$ adjusts
the amplitude of the BBH model to match that of an NSBH merger of otherwise
identical parameters, with NS-matter effects parametrized by $\lambdans$.
During early inspiral this term is set to
unity, but is a sensitive function of $\lambdans$ close to merger. The term with
$\Delta\Phi$ corrects the waveform phasing. During inspiral,
$\Delta\Phi$ is set to the PN tidal phasing corrections,
at the leading and next-to-leading orders~\cite{Vines2011}; close to merger,
additional phenomenological terms are needed. Both $A$ and $\Delta\Phi$ are
calibrated to all $134$ available NR simulations.


In this paper we use LEA for our signal and template modeling, but switch the 
underlying BBH model to SEOBNRv2 (and refer to it as enhanced-LEA or
``LEA+'')~\cite{Taracchini:2013rva}. We using the reduced-order
frequency-domain version of SEOBNRv2, which has the additional benefit of
reducing computational cost~\cite{Purrer:2015tud}. We expect this enhancement
from LEA$\rightarrow$LEA+ to make our conclusions more robust because: (a) the 
SEOBNRv2 model is more accurate~\cite{Kumar:2015tha,Kumar:2016dhh}, and (b)
the differences between the two EOB models are caused by the
inaccuracies of SEOBNRv1 during the {\it inspiral} phase, many orbits before 
merger~\cite{Kumar:2015tha}.
Since LEA only augments inspiral phasing with PN tidal terms, our
change in the underlying BBH model does not change LEA's construction, {\it and}
increases the overall model accuracy during inspiral.
% 
Finally, we note that we approximate the full GW signal with its dominant
$l=|m|=2$ modes, that are modeled by LEA+. For use in future LIGO science
efforts, we have implemented the LEA+ model in the LIGO Algorithms
Library~\cite{LAL}.
% ~\footnote{The 
% $l=|m|=2$ multipoles of the GW strain, when decomposed in a spin $-2$ weighted
% spherical harmonic basis, contain more than $99\%$ of the total signal power}
% This approximation has the added advantage of making various angle parameters
% degenerate with source distance.


% % % % % % % % % % % % % % % % % % % % % % % % % % % % % % % % % % % % % % % 
% % % % % % % % % % % % % % % % % % % % % % % % % % % % % % % % % % % % % % % 
\subsection{Bayesian methods}\label{s2:bayesian}
% % Describe the emcee-based PE code
% \red{
% Section is immensely technical. Perform careful polishing pass to 
% improve flow and to make it more accessible.
% }


The process of measuring systematic and statistical measurement errors
involves simulating many artificial GW signals, and inferring source binary
parameters from them using Bayesian statistics.
We start with generating a signal waveform, using the model LEA+, and injecting
it in zero noise to obtain a stretch of data $d_n$. Source binary parameters
$\vec{\Theta}:=\vec{\theta}\cup\{\lambdans\}$ are reconstructed from this
injected signal. Using Bayes' theorem, their joint inferred probability
distribution of $\vec{\Theta}$ can be written	 as
\begin{equation}\label{eq:postprob}
 p(\vec{\Theta} | d_n, H) = \dfrac{p(d_n|\vec{\Theta}, H)\,p(\vec{\Theta} | H)}{p(d_n|H)}.
\end{equation}
Here, $p(\vec{\Theta} | H)$ is the {\it a priori} probability of binary parameters
$\vec{\Theta}$
taking particular values, given $H$ - which denotes all our collective knowledge,
except for expectations on binary parameters that enter
our calculations explicitly. Throughout this paper, we impose a prior that is
uniform in
individual component masses, BH spin, and the tidal deformability of the NS. In
addition, we restrict mass-ratios to $q\geq 2$, as LEA+ is not calibrated for 
$1\leq q\leq 2$. $p(d_n|\vec{\Theta}, H)$ is the
likelihood of obtaining the given stretch of data $d_n$ if we assume that a
signal parameterized by $\vec{\Theta}$ is buried in it, and is given by
\begin{equation}\label{eq:likelihood}
 p(d_n| \vec{\Theta}, H) \equiv \LL(\vec{\Theta}) = \mathcal{N}\, \mathrm{exp}[- \langle d_n - h | d_n - h\rangle ],
\end{equation}
where $h\equiv h(\vec{\Theta})$ is a filter template with parameters 
$\vec{\Theta}$, $\langle\cdot|\cdot\rangle$ is a suitably defined
detector-noise weighted inner-product~\footnote{The inner product
$\langle\cdot|\cdot\rangle$  is defined as~\cite{Shoemaker2009}
\begin{equation}
\langle a|b\rangle \equiv 4\,\mathrm{Re}\left[\int_0^\infty \dfrac{\tilde{a}(f) \tilde(b)(f)^*}{S_n(|f|)}\,\D f\right],
\end{equation}
where $\tilde{a}(f)$ is the Fourier transform of the finite time series $a(t)$,
and $S_n(|f|)$ is the one-sided amplitude spectrum of detector noise.}, and
$\mathcal{N}$ is an SNR-dependent normalization constant.
% \red{[MP: INSERT A LINE ABOUT APPROXIMATING $\LL \approx e^{\rho^2/2}$}
As in Ref.~\cite{Purrer:2015nkh} we use a likelihood that is maximized over
over the template norm
and we ignore extrinsic parameters that only enter in the template norm such
as distance, orientation and sky location. In addition, we also maximize over
coalescence time and phase.
The denominator in Eq.~\ref{eq:postprob} is the {\it a priori} probability of finding
the particular signal in $d_n$ and we assume that each injected signal is as
likely as any other. Having constructed the probability distribution function
$p(\vec{\Theta} | d_n, H)$, extracting the measured probability distribution
for a single parameter (say $\alpha$) involves integrating
\begin{equation}\label{eq:marginalize}
 p(\alpha | d_n, H) = \int\D \vec{\Theta}_\alpha\, p(\vec{\Theta} | d_n, H),
\end{equation}
where $\vec{\Theta}_\alpha$ is the set of remaining parameters, i.e.
$\vec{\Theta}_\alpha:=\vec{\Theta} - \{\alpha\}$.

We use the ensemble sampler Markov-chain Monte-Carlo algorithm implemented in
the {\tt emcee} package~\cite{emcee}, to sample the probability distribution 
$p(\vec{\Theta} | d_n, H)$. We run 100 independent chains, each of which is
allowed to collect 100, 000 samples and combine samples from chains that have
a Gelman-Rubin statistic~\cite{gelman1992} close to unity. This procedure yields
about 10,000 independent samples.
% 
% We perform Bayesian parameter estimation for tidal signals, using
% both tidal and non-tidal templates. The signal parameters are selected 
% to lie on a grid given by:
% $q=\{2,3,4,5\}\times\chibh=\{-0.5,0,0.5,0.75\}\times\lambdans=\{500,800,1000,1500,2000\}\times\rho=\{10,20,30,50,70\}$.
% Corresponding signal waveforms are generated using the model described in the 
% previous sub-section, and each injected into a separate zero noise data segment
% $d_n$. Templates can be either tidal or non-tidal (LEA+ with $\lambdans=0$).
% 
% Inferred from $d_n$, the joint probability distribution for binary parameters
% $\vec{\Theta}:=\vec{\theta}\cup\{\lambdans\}$ is given by
% \begin{equation}\label{eq:postprob}
%  p(\vec{\Theta} | d_n, H) = \dfrac{p(d_n|\vec{\Theta}, H)\,p(\vec{\Theta} | H)}{p(d_n|H)}.
% \end{equation}
% Here, $p(\vec{\Theta} | H)$ is the prior probability for binary parameters of
% taking particular values. In this paper, we use a uniform prior on the
% individual masses within pre-chosen ranges, as well as a uniform prior on the
% spin of the black hole. We set neutron star's spin $=0$, and its tidal
% deformability parameter $\lambdans$ is sampled uniformly from $[0, 4000]$. In
% addition, we exclude mass-ratios between $[1,2]$ from our prior. This is done
% because of model constraints for LEA+. The prior probability of obtaining a 
% particular realization of data $p(d_n|H)$ is absorbed into the overall 
% normalization. Finally, the first term in the numerator
% $p(d_n|\vec{\theta}, \lambdans, H)$ is the likelihood of obtaining the given
% stretch of data $d_n$ if we assume that a signal parameterized by
% $\vec{\theta}\cup\{\lambdans\}$ is buried in it, i.e.
% \begin{equation}\label{eq:likelihood}
%  \LL(\vec{\Theta}) \equiv p(d_n| \vec{\Theta}, H) = \mathcal{N} \mathrm{exp}[- \langle d_n - h | d_n - h\rangle ],
% \end{equation}
% where $h\equiv h(\vec{\Theta})$ is a template given by our waveform model LEA+,
% % and the detector noise weighted inner-product $\langle a|b\rangle$ is defined 
% % as
% and $\langle\cdot|\cdot\rangle$ is the detector noise weighted 
% inner-product~\footnote{%
% This inner product is defined as
% \begin{equation}
% \langle a|b\rangle \equiv 4\,\mathrm{Re}\left[\int_0^\infty \dfrac{\tilde{a}(f) \tilde(b)(f)^*}{S_n(|f|)}\,\D f\right],
% \end{equation}
% % 
% where $\tilde{a}(f)$ is the Fourier transform of the finite time series $a(t)$,
% and $S_n(|f|)$ is the one-sided amplitude spectrum of detector noise.}
% The 
% definition of $\LL$ in Eq.~\ref{eq:likelihood} assumes that the instrument
% noise is colored Gaussian. The marginalized probability distribution for any
% single binary parameter (say $\alpha$) can be obtained by integrating
% Eq.~\ref{eq:postprob} over all other parameters, i.e. 
% \begin{equation}
%  p(\alpha | d_n, H) = \int\D \vec{\Theta}_\alpha\, p(\vec{\Theta} | d_n, H),
% \end{equation}
% where $\vec{\Theta}_\alpha$ is the set of remaining parameters, i.e.
% $\vec{\Theta}_\alpha:=\vec{\Theta} - \{\alpha\}$.
% 
% 
% 
% We sample the probability distribution of Eq.~\ref{eq:postprob} using
% an ensemble sampler Markov-chain Monte-Carlo algorithm based on the {\tt emcee}
% package~. A total of $100$ independent walkers were used, each of which
% was allowed to collect $100,000$ samples. We measure the correlation time and keep
% only $20,000$ independent posterior points to further analyze the calculated 
% posterior probability distributions. 
% 
One simplification we make to mitigate computational cost is to set the
frequency sampling interval to $\Delta f=0.4$~Hz, which we find to be
sufficient for robust likelihoods calculations in zero noise~\cite{Purrer:2015nkh}.
We integrate
Eq.~\ref{eq:marginalize} to obtain marginalized probability distributions
for the NS tidal deformability parameter: $p(\lambdans|d_n,H)$. We will quote
the median value of this distribution as our {\it measured} value for
$\lambdans$, and the $90\%$ credible intervals associated with the distribution
as the statistical error-bars.
% 




%%%%%%%%%%%%%%%%%%%%%%%%%%%%%%%%%%%%%%%%%%%%%%%%%%%%%%%%%%%%%%%%%%%%%%%%%%%%%%%
\section{How is PE affected if we ignore NS matter effects?}\label{s1:PEwithnoNS}
%%%%%%%%%%%%%%%%%%%%%%%%%%%%%%%%%%%%%%%%%%%%%%%%%%%%%%%%%%%%%%%%%%%%%%%%%%%%%%%
% \begin{figure*}
% \centering 
% \textbf{Illustrative posterior probability distributions for mass-ratio $\eta$,}\par
% \textbf{at different SNR values (shown in legend).}\par\medskip
% \includegraphics[width=1.8\columnwidth]{plots/SingleSystemEta_q4_0_mc2_25_chi0_50}
% \caption{We show here probability distributions for mass ratio $\eta$ as measured
% for the same injection at different SNRs. The intrinsic parameters of the source
% are: $q = \mbh/\mns = 5.4M_\odot/1.35M_\odot = 4$, $\chibh=+0.5$, and $\lambdans=2000$;
% and the signal is injected at SNRs $\rho=\{20,30,50\}$ (left to right). The templates
% {\it ignore} tidal effects.
% % 
% In each panel: the dashed red line marks the median value
% $\eta^\mathrm{Median}$, while the dashed green line show the true value
% $\eta^\mathrm{Injected}$. The difference between the true and median values
% is the systematic bias introduced
% by ignoring tidal effects in templates. The darker shading shows
% the recovered $90\%$ credible interval for $\eta$, $(\Delta\eta)^{90\%}$, 
% which is a direct measure of our statistical uncertainty on $\eta$.
% % 
% Comparing systematic and statistical errors, we find that:
% at $\rho=20$, $\eta$ measurement is dominated by statistical
% errors; at $\rho=30$, the two become comparable; and 
% for louder signals ($\rho\simeq50$), the systematic errors dominate.
% }
% \label{fig:SingleSystemEtaPDFvsSNR}
% \end{figure*}
% ################# FIXME FIXME
% \begin{figure*}
% \centering 
% \textbf{Statistical measurement uncertainty for NSBH parameters $\mchirp$ (top),
% $\eta$ (middle) and $\chibh$ (bottom);\newline ignoring tidal effects.}\par\medskip
% \includegraphics[trim = {2cm 0 0 0},width=2.1\columnwidth]{plots/TNMchirpCIWidths90_0_Lambda_SNR}\\
% \includegraphics[trim = {2cm 0 0 0},width=2.1\columnwidth]{plots/TNEtaCIWidths90_0_Lambda_SNR}\\
% \includegraphics[trim = {2cm 0 0 0},width=2.1\columnwidth]{plots/TNChiBHCIWidths90_0_Lambda_SNR}
% \caption{
% We show here the statistical uncertainty of our measurement of the 
% dimensionless mass ratio $\eta$ (at $90\%$ credibility), over the NSBH
% parameter space.
% Individual panels show the same as a function of BH mass and spin.
% Across each row, we see the effect of increasing signal strength (i.e.
% SNR) with the tidal deformability of the NS $\lambdans$ fixed. Down each
% column, we see the effect of increasing $\lambdans$, at fixed SNR.
% }
% \label{fig:CIWidths90_Lambda_SNR}
% \end{figure*}
% #################
% 
\begin{figure*}
\centering 
\textbf{Ratio of systematic to statistical errors in measuring $\mchirp$;}\par
\textbf{ignoring tidal effects.}\par\medskip
\includegraphics[width=1.6\columnwidth]{plots/TNMchirpBiasesOverCIWidths_CI90_0_Lambda_SNR30_70_linear}
\includegraphics[width=0.8\columnwidth]{plots/TNMchirpBiasesOverCIWidthsVsSNR_Lambda500_CI90_0}
% \includegraphics[width=0.67\columnwidth]{plots/TNMchirpBiasesOverCIWidthsVsSNR_Lambda800_CI90_0}
\includegraphics[width=0.8\columnwidth]{plots/TNMchirpBiasesOverCIWidthsVsSNR_Lambda1000_CI90_0}\\
\includegraphics[width=0.8\columnwidth]{plots/TNMchirpBiasesOverCIWidthsVsSNR_Lambda1500_CI90_0}
\includegraphics[width=0.8\columnwidth]{plots/TNMchirpBiasesOverCIWidthsVsSNR_Lambda2000_CI90_0}
\caption{We show here the ratio of systematic and statistical
measurement uncertainties for the binary chirp mass over the NSBH parameter 
space. Each panel shows the same as a function of BH mass and spin. NS mass
is fixed at $\mns=1.35M_\odot$, and its spin is set to zero. Down each column,
we can see the effect of the increasing tidal deformability of the NS at fixed
SNR. Across each row, we can see the effect of increasing the signal strength
(SNR), with the tidal deformability of the NS fixed. We also show dashed
contours for $\arr_{\mchirp}=10\%$, $50\%$, or $100\%$ (labeled
without the ``$\%$'' symbol).
% 
For BBHs, the statistical errors dominate systematic ones for contemporary
waveform models~\cite{Inprerp-LVC-WaveModels:2016,Kumar:2016dhh}. We find that
its not much different for NSBH binaries, until we get to very high SNRs
$\rho\gtrsim 70$.
}
\label{fig:TN_chirpMassBias_vs_Lambda_SNR}
\end{figure*}
%
\begin{figure*}[!t]
\centering
\textbf{Ratio of systematic to statistical errors in measuring $\eta$;}\par
\textbf{ignoring tidal effects.}\par\medskip
\includegraphics[width=2\columnwidth]{plots/TNEtaBiasesOverCIWidths_CI90_0_Lambda_SNR20_70_linear}
\caption{This figure is similar to Fig.~\ref{fig:TN_chirpMassBias_vs_Lambda_SNR}
with the difference that here we show the ratio of systematic and statistical
error sources for the symmetric mass-ratio $\eta$ and not chirp mass. We find
that for fairly loud GW signals, at $\rho\simeq 50$, not including the
effects of tidal deformation of the NS on GW emission can become the dominant
source of error for astrophysical searches with Advanced LIGO. However, for
quieter signals with $\rho\leq 30$, it will have a negligible effect on the
measurement of $\eta$. We remind the reader that the SNRs here are always
single detector values.
}
\label{fig:TN_EtaBias_vs_Lambda_SNR}
\end{figure*}
% 
% 
\begin{figure*}
\centering
\textbf{Ratio of systematic to statistical errors in measuring $\chibh$;}\par
\textbf{ignoring tidal effects.}\par\medskip
\includegraphics[width=1.95\columnwidth]{plots/TNChiBHBiasesOverCIWidths_CI90_0_Lambda_SNR_linear}
% \includegraphics[width=\columnwidth]{plots/TNChiBHBiasesOverCIWidthsVsSNR_All_CI68_3.pdf}
% \includegraphics[width=\columnwidth]{plots/TNChiBHBiasesOverCIWidthsVsSNR_All_CI90_0.pdf}
\caption{This figure shows the ratio of the systematic and statistical
measurement errors for BH spins, with other attributes identical to 
Fig.~\ref{fig:TN_chirpMassBias_vs_Lambda_SNR}, and~\ref{fig:TN_EtaBias_vs_Lambda_SNR}.
Similar to the case of mass parameters, we find that below $\rho\approx 30$,
ignoring tidal effects in templates introduces minor systematic effects,
which remain sub-dominant to the statistical measurement uncertainties.
}
\label{fig:TN_BHspinBias_vs_Lambda_SNR}
\end{figure*}
%
% ##############################################################
% 
\begin{figure*}
\centering 
\textbf{Illustrative posterior probability distributions for NS tidal
deformability $\lambdans$,}\par
\textbf{for signals with different $\lambdans$ values; $\lambdans = 1000,1500,2000,$
from left to right.}\par\medskip
% \includegraphics[width=1.8\columnwidth]{plots/SingleSystem_q4_0_mc2_25_chi0_50}
\includegraphics[width=1.9\columnwidth]{plots/SingleSystemLambdaVary_q4_0_mc2_25_chi0_50_snr50}
\caption{We show here probability distributions recovered for the NS tidal
deformability parameter $\lambdans$ from three GW injections, with parameters:
$q = \mbh/\mns = 5.4M_\odot/1.35M_\odot = 4$, $\chibh=+0.5$, and 
$\lambdans=\{1000,1500,2000\}$ from left to right. The injection SNR is fixed at
$\rho=50$. The templates {\it include} tidal effects, with a prior $0\leq\lambdans\leq 4000$.
% These figures show that this approximation holds up to (SNR)
% $\rho\simeq 30-50$.
% 
In each panel- the dashed red line marks the median value for
$\lambdans$, and the dashed green line marks its {\it true} value.
The darker shading shows the $90\%$ credible interval, whose width
$(\Delta\lambdans)^{90\%}$ is a direct measure of our statistical uncertainty.
By comparing the measurement uncertainty for these three injections, we see
that $(\Delta\lambdans)^{90\%}$ grows very slowly with $\lambdans$. Therefore,
the fractional measurement error - $(\Delta\lambdans)^{90\%}/\lambdans$ -
decreases monotonically as $\lambdans$ increases (with signal strength fixed).
% 
% Comparing systematic and statistical errors, we find that:
% at $\rho=20$, $\lambdans$ measurement is dominated by statistical
% errors; at $\rho=30$, the two become comparable; and 
% for louder signals ($\rho\simeq50$), the systematic errors dominate.
}
\label{fig:SingleSystemLambdaPDFvsSNR}
\end{figure*}
% %% FIXME FIXME
% \begin{figure*}
% \centering
% \textbf{Illustrative posterior probability distributions for NSBH parameters,}\par
% \textbf{for signals at different SNR values; $\rho = 20$ (left) and $50$ (right).}\par\medskip
% \includegraphics[width=1.05\columnwidth,trim=2cm 0 0 0]{plots/AllParamsMcEtPDF1D2D_q4_mc2_25_chi0_50_snr20}%\\
% \includegraphics[width=1.05\columnwidth,trim=2cm 0 0 0]{plots/AllParamsMcEtPDF1D2D_q4_mc2_25_chi0_50_snr50}%\\
% % \includegraphics[width=0.5\columnwidth]{plots/McEta2D_q4_mc2_25_chi0_50_snr30}
% % \includegraphics[width=0.5\columnwidth]{plots/EtaChiBH2D_q4_mc2_25_chi0_50_snr30}
% % \includegraphics[width=0.5\columnwidth]{plots/McChiBH2D_q4_mc2_25_chi0_50_snr30}\\
% % \includegraphics[width=0.5\columnwidth]{plots/LambdaMc2D_q4_mc2_25_chi0_50_snr30}
% % \includegraphics[width=0.5\columnwidth]{plots/LambdaEta2D_q4_mc2_25_chi0_50_snr30}
% % \includegraphics[width=0.5\columnwidth]{plots/LambdaChiBH2D_q4_mc2_25_chi0_50_snr30}\\
% % \includegraphics[width=0.47\columnwidth]{plots/LambdaMc2D_q4_mc2_25_chi0_50_snr50}
% % \includegraphics[width=0.47\columnwidth]{plots/LambdaEta2D_q4_mc2_25_chi0_50_snr50}
% % \includegraphics[width=0.47\columnwidth]{plots/LambdaChiBH2D_q4_mc2_25_chi0_50_snr50}
% \caption{%
% We illustrate here two-dimensional joint probability distributions, as
% recovered for the intrinsic parameters of a binary with
% $q = \mbh/\mns = 5.4M_\odot/1.35M_\odot = 4$, $\chibh=+0.5$,
% $\lambdans=2000$, and $\rho=30$ (same as in
% Fig.~\ref{fig:SingleSystemLambdaPDFvsSNR}).
% The top row shows the same for combinations of 
% non-tidal parameters, i.e. binary masses and spins, while the middle row 
% shows combinations of $\lambdans$ with non-tidal parameters. The bottom row
% is similar to the middle, with the only difference being that the SNR has been
% turned up to $\rho=50$.
% }
% \label{fig:SingleSystemLambda2DPDFs}
% \end{figure*}
% #################
% \begin{figure*}
% \centering    
% \textbf{Statistical uncertainty in $\lambdans$ measurement}\par\medskip
% \includegraphics[trim={2cm 0 0 0},width=2.2\columnwidth]{plots/TTLambdaRawCIWidths90_0_Lambda_SNR}
% % \includegraphics[trim={3cm 0 0 0},width=2.\columnwidth]{plots/TTLambdaCIWidths90_0_Lambda_SNR.pdf}
% \caption{Here we show the statistical uncertainty in the measurement of
% $\lambdans$, as a percentage of the injected/true value. In each panel, the
% same is shown as a function of the BH mass and spin, keeping $\lambdans$ and
% injection's SNR $\rho$ fixed (noted in the panel). Rows contain panels
% with the same value of $\lambdans$, with $\rho$ increasing from left to right.
% Each column contains panels with the same value of $\rho$, with $\lambdans$ 
% increasing from top to bottom.
% % 
% Contours demarcate regions where we can constrain the
% $\lambdans$ parameter well (within a factor of two of the injected value).
% % 
% We note that, as expected, the measurement accuracy for $\lambdans$ improves with (i) increasing
% SNR, (ii) decreasing BH mass, (iii) increasing BH spin, and 
% (iv) increasing $\lambdans$, i.e. the tidal deformability of the neutron star.
% }
% \label{fig:TT_LambdaCIWidths90_0_Lambda_SNR}
% \end{figure*}
% %
% 
% 
Past (and future) efforts with Advanced LIGO have used (or plan to use) BBH
waveform templates to search for and characterize NSBH sources. In doing so,
they ignore the signature of NS tidal effects on the emitted GWs. In this
section we present a fully Bayesian analysis of the effect of this
simplification on the recovery of non-tidal parameters from NSBH signals.


We inject LEA+ NSBH signals into zero noise, and run an MCMC sampler on
them using equivalent BBH templates (same model, tidal terms $\rightarrow 0$).
We fix $\mns=1.35M_\odot$ and $\chins=0$, and explore a range of NS
equations of state via the single tidal deformability parameter
$\lambdans\in\{500, 800, 1000, 1500, 2000\}$. Our injections also span a
rectangular grid in the BH parameter space, with vertices at
$q\in\{2,3,4,5\}$, i.e.
$\mbh\in\{2.7M_\odot, 4.05M_\odot, 5.4M_\odot, 6.75M_\odot\}$, and BH spins
$\chibh\in\{-0.5, 0, +0.5, +0.75\}$. 
Finally, we sample all other source-related parameters, that determine the
signal strength but not character\footnote{For aligned-spin signals and
aligned-spin templates both, we only consider the contribution of the dominant
$l=|m|=2$ waveform multipoles. This approximation has the additional benefit
of combining the dependence of the waveforms on inclination, polarization
and sky location angles, as well as on distance, into the luminosity
or {\it effective} distance. This quantity only appears as an overall scaling
factor, and therefore only affects signal strength~\cite{Sathyaprakash:2009xs}.
}, by sampling the SNR $\rho\in\{20,30,50,70\}$. Our choice of injection
parameters here
is motivated by two factors: (i) previous studies of the signatures of NS tidal
effects on gravitational waves~\cite{FoucartEtAl:2011,Foucart:2013psa,
Foucart:2014nda} (which suggest that necessary conditions for the observation
of tidal effects with aLIGO include high SNRs and a low-mass spinning companion
BH); and (ii) technical constraints of our chosen LEA+ model~\cite{Lackey:2013axa}.
At design sensitivity, if we expect $0.2-300$ NSBH detections a 
year~\cite{Abadie:2010cfa}, we can expect to see $0.02-25$ 
{\it disruptive}\footnote{We assume here that BH mass values are {\it uniformly}
likely from $2M_\odot$ to $\sim 35M_\odot$~\cite{LIGOVirgo2016a}, but NSs are
disrupted in NSBH mergers only if $q\leq 6$ and $\chibh\geq 0$~\cite{Foucart:2014nda,
Foucart:2013psa}.
} NSBH mergers a year, of which we will have $0.005-7$ observations with 
$\rho\geq 20$, and $0.002-3$ a year with $\rho\geq 30$.
Therefore, our injection parameters span a physically interesting subset of NSBH
binaries, that is {\it also} likely observable in the near future. 
For our Bayesian priors, we choose uniform distributions for both component
masses and black hole spin: $\mbh\in[1.2,25]M_\odot$; $\mns\in[1.2,3]M_\odot$;
and $-0.75\leq \chibh\leq +0.75$.



As an illustration, in Fig.~\ref{fig:SingleSystemEtaPDFvsSNR} we show the
recovered probability distributions for binary mass ratio $\eta$ for
three NSBH injections, with $\rho=20$ (left), $30$ (middle), and $\rho=50$
(right), and other parameters held fixed ($\mns=1.35M_\odot$, $\chins=0$,
$\mbh=5.4M_\odot$, $\chibh=+0.5$ and $\lambdans=2000$). In each panel, both
the {\it true} and the median values of $\eta$ are marked.
The effect of ignoring tidal corrections in templates manifests as a systematic
shift of the median $\eta$ value away from its true value. We measure
the associated {\it systematic} bias/error by the differences between the
true and median $\eta$ values. Darker
shading of the probability distributions marks the $90\%$ credible interval,
whose width $(\Delta\eta)^{90.0\%}$ is a direct measure of the
{\it statistical} measurement uncertainty/error\footnote{We generalize the
notation $(\Delta X)^{90.0\%}$ to mean the $90\%$ credible interval width
for any measured source parameter $X$.}.
% 
We see clearly that even when the signal is moderately loud, with $\rho=20$,
statistical errors dominate over systematics for $\eta$. As we turn up the SNR
further, we find that the two error sources become comparable at $\rho\sim 30$,
and systematic errors dominate finally when $\rho\simeq 50$.


% \textcolor{gray}{%
% Let us first delve into statistical errors. In
% Fig.~\ref{fig:CIWidths90_Lambda_SNR}, we show how {\it precisely} can we
% measure NSBH parameters $X=\{\mchirp,\eta,\chibh\}$ using BBH templates.
% The three panels correspond to $\mchirp$ (top), $\eta$ (middle), and $\chibh$
% (bottom), and show the width of these credible intervals $(\Delta X)^{90\%}$
% as a function of BH mass/spin (within each sub-panel), and NS properties, i.e.
% $\lambdans$ (downwards in each column)~\footnote{We restrict NS mass to
% $1.35M_\odot$ and its spin to zero. Varying its tidal deformability $\lambdans$
% does not significantly change the measurement uncertainties for non-tidal
% binary parameters, as is evident from comparing the two rows in each panel of
% Fig.~\ref{fig:CIWidths90_Lambda_SNR}.}.
% % 
% From the left-most column, we find that: (i) at $\rho=20, \mchirp$ is already
% measured remarkably well - to a precision of $0.16\%$ of its true value, and
% (ii) so is $\chibh$. (iii) Mass ratio $\eta$ is determined more loosely,
% with $\gtrsim 25\%$ uncertainty. If the signal is even louder ($\rho\geq 30$),
% all three measurements gain further precision, with $\eta$ errors shrinking down
% to single-digit percents of its true value.
% % % 
% % We remind ourselves that this {\it precision}, however, is only
% % meaningful so long as the measurement is {\it accurate} to begin with.
% % We define $\arr_X$ as the ratio between systematic and statistical
% % errors associated with the measurement of parameter $X$,
% % \begin{equation}\label{eq:arr}
% % \arr_X = \dfrac{(X^\mathrm{Median} - X^\mathrm{Injected})}{(\Delta X)^{90\%}},
% % \end{equation}
% % in order to compare the relative importance of both. Only when
% % $|\arr_X| < 1$ can we hope that ignoring tidal effects in our templates
% % will not hamper the measurement of non-tidal parameters from NSBH signals.
% % Due to
% % unmodeled tidal effects, parameter measurements performed on NSBH signals with
% % BBH templates will have systematic biases. If these biases are too large,
% % using non-tidal BBH templates should severely impair our inferencing ability.
% % 
% % there are physical effects that have not been
% % incorporated in our (BBH) search templates which we use here to decipher NSBH
% % signals. Therefore, we cannot know the {\it accuracy} of our parameter
% % measurements unless we know how large the systematic biases due to an
% % incomplete template model are. To quantify this, we find it useful to define
% % $\arr_X$, the ratio of the systematic biases induced by unmodeled tidal effects in
% % templates, to the statistical uncertainties associated with the measurement
% % itself. If $\arr_X\gtrsim 1$, including the effects of NS tidal distortion
% % (during inspiral) and possible disruption (close to merger) would be imperative
% % to obtain {\it accurate} estimates of source masses and spins from observed GW
% % signals.}
% }


%\textcolor{blue}$ showing the precision with which
$X=\{\mchirp,\eta,\chibh\}$ can be measured, are presented in
Appendix~\ref{as1:nontidalerrors}.
%}
We remind ourselves that this {\it precision} is only meaningful so long as the
measurement is {\it accurate} to begin with. Therefore, we define $\arr_X$ as
the ratio between systematic and statistical errors associated with the
measurement of parameter $X$,
\begin{equation}\label{eq:arr}
\arr_X = \dfrac{(X^\mathrm{Median} - X^\mathrm{Injected})}{(\Delta X)^{90\%}},
\end{equation}
in order to compare the relative magnitude of both. Only when
$|\arr_X| \ll 1$ can we ignore tidal effects in our templates
without hampering the measurement of non-tidal parameters from NSBH signals.
When $\arr_X$ approaches a few tens of percent of unity, we can begin to favor
tidal templates for NSBH studies.


We start with calculating $\arr_{\mchirp}$ as a function of various source
parameters and show it in Fig.~\ref{fig:TN_chirpMassBias_vs_Lambda_SNR}.
$\mchirp$ is the leading order
mass combination that affects the GW strain emitted by compact binaries as they
spiral in, and is therefore determined the most precisely. We notice
immediately that for $\rho\leq 30$ the systematics are well under control
and so we can obtain reliable chirp mass estimates for NSBH signals using BBH
templates.
% 
For louder and less likely SNRs ($\rho\simeq 50$), we find that
$\arr_{\mchirp}$ can become comparable to unity, but only if the BH has
prograde spin $\chibh\gtrsim 0.4$, {\it and} the true NS tidal deformability
is large enough, s.t. $\lambdans \gtrsim 1000$.
% 
We can therefore conclude that only for very loud signals, with
$\rho\gtrsim 50-70$, will the inclusion of tidal terms in our template
models improve our $\mchirp$ estimation. For lower SNRs, inclusion of new
physical content in templates will instead get washed out by detector noise.
% 
In addition, we also note that $\arr_{\mchirp}\geq 0$ always,
i.e. $\mchirp$ is always being over-estimated. This is to be expected since
the tidal deformation and disruption of NSs close to merger reduces GW signal
power at high frequencies, making the resulting signal resemble a BBH signal of
lower merger frequency - and therefore of higher mass.


Next, in Fig.~\ref{fig:TN_EtaBias_vs_Lambda_SNR}, we show the ratio of
measurement errors $\arr_\eta$ for the symmetric mass-ratio.
% 
Going through the figure from left to right, we find that for realistic
SNRs ($\rho\leq 30$) the systematics remain below statistical errors for
$\eta$ measurement. The worst case is of the most deformable NSs
($\lambdans = 2000$), but even for them systematics in $\eta$ are $2\times$
smaller than the statistical measurement errors.
% 
Moving to louder signals with $\rho\simeq 50$, we find that for binaries of
fairly deformable NSs ($\lambdans\gtrsim 1500$) and low-mass BHs
($\mbh\leq 5M_\odot$) that have prograde spins ($\chibh\gtrsim +0.4$), our
measurement of mass-ratio can be seriously compromised by ignoring tidal
physics in template models. 
% 
This pattern is continued at even higher SNRs, as we can see from
Fig.~\ref{fig:TN_EtaBias_vs_Lambda_SNR}. We therefore conclude that, even if
under moderate restrictions on BH and NS parameters, $\rho=30-50$ is loud enough
to motivate the use of tidal templates in aLIGO data analyses.
% 
In addition, we also notice that, unlike for $\mchirp$, the median value of
$\eta$ is always {\it lower} than its true value, which is what we expect if we
want BBH templates to fit NSBHs that disrupt and merge at lower frequencies.



Moving on from mass to spin parameters, we now consider the measurement of BH
spin angular momentum $\chibh$. The ratio of systematic and statistical errors
for $\chibh$ are shown in
Fig.~\ref{fig:TN_BHspinBias_vs_Lambda_SNR}. The presentation of information in this 
figure is identical to that of Fig.~\ref{fig:TN_chirpMassBias_vs_Lambda_SNR}
and~\ref{fig:TN_EtaBias_vs_Lambda_SNR}. A diverging colormap is used so that both 
extremes of the colorbar range point to large systematic biases, while its zero (or
small) value lies in the middle.
% 
For the lowest SNR considered ($\rho=30$), $\chibh$ bias is about $2\times$
smaller than its statistical measurement uncertainty, and is therefore mostly
negligible. Both do become somewhat comparable, but only when we have the most
deformable NSs in orbit around low-mass BHs. 
% 
At higher SNRs $(\rho\simeq50-70)$, we find that the systematics in $\chibh$
measurement can dominate completely, especially for binaries containing
mass-gap violating BHs and/or deformable NSs with $\lambdans\geq1000$.
% 
From Fig.~\ref{fig:TN_BHspinBias_vs_Lambda_SNR} we additionally note that when
the source spin magnitudes approach the highest allowed, i.e. at both extremes
of the $x$-axes, $\chibh\times\arr_{\chibh}<0$. This is to be expected because
the median of the recovered posterior distributions for $\chibh$ can only get
pushed inwards from the boundaries.



Summarizing these results, we find that irrespective of system parameters,
below a signal-to-noise ratio of $30$, our measurements of mass and spin
parameters of astrophysical NSBH binaries will remain limited by the intrinsic
uncertainty due to instrument noise, and do not depend on whether we include
tidal effects in template models. However, when the signal-to-noise ratio
exceeds $30$ the systematic bias in binary mass and spin measurements become
comparable to and can exceed the uncertainty due to noise. Of the different
non-tidal parameters considered, we find that the measurement of $\eta$
degrades worst (in a relative-error sense) due to the use of BBH templates in
deciphering an NSBH signal. Of all the sub-categories, we find that tidal
templates could especially help with the parameter estimation of astrophysical
{\it mass-gap violating} NSBH binaries,







%%%%%%%%%%%%%%%%%%%%%%%%%%%%%%%%%%%%%%%%%%%%%%%%%%%%%%%%%%%%%%%%%%%%%%%%%%%%%%%
\section{What do we gain by using templates that include NS matter effects?}\label{s1:PEwithNS}
%%%%%%%%%%%%%%%%%%%%%%%%%%%%%%%%%%%%%%%%%%%%%%%%%%%%%%%%%%%%%%%%%%%%%%%%%%%%%%%
% % 
% \begin{figure*}
% \centering 
% \textbf{Illustrative posterior probability distributions for NS tidal
% deformability $\lambdans$,}\par
% \textbf{for signals with different $\lambdans$ values; $\lambdans = 1000,1500,2000,$
% from left to right.}\par\medskip
% % \includegraphics[width=1.8\columnwidth]{plots/SingleSystem_q4_0_mc2_25_chi0_50}
% \includegraphics[width=1.9\columnwidth]{plots/SingleSystemLambdaVary_q4_0_mc2_25_chi0_50_snr50}
% \caption{We illustrate here probability distributions recovered for the NS tidal
% deformability parameter $\lambdans$ from three GW injections, with parameters:
% $q = \mbh/\mns = 5.4M_\odot/1.35M_\odot = 4$, $\chibh=+0.5$, and 
% $\lambdans=\{1000,1500,2000\}$ from left to right. The injection SNR is fixed at
% $\rho=50$. The templates {\it include} tidal effects, with a prior $0\leq\lambdans\leq 4000$.
% % These figures show that this approximation holds up to (SNR)
% % $\rho\simeq 30-50$.
% % 
% In each panel- the dashed red line marks the median value for
% $\lambdans$, and the dashed green line marks its {\it true} value.
% The darker shading shows the $90\%$ credible interval, whose width
% $(\Delta\lambdans)^{90\%}$ is a direct measure of our statistical uncertainty.
% By comparing the measurement uncertainty for these three injections, we see
% that $(\Delta\lambdans)^{90\%}$ grows very slowly with $\lambdans$. Therefore,
% the fractional measurement error - $(\Delta\lambdans)^{90\%}/\lambdans$ -
% decreases monotonically as $\lambdans$ increases (with signal strength fixed).
% % 
% % Comparing systematic and statistical errors, we find that:
% % at $\rho=20$, $\lambdans$ measurement is dominated by statistical
% % errors; at $\rho=30$, the two become comparable; and 
% % for louder signals ($\rho\simeq50$), the systematic errors dominate.
% }
% \label{fig:SingleSystemLambdaPDFvsSNR}
% \end{figure*}
% % %% FIXME FIXME
% % \begin{figure*}
% % \centering
% % \textbf{Illustrative posterior probability distributions for NSBH parameters,}\par
% % \textbf{for signals at different SNR values; $\rho = 20$ (left) and $50$ (right).}\par\medskip
% % \includegraphics[width=1.05\columnwidth,trim=2cm 0 0 0]{plots/AllParamsMcEtPDF1D2D_q4_mc2_25_chi0_50_snr20}%\\
% % \includegraphics[width=1.05\columnwidth,trim=2cm 0 0 0]{plots/AllParamsMcEtPDF1D2D_q4_mc2_25_chi0_50_snr50}%\\
% % % \includegraphics[width=0.5\columnwidth]{plots/McEta2D_q4_mc2_25_chi0_50_snr30}
% % % \includegraphics[width=0.5\columnwidth]{plots/EtaChiBH2D_q4_mc2_25_chi0_50_snr30}
% % % \includegraphics[width=0.5\columnwidth]{plots/McChiBH2D_q4_mc2_25_chi0_50_snr30}\\
% % % \includegraphics[width=0.5\columnwidth]{plots/LambdaMc2D_q4_mc2_25_chi0_50_snr30}
% % % \includegraphics[width=0.5\columnwidth]{plots/LambdaEta2D_q4_mc2_25_chi0_50_snr30}
% % % \includegraphics[width=0.5\columnwidth]{plots/LambdaChiBH2D_q4_mc2_25_chi0_50_snr30}\\
% % % \includegraphics[width=0.47\columnwidth]{plots/LambdaMc2D_q4_mc2_25_chi0_50_snr50}
% % % \includegraphics[width=0.47\columnwidth]{plots/LambdaEta2D_q4_mc2_25_chi0_50_snr50}
% % % \includegraphics[width=0.47\columnwidth]{plots/LambdaChiBH2D_q4_mc2_25_chi0_50_snr50}
% % \caption{%
% % We illustrate here two-dimensional joint probability distributions, as
% % recovered for the intrinsic parameters of a binary with
% % $q = \mbh/\mns = 5.4M_\odot/1.35M_\odot = 4$, $\chibh=+0.5$,
% % $\lambdans=2000$, and $\rho=30$ (same as in
% % Fig.~\ref{fig:SingleSystemLambdaPDFvsSNR}).
% % The top row shows the same for combinations of 
% % non-tidal parameters, i.e. binary masses and spins, while the middle row 
% % shows combinations of $\lambdans$ with non-tidal parameters. The bottom row
% % is similar to the middle, with the only difference being that the SNR has been
% % turned up to $\rho=50$.
% % }
% % \label{fig:SingleSystemLambda2DPDFs}
% % \end{figure*}
% #################
\begin{figure*}
\centering    
\textbf{Statistical uncertainty in $\lambdans$ measurement}\par\medskip
\includegraphics[trim={2cm 0 0 0},width=2.2\columnwidth]{plots/TTLambdaRawCIWidths90_0_Lambda_SNR}
% \includegraphics[trim={3cm 0 0 0},width=2.\columnwidth]{plots/TTLambdaCIWidths90_0_Lambda_SNR.pdf}
\caption{Here we show the statistical uncertainty in the measurement of
$\lambdans$, as a percentage of the injected/true value. In each panel, the
same is shown as a function of the BH mass and spin, keeping $\lambdans$ and
injection's SNR $\rho$ fixed (noted in the panel). Rows contain panels
with the same value of $\lambdans$, with $\rho$ increasing from left to right.
Columns contain panels with the same value of $\rho$, with $\lambdans$ 
increasing from top to bottom.
% 
Contours at $(\Delta\lambdans)^{90\%}=\{50\%, 75\%, 100\%, 150\%, 200\%\}\times\lambdans^\mathrm{Injected}$ demarcate regions where we can constrain the
$\lambdans$ parameter well (within a factor of two of the injected value).
% 
We note that, as expected, the measurement accuracy for $\lambdans$ improves
with (i) increasing SNR, (ii) increasing $\lambdans$, (iii) increasing BH spin,
and (iv) decreasing BH mass.
}
\label{fig:TT_LambdaCIWidths90_0_Lambda_SNR}
\end{figure*}
% % 
\begin{figure}
\centering    
% \includegraphics[trim={0 0 0 2cm}, width=0.95\columnwidth]{plots/TTSNRThresholdFor200LambdaMeasurement_BHspin_BHmass_Lambda1500_0_CI90_0}
\includegraphics[width=\columnwidth]{plots/TTSNRThresholdFor100LambdaMeasurement_BHspin_BHmass_Lambda2000_0_CI90_0}
% \includegraphics[width=1.025\columnwidth]{plots/TTSNRThresholdFor100LambdaMeasurement_BHspin_BHmass_Lambda2000_0_CI90_0}\\
\caption{
We show here, as a function of BH mass and spin, the {\it minimum} signal
strength (SNR) required to constrain $\lambdans$ within an interval of width
equal to $100\%$ of its true value, i.e. with $\pm 50\%$ error-bars. The NS mass
is fixed at $1.35M_\odot$, spin at zero, and $\lambdans=2000$.
% 
We can see that, even in the most conducive circumstances with large aligned 
$\chibh$ and a comparable mass BH, we can only constrain $\lambdans$ to better
than $\pm 50\%$ {\it if} the SNR is $\gtrsim 29$. In the era of design
sensitivity LIGO instruments, we expect this to happen approximately once in a
year of observation~\cite{Abadie:2010cfa}.
% 
% We note that this information is, in principle, contained in 
% Fig.~\ref{fig:TT_LambdaCIWidths90_0_Lambda_SNR}, which we gather here to better understand
% the effect of NS's deformability itself on its measurement. 
% 
% As hinted at in Fig.~\ref{fig:TT_LambdaCIWidths90_0_Lambda_SNR}, we note that the measurability
% of NS matter effects improves with all factors that enhances the signature of the NS's disruption on
% the \textit{detectable} portion of the emitted GW signal (implying, within a frequency band
% set by the detectorTT_SNRThresholds_BHspin_BHmass_CI90_0s).
}
\label{fig:TT_SNRThresholds_BHspin_BHmass_CI90_0}
\end{figure}
% %

%

In the previous section, we showed that the effects of the tidal deformation of
NSs by their companion BHs become discernible in
the GW spectrum under certain favorable conditions, including (a) BH mass is
sufficiently small, (b) BH spin is positive aligned, i.e. $\chibh\gtrsim +0.4$,
(c) the NS is not very compact, with $\lambdans\gtrsim 1000$, and (d) the
source location and orientation are such that its GW SNR $\gtrsim 30$.
% 
Both condition (a) and (b) enhance the tidal distortion of the star and increase
the number of orbits the system goes through at small separation, where the
differences between NSBH and BBH signals are maximal.
% 
Conditions (a)-(c) also reduce the onset frequency of the disruption of the NS,
allowing for it to happen earlier in the orbit. 
% 
We expect that these conditions are also the ones which should maximize the
likelihood of {\it measuring} tidal effects in NSBH signals. Here,
%
%In addition, a SNR of $30$ corresponds to a $1.35+5M_\odot$ binary with 
%$\chibh = 0.4$ at a luminosity distance of $XXX$~Mpc, if optimally aligned
%with LIGO ($XXX$~Mpc if the binary subtends a $45^o$ angle with LIGO's line of
%sight.
%
%These conditions are strong, but not unrealistic. For e.g., a detection rate
%of $20$ NSBHs a year implies $1-2$ of them will have SNRs$\gtrsim 30$, since
%the number of expected events scales as $1/\rho^2$ if we assume that sources
%are distributed uniformly in spacial volume.
%
we turn the question around to ask: under similarly favorable circumstances,
can we gain insights about the internal structure of neutron stars from GW
observations?



In this section, we calculate the accuracy with which we measure $\lambdans$ from
{\it single} GW observations. We sample the same set of disruptive NSBH mergers
as in the previous section, i.e. those with $q=\{2,3,4,5\}$,
$\chibh=\{-0.5,0,+0.5,+0.75\}$, and $\lambdans=\{500, 800, 1000, 1500, 2000\}$;
fixing the NS mass $\mns=1.35 M_\odot$ and $\chins=0$. For each unique
combination of these parameters, we inject LEA+ signals into zero noise and
perform a fully Bayesian parameter estimation analysis of each with LEA+
templates. Our  priors on component masses and spins remain as in the
previous section, with mass-ratio additionally restricted to $2\leq q\leq 6$,
and $\lambdans$ sampled uniformly from $[0,4000]$.
% 
%
As an illustration of individual injections, we show the recovered probability
distribution for $\lambdans$ for three specific configurations in 
Fig.~\ref{fig:SingleSystemLambdaPDFvsSNR}. We fix
$q = \mbh /\mns = 5.4M_\odot/1.35M_\odot = 4$, with $\chibh=+0.5$, and
vary $\lambdans$ over $\{1000, 1500, 2000\}$ between the three panels.
The SNR is fixed at $\rho=50$. The darker shaded regions mark the $90\%$ credible
interval on $\lambdans$. We note that $\lambdans$ is estimated to within
$\pm 2000$ of its true value at this SNR. Another interesting thing to note 
is that while $(\Delta\lambdans)^{90\%}$ slowly grows with $\lambdans$, the
fractional uncertainty
\begin{equation}
\delta\lambdans^{90\%}:= (\Delta\lambdans)^{90\%}/\lambdans
\end{equation}
decreases instead.
%
% \textcolor{blue}{%
Further illustrations, showing the correlation between tidal and non-tidal
parameters, are presented in Appendix~\ref{as1:illustrations}.
We will continue here to focus on the measurement of $\lambdans$ itself.
% }
% 
% \textcolor{gray}{%
% Next, in Fig.~\ref{fig:SingleSystemLambda2DPDFs} we show the correlation of
% mass, spin, and tidal parameter measurements. We keep the binary parameters
% as in Fig.~\ref{fig:SingleSystemLambdaPDFvsSNR}, with $\lambdans=2000$, and
% set $\rho=30$ (left panel) or $\rho=50$ (right panel).
% %
% We find that the measurement of $\lambdans$ is weakly degenerate with
% other parameters, and at realistic SNRs it would improve by a few tens of 
% percent if we knew non-tidal parameters to better accuracy. The predominant
% factor that would enhance the measurement accuracy for $\lambdans$ is the
% signal strength, and only when $\rho\gtrsim 50$ do we expect $\lambdans$ 
% measurement to be limited by its degeneracy with non-tidal parameters at 
% a factor of few level, as was reported by Ref.~\cite{Lackey:2013axa}.
% }



In Fig.~\ref{fig:TT_LambdaCIWidths90_0_Lambda_SNR} we show the main results of
this section. In each panel, as a function of black hole mass and spin, we show
the measured $90\%$ credible interval widths $(\Delta\lambdans)^{90\%}$. These
correspond to the full width of the dark shaded regions in the illustrative
Fig.~\ref{fig:SingleSystemLambdaPDFvsSNR}. The effect of increasing signal
strength can be seen as we go from left to right in each row. The effect of the
NS tidal deformability parameter $\lambdans$ on its own measurability can be
seen by comparing panels within each column, with the NS becoming more
deformable from top to bottom. 
%
A uniform pattern emerges in the left-most column, which corresponds to $\rho=20$.
We find that at this signal strength, our measurement of $\lambdans$ is
dominated by the width of our prior on it. The $90\%$ credible intervals span
the entire allowed range for $\lambdans$, making a reasonable estimation of
$\lambdans$ at $\rho\simeq20$ difficult.
% 
Increasing the signal strength to $\rho=30$ gives marginally better results,
bringing down the statistical uncertainties to within $\pm 75-100\%$ of the
true $\lambdans$ value~\footnote{The symmetric error-bars of $\pm\mathrm{X}\%$
correspond to $\dlambda = 2\mathrm{X}\%$.}.
%
It is not until we reach an SNR as high as $\rho\gtrsim 50$, can we put
meaningful (i.e. $\mathcal{O}(10\%)$) constraints on $\lambdans$. For e.g.,
with a {\it single} observation of a $q=4$ binary with $\chibh\geq 0.6$ and
$\rho = 50$~\footnote{For an optimally oriented source with
$q=4, \mns=1.35M_\odot, \chibh=0.6$, an SNR of $\rho = 50$ corresponds to
a luminosity distance of $\approx 113$Mpc.}, we would be able to estimate 
$\lambdans$ to within $\pm 40\%$ of its true value (which is equivalent to
measuring the ratio of NS radius to mass with an uncertainty of about
$\pm 10\%$).
% 
The results presented so far are in agreement with Sec.~\ref{s1:PEwithnoNS}.




Amongst other source parameters, BH mass and spin play a dominant role. A smaller
BH with a larger spin always allows for a more precise measurement on $\lambdans$.
We can see this in the bottom right corner of each panel in
Fig.~\ref{fig:TT_LambdaCIWidths90_0_Lambda_SNR}, which corresponds to low-mass BHs
with large spins, and is simultaneously the region of smallest measurement errors on $\lambdans$.
The actual deformability of the NS also plays an important role on its own
measurability. For e.g., when $\lambdans\leq 1000$, it is fairly difficult
to meaningfully constrain $\lambdans$ without requiring the source to be
close ($\approx 100$Mpc) with a GW SNR $\rho\gtrsim 50$. Quantifying this further,
in Fig.~\ref{fig:TT_SNRThresholds_BHspin_BHmass_CI90_0} we show the minimum
signal strength required to attain a certain level of credibility in our
$\lambdans$ measurement, as a function of BH properties. The NS is allowed
the most favorable (hardest) EoS considered, with $\lambdans^\mathrm{true}=2000$.
% 
We first note that, even with the most favorable BH and NS properties, achieving
a $\pm 50\%$ measurement certainty on $\lambdans$ will require a GW SNR
$\rho\gtrsim 30$. If we additionally restrict BH masses to lie outside of the so-called
astrophysical mass-gap~\cite{Bailyn:1997xt,Kalogera:1996ci,Kreidberg:2012,
Littenberg:2015tpa}, we will simultaneously need to restrict BH spins
to $\chibh\gtrsim +0.5$ to obtain the same measurement credibility at the same
source location.



In summary, with a single moderately loud ($\rho\lesssim 30$) GW signal from
a disruptive BHNS coalescence, we can constrain
the NS compactness parameter $\lambdans$ within $\pm 100\%$ of its true value.
To measure better with one observation, we will need a more fine-tuned source, with
$\rho\geq 30$ and high BH spins, or $\rho\geq 50$.
% 
Finally, we note that these results are {\it conservative}, and 
BHs with spins $\chibh > 0.75$ will prove to be even more favorable laboratories
for $\lambdans$ measurement. However, we are presently unable to explore this case
in quantitative detail due to waveform model restrictions~\cite{Lackey:2013axa},
which will restrict our analyses of actual GW signals as well.



%%%%%%%%%%%%%%%%%%%%%%%%%%%%%%%%%%%%%%%%%%%%%%%%%%%%%%%%%%%%%%%%%%%%%%%%%%%%%%%
\section{Combining observations: looking forward with Advanced LIGO}\label{s1:multiple_observations}
%%%%%%%%%%%%%%%%%%%%%%%%%%%%%%%%%%%%%%%%%%%%%%%%%%%%%%%%%%%%%%%%%%%%%%%%%%%%%%%
% 
\begin{figure}
\centering    
\textbf{Improvement in $\lambdans$ measurement accuracy\\ with multiple observations}\par\medskip
\includegraphics[width=0.8\columnwidth]{plots/pdfLambda_vs_N_L800.pdf}\\
\includegraphics[width=0.8\columnwidth]{plots/FillBetweenErrorBarsLambda_vs_N_L800.pdf}
\caption{%
These panels show the accumulation of information from a population whose
$\lambdans=800$.
{\it Left:} Posterior probability distributions for $\lambdans$ (colored 
curves), and associated $90\%$ credible intervals (grey vertical lines),
shown for different number of accumulated observations $N$. Distributions are
normalized to unit area.
{\it Right:} Measured median value of $\lambdans$ (as solid circles) and the 
associated $90\%$ credible intervals (as the vertical extent of filled
region), shown as a function of number of observations $N$. Solid horizontal
line indicates the true value of $\lambdans$. Dashed and dotted horizontal
lines (a pair for each line-style) demarcate $\pm 25\%$ and $\pm 50\%$ error
bounds.
}
\label{fig:TT_Lambda_vs_N_L800_CI90_0}
\end{figure}
%
% % 
% \begin{figure*}
% \centering    
% \includegraphics[width=.9\columnwidth]{plots/FillBetweenErrorBarsLambda_vs_N_L2000.pdf}
% \includegraphics[width=.9\columnwidth]{plots/FillBetweenErrorBarsLambda_vs_N_L1500.pdf}\\
% \includegraphics[width=.67\columnwidth]{plots/FillBetweenErrorBarsLambda_vs_N_L1000.pdf}
% \includegraphics[width=.67\columnwidth]{plots/FillBetweenErrorBarsLambda_vs_N_L800.pdf}
% \includegraphics[width=.67\columnwidth]{plots/FillBetweenErrorBarsLambda_vs_N_L500.pdf}
% \caption{Filled-region plots showing the median and $90\%$ credible intervals
% for $\lambdans$ measurement, as a function of the number of observed events $N$.
% Each panel corresponds a unique population with injected $\lambdans$ fixed to
% the value shown in its title. In each panel, the recovered median for $\lambdans$
% is shown by filled circles, and the $90\%$ credible interval is shown by the 
% height of the filled region. In addition, the dashed and 
% dotted lines (2 of each line-style) show the $\pm 25\%$ and $\pm 50\%$ error
% bounds, with respect to the true value.
% }
% \label{fig:TT_Lambda_vs_N_L500_2000_CI90_0}
% \end{figure*}
%
\begin{figure}
\centering    
\includegraphics[width=1.05\columnwidth,trim=1cm 0 0 0]{plots/FillBetweenRelErrorBarsLambda_vs_NShifted_AllLambda.pdf}
\caption{%
In this figure, the filled regions show how our measurement of $\lambdans$
improves as the number of observed events ($N$, shown on $x$-axis) increases.
Each color corresponds to an independent population with its true value of
$\lambdans$ given in the legend. For each population, we show the median 
$\lambdans$ value (as filled circles), as well as the associated
$90\%$ credible intervals for the measurement (as the vertical extent of the
filled region about the median), as functions of $N$.
}
\label{fig:TT_Lambda_vs_N_CI90_0}
\end{figure}
% 
% \begin{figure*}
% \centering
% \includegraphics[trim=20 0 0 0, width=1.02\columnwidth]{plots/LambdaMedian_vs_N_AllPopulation}
% \includegraphics[trim=20 0 0 0, width=1.02\columnwidth]{plots/LambdaMedian_vs_N_AstroPopulation}
% \caption{%
% {\it Left:} This figure shows the median value of the recovered
% probability distribution for $\lambdans$, as a function of the number of events
% in the population $N$. There are four family of curves, one corresponding each
% to $\lambdans=\{500,1000,1500,2000\}$, with $100$ independent populations
% within each family. One curve in each family is highlighted in color, 
% representing that it is the same population as was illustrated in
% Fig.~\ref{fig:TT_Lambda_vs_N_L800_CI90_0}-\ref{fig:TT_Lambda_vs_N_CI90_0}.
% In the same color we show $\pm 10\%$ error-bounds on $\lambdans$ with
% horizontal dash-dotted lines. We observe that within $10-25$ observations, 
% the median of the measured cumulative probability distribution for $\lambdans$
% will lie within $10\%$ of the true value.
% {\it Right:} This figure is identical to the left panel, with the only
% difference that the BH masses in each population are restricted to lie
% {\it outside} the astrophysical mass-gap (i.e. paradigm B). The difference that
% we observe under this paradigm is that we need more ($30+$) events to achieve 
% the same ($10\%$) measurement accuracy for populations with $\lambdans<1000$.
% For more deformable neutron stars, $10-25$ events would suffice.
% }
% \label{fig:TT_LambdaMedian_vs_N_AllInOne}
% \end{figure*}
% %
% % 
% \begin{figure*}
% \centering    
% \includegraphics[width=0.8\columnwidth]{plots/LambdaCIWidths_vs_N_AllPopulations_Log_L2000.pdf}
% \includegraphics[width=0.8\columnwidth]{plots/LambdaCIWidths_vs_N_AstroPopulations_Log_L2000.pdf}\\
% \includegraphics[width=0.8\columnwidth]{plots/LambdaCIWidths_vs_N_AllPopulations_Log_L1500.pdf}
% \includegraphics[width=0.8\columnwidth]{plots/LambdaCIWidths_vs_N_AstroPopulations_Log_L1500.pdf}\\
% \includegraphics[width=0.8\columnwidth]{plots/LambdaCIWidths_vs_N_AllPopulations_Log_L1000.pdf}
% \includegraphics[width=0.8\columnwidth]{plots/LambdaCIWidths_vs_N_AstroPopulations_Log_L1000.pdf}\\
% \includegraphics[width=0.8\columnwidth]{plots/LambdaCIWidths_vs_N_AllPopulations_Log_L800.pdf}
% \includegraphics[width=0.8\columnwidth]{plots/LambdaCIWidths_vs_N_AstroPopulations_Log_L800.pdf}\\
% \includegraphics[width=0.8\columnwidth]{plots/LambdaCIWidths_vs_N_AllPopulations_Log_L500.pdf}
% \includegraphics[width=0.8\columnwidth]{plots/LambdaCIWidths_vs_N_AstroPopulations_Log_L500.pdf}
% \caption{%
% These figures show the width of the $90\%$ credible interval for $\lambdans$
% (normalized by its true value), as a function of the number of observed events
% $N$. The left column panels show populations sampled under paradigm A, which
% allows BH masses to fall within the astrophysical mass-gap, while those on the
% right show populations drawn under paradigm B which respects the mass-gap.
% Each panel corresponds to a unique value of populations' $\lambdans$,
% decreasing from $2000\rightarrow 500$ as we go from top to bottom. In each panel,
% $100$ curves are shown, each corresponding to an independent population draw. The
% one colored curve in each panel highlights the population used in
% Fig.~\ref{fig:TT_Lambda_vs_N_L800_CI90_0}-\ref{fig:TT_Lambda_vs_N_CI90_0}.
% The insets in each panel show the same information, with the ordinate {\it not}
% normalized by the true value of $\lambdans$.
% % 
% We find that with approximately $25$ or so events, we begin to put
% statistically meaningful constraints on $\lambdans$, restricting it to within
% $\pm 50\%$ of the true value. We can expect to achieve this with a few years
% of design aLIGO operation~\cite{Abadie:2010cfa}. Further tightening of the
% credible intervals will require $40+$ events.
% }
% \label{fig:TT_LambdaError_vs_N_L500_2000_CI90_0_AllInOne}
% \end{figure*}
% % 
\begin{figure*}
\centering
\textbf{No Mass-Gap \hspace{6cm} Mass-Gap}\par\medskip
\includegraphics[trim=1cm 0 0 0, width=1.025\columnwidth]{plots/LambdaMedian_vs_N_AllPopulation}
\includegraphics[trim=0 0 1cm 0, width=1.025\columnwidth]{plots/LambdaMedian_vs_N_AstroPopulation}\\
\includegraphics[trim=1cm 0 0 0, width=1.025\columnwidth]{plots/LambdaMedian90pc_vs_N_AllPopulation}
\includegraphics[trim=0 0 1cm 0, width=1.025\columnwidth]{plots/LambdaMedian90pc_vs_N_AstroPopulation}\\
\caption{%
{\it Top left}: This figure shows the median value of the recovered
probability distribution for $\lambdans$, as a function of the number of events
in the population $N$. There are four ensembles of curves,
corresponding to $\lambdans=\{500,1000,1500,2000\}$, with a hundred
independent population draws within each ensemble. One curve in each ensemble
is highlighted in color, representing only that it is the same population as 
was discussed in
Fig.~\ref{fig:TT_Lambda_vs_N_L800_CI90_0}-\ref{fig:TT_Lambda_vs_N_CI90_0}.
In the same color we show $\pm 10\%$ error-bounds on $\lambdans$ with
horizontal dash-dotted lines.
{\it Bottom left}: Here we show the interval of $\lambdans$ values within
which the median $\lambdans$ lies for $90\%$ of the populations in
each ensemble shown in the top left panel.
% 
We observe that within $10-25$ observations, the median of the measured 
cumulative probability distribution for $\lambdans$ converges to within $10\%$
of its true value.
% 
{\it Right:} These panels are identical to their counterparts on the left,
with the only difference that the BH masses in each population are restricted
to lie {\it outside} the astrophysical mass-gap (i.e. paradigm B). The
difference that
we observe under this paradigm is that we need more ($30+$) events to achieve 
the same ($10\%$) measurement accuracy for populations with $\lambdans<1000$.
For more deformable neutron stars, $10-25$ events would suffice.
}
\label{fig:TT_LambdaMedian_vs_N_AllInOne}
\end{figure*} 
% 
% 
\begin{figure*}
\centering    
\textbf{No Mass-Gap \hspace{6cm} Mass-Gap}\par\medskip
\includegraphics[width=1.025\columnwidth,trim=1cm 0 0 0]{plots/LambdaCIWidths90pc_vs_N_AllPopulation}
\includegraphics[width=1.025\columnwidth,trim=0 0 1cm 0]{plots/LambdaCIWidths90pc_vs_N_AstroPopulation}
\caption{%
{\it Left}: This panel shows the width of $\lambdans$ interval within
which the $90\%$ credible intervals for $\lambdans$ lie, for $90\%$ of 
the populations in each ensemble, as a function of the number of observed events
$N$. Details of how this is calculated are given in the text.
% 
The populations are sampled under paradigm A, which allows BH masses to
fall within the astrophysical mass-gap.
Each panel corresponds to a unique value of populations' $\lambdans$,
decreasing from $2000\rightarrow 500$ as we go from top to bottom.
% 
{\it Right}: This panel shows populations drawn under paradigm B, which
respects the mass-gap.
% 
We find that with approximately $25$ or so events, we begin to put
statistically meaningful constraints on $\lambdans$, restricting it to within
$\pm 50\%$ of the true value. We can expect to achieve this with a few years
of design aLIGO operation~\cite{Abadie:2010cfa}. Further tightening of 
$\lambdans$ credible intervals will require $40+$ events.
}
\label{fig:TT_LambdaError_vs_N_L500_2000_CI90_0_AllInOne}
\end{figure*}




In the previous section, we showed that single observations of NSBH
coalescences at moderate SNRs have little information about the internal
structure of neutron stars that will be accessible to Advanced LIGO at its
design sensitivity. We expect all neutron stars to share the same equation of
state, and hence the same $\lambdans(\mns)$. In addition, we know that the mass
distribution of (most) NSs that have not been spun up to millisecond periods
(which are the ones we focus on in this paper, by setting $\chins\approx 0$) is
narrowly peaked around $\sim 1.35M_\odot$~\cite{Kiziltan2013}. Therefore,
information from multiple NSBH observations can be combined to improve our
estimation of $\lambdans$. We explore the same in this section within a fully
Bayesian framework. We refer the reader to Ref.~\cite{Mandel:2009pc,Lackey2014,
Wade:2014vqa} for similar analyses of BNS inspirals.




% \textbf{Multiple identical sources at low SNR: }\label{s2:identical_multiple}
% 
An intuitive understanding of the problem is gained by considering first
multiple {\it identical} sources with realistic but different SNRs. Let us consider the case
of a population of optimally oriented binaries~\footnote{An optimally oriented
binary is one which is located directly overhead the detector, with the 
orbital angular momentum parallel to the line joining the detector to the
source. Such a configuration maximizes the observed GW signal strength in 
the detector.}, distributed uniformly in spatial volume out
to a maximum {\it effective} distance~\footnote{{\it effective} distance $D$ 
is a combination of distance to the source, its orientation, and its sky
location angles; and has a one-to-one correspondence with SNR for non-precessing
sources. This is so because for such sources, their location and orientation
remain constant over the timescales within which they sweep through
aLIGO's sensitive frequency band.}.
$D^\mathrm{max}$. $D^\mathrm{max}$ is set by the minimum SNR 
threshold $\rho_\mathrm{min}$ at which a source is considered
detectable~\footnote{which
we take as $\rho_\mathrm{min}=10$ throughout.}. Next, we divide this volume into $I$
concentric shells, with radii $D_i$. If we have a measurement error
$\sigma_0$ for $\lambdans$, associated with a source located at $D=D_0$,
the same error for the same source located within the $i-$th shell would
be $\sigma_i=\sigma_0 \dfrac{D_i}{D_0}$. Ref.~\cite{Markakis:2010mp}
calculated that the combined error $\sigma$ from $N$ independent
measurements of $\lambdans$ in such a setting to be
% 
\begin{align}\label{eq:1oversigma}
\frac{1}{\sigma^2} =& \sum_{i=1}^I \frac{N_i}{\sigma_i^2} = \left(\frac{D_0}{\sigma_0}\right)^2 \sum_{i=1}^I\frac{N_i}{D_i^2}\\ \nonumber =& \left(\frac{D_0}{\sigma_0}\right)^2 \int_0^{D^\mathrm{max}} \dfrac{4\pi D^2 n}{D^2}\D D = \left(\frac{D_0}{\sigma_0}\right)^2 \dfrac{3N}{(D^\mathrm{max})^2},
\end{align}
% 
where $N_i$ is the number of sources within the $i-$th shell (s.t.
$N:=\sum N_i$), and $n$ is the number density of sources in volume.
% 
% \begin{equation}\label{eq:rmsSigmaIdenticalSources}
%  \sigma_{avg} := \frac{1}{\sqrt{\langle1/\sigma^{2}\rangle}} = \frac{\sigma_0}{D_0} \deff^\mathrm{max} \frac{1}{\sqrt{3\langle N\rangle}},
% \end{equation}
% 
% 
% 
% 
% \prayush{%
% The approximation of using (only) the dominant $l=|m|=2$ multipoles of
% gravitational-wave strain to construct the signal, ignoring other $l\neq 2$
% multipoles, has the advantage of making various angle parameters 
% associated with the source degenerate with its distance. As a result, source
% distance, source sky location angles, and the inclination angle of
% the binary's orbital angular momentum with respect to the detector,
% all combine into a single ``effective distance'' $\deff$ that scales
% out as the constant amplitude factor $1/\deff$ (constant because none
% of the above parameters vary temporally for aligned-spin binaries).
% In order to consider multiple events, lets imagine a population uniformly
% distributed in effective volume, i.e. within a sphere of radius 
% $\deff^\mathrm{max}$.
% }
% % 
% \red{These sentences are distracting. I suggest to simply say 'sources
% uniform in volume'. Probably not necessary to discuss effective distance
% at all. If so, then do so in a footnote.}
% \red{Explain (or state, if explanation is trivial) that uniform in volume,
% uniform in orientation, is equivalent to uniform in effective volume.}
% % 
% \prayush{%
% The radius of this sphere is set by the lowest SNR that 
% is distinguishable from noise by LIGO searches. Now, divide the sphere into
% $I$ shells of equal thickness. The radius of the $i$-th sphere would then
% be $D_i = \deff^\mathrm{max} (i - 1/2)/I$.}
% \red{..into $I$ shells of same radial thickness. The radius at the middle of
% the $i-$th shell $(i=1,2,...,I)$ is $D_i:=\frac{i-\frac{1}{2}}{I}\,D_\mathrm{max}$,
% and the number of sources in this shell is $N_i=4\pi\beta D_i^2 \frac{D_\mathrm{max}}{I}$,
% where $\beta$ is the volume density of sources.
% }
% The measurement uncertainty in
% $\lambdans$ scales inversely with SNR, and hence directly with the 
% effective distance to the source. Therefore, if we have a measurement error 
% $\sigma_0$ for a source located at $\deff = D_0$, the same error for the same
% source located within the $i$-th shell would be 
% $\sigma_i = \sigma_0 \dfrac{D_i}{D_0}$. The combined error from independent
% measurements of $\lambdans$ for identical sources at different distances is
% given by
% % 
% \begin{equation}\label{eq:1oversigma}
% \frac{1}{\sigma^2} = \sum_{i=1}^I \frac{N_i}{\sigma_i^2} = \left(\frac{D_0}{\sigma_0}\right)^2 \sum_{i=1}^I\frac{N_i}{D_i^2},
% \end{equation}
% where $N_i$ is the number of sources detected in the $i$-th shell.
% % , and is a random
% % variable, with its probability proportional to the volume of the shell, i.e.
% % $$
% % p(N_i) \propto \frac{V_i}{V_\mathrm{total}} \propto \dfrac{\left(\deff^\mathrm{max} \frac{i}{I}\right)^3 - \left(\deff^\mathrm{max} \frac{i-1}{I}\right)^3}{(\deff^\mathrm{max})^3} \propto \frac{1}{I^3} [i^3 - (i-1)^3].
% % $$
% % Therefore, the expected number of detections in the $i$-th shell would be
% % $$
% % \langle N_i\rangle = \frac{N}{I^3} [i^3 - (i-1)^3],
% % $$
% % where $N=\sum_{i=1}^I N_i$ is the total number of NSBH detections. $N$ is expected
% % to be Poisson distributed around the mean detection rate
% % $\mathcal{R}\equiv\langle N\rangle$, where $\mathcal{R}$ can vary from $0.6-1000$ per
% % $\mathrm{Gpc}^3$ per year~\cite{Abadie:2010cfa}. If there are $n$ resulting detections
% % a year per unit effective volume, then
% % \begin{equation}
% % \langle N\rangle = \int_0^{\deff^\mathrm{max}} 4\pi n D^2 \D D = \frac{4\pi}{3} n (\deff^\mathrm{max})^3,
% % \end{equation}
% % and
% % \begin{eqnarray}
% %  \langle \frac{1}{\sigma^2}\rangle &=& \left(\frac{D_0}{\sigma_0}\right)^2 \int_0^{\deff^\mathrm{max}} \frac{4\pi n D^2 }{D^2}\D D\\
% %  &=& \left(\frac{D_0}{\sigma_0}\right)^2 4\pi n \deff^\mathrm{max},
% % \end{eqnarray}
% % where in the previous equation we have converted the summation in Eq.~\ref{eq:1oversigma} 
% % to an integral. 
The root-mean-square (RMS) averaged measurement error from $N$ sources is 
then~\cite{Markakis:2010mp}
\begin{equation}\label{eq:rmsSigmaIdenticalSources}
 \sigma_{avg} := \frac{1}{\sqrt{1/\sigma^{2}}} = \frac{\sigma_0}{D_0} D^\mathrm{max} \frac{1}{\sqrt{3 N}},
\end{equation}
given a fiducial pair $(\sigma_0, D_0)$. It is straightforward to deduce from
Eq.~\ref{eq:rmsSigmaIdenticalSources} that measurement uncertainty scales as 
$1/\sqrt{N}$, and the uncertainty afforded by a single observation with a high
SNR $\rho_c$ can be attained with $N = \rho_c^2/300$ realistic observations
that have $\rho\geq\rho_\mathrm{min}$. E.g., to get to the
level of certainty afforded by a single observation with $\rho=70$, we would
need $49/3\approx 16-17$ realistic (low SNR) detections.

While we discussed Eq.~\ref{eq:rmsSigmaIdenticalSources} for a population
of optimally oriented sources, it is valid for a more general population
distributed uniformly in effective volume~\cite{Markakis:2010mp}
($\propto D^3$).
% ~\footnote{{\it effective} distance $D$ 
% is a combination of distance to the source, its orientation, and its sky
% location angles, and has a one-to-one correspondence with SNR for non-precessing
% systems, for which these angles are constant in time. Effective volume is 
% $\frac{4\pi}{3}D^3$.}
However, it 
still only applies to sources with identical masses and spins, and we 
overcome this limitation by performing a fully Bayesian analysis next.


\textbf{Astrophysical source population: }\label{s2:astro_multiple}
% 
Imagine that we have $N$ stretches of data, $d_1, d_2, \cdots, d_N$, each 
containing a single signal emitted by an NSBH binary. Each of these signals can
be characterized by the non-tidal source parameters
$\vec{\theta} := \{\mbh, \mns, \chibh, \chins, \vec{\alpha}\}$,
and $\{\lambdans\}$, where $\vec{\alpha}$ contains extrinsic parameters,
such as source distance, inclination, and sky location angles.
% 
As before, let $H$ denote all of our collective prior knowledge; for instance,
$H$ includes our assumption that all NSs in a single population have the same
deformability parameter $\lambdans$, and that its cumulative measurement is
therefore possible.
% 
The probability distribution for $\lambdans$, given $N$ unique and
independent events, is
% 
\begin{eqnarray}
 p(\lambdans |&& \hspace{-4mm}d_1, d_2, \cdots, d_N, H)\hspace{50mm}\nonumber\\ &=& \dfrac{p(d_1,d_2,\cdots,d_N |\lambdans , H)\,p(\lambdans|H)}{\int p(\lambdans |H) p(d_1,d_2,\cdots,d_N |\lambdans , H)\D\lambdans},\label{eq:p11}\\
  &=& \dfrac{p(\lambdans|H) \prod_{i} p(d_i|\lambdans, H)}{\int p(\lambdans ) p(d_1,d_2,\cdots,d_N |\lambdans , H)\D\lambdans},\label{eq:p12} \\
  &=& \dfrac{p(\lambdans|H) \prod_{i} \left( p(\lambdans |d_i, H)\dfrac{p(d_i)}{p(\lambdans|H)} \right)}{\int\, p(\lambdans|H )\, p(d_1,d_2,\cdots,d_N |\lambdans , H)\D\lambdans}\label{eq:p13};
\end{eqnarray}
% 
where Eq.~\ref{eq:p11} and Eq.~\ref{eq:p13} are application of Bayes' theorem,
while Eq.~\ref{eq:p12} comes from the mutual independence of all events.
Assuming in addition that all events are {\it equally likely}: 
$p(d_i) = p(d_j) = p(d)$, we get
% 
\begin{eqnarray}
 p(&&\hspace{-4mm}\lambdans | d_1, d_2, \cdots, d_N, H)\hspace{50mm}\nonumber\\
%   &=& \dfrac{p(\lambdans) \left(\dfrac{p(d)}{p(\lambdans)}\right)^N\prod_{i} p(\lambdans |d_i, H) }{\int p(\lambdans ) p(d_1,d_2,\cdots,d_N |\lambdans , H)\D\lambdans}\label{eq:p21},\\
  &=& p(\lambdans)^{1-N}\times \dfrac{p(d)^N}{\int p(\lambdans) p(d_1, d_2, \cdots, d_N |\lambdans, H)\D\lambdans}\nonumber\\ &&\hspace{3mm}\times\prod_i p(\lambdans |d_i, H)\label{eq:p22},
\end{eqnarray}
% 
where the prior probability $p(\lambdans|H)$ is written $p(\lambdans)$ for
brevity. {\it A priori}, we assume that no particular value of $\lambdans$ is
preferred over another within the range $[0, 4000]$, i.e.
\begin{equation}\label{eq:lprior}
 p(\lambdans | H) = \dfrac{1}{4000}\,\mathrm{Rect}\left(\frac{\lambdans-2000}{4000}\right).
\end{equation}
With a uniform prior, the first two factors in Eq.~\ref{eq:p22} can be
absorbed into a normalization factor $\mathcal{N}$, simplifying it to
% 
\begin{equation}\label{eq:lambdaMultiple}
 p(\lambdans | d_1, d_2, \cdots, d_N; H) = \mathcal{N}\prod_{i=1}^N p(\lambdans | d_i, H).
\end{equation}
% 
% 
% % 
% Using Bayes' theorem, the measured probability distribution function for 
% $\lambdans$ from $N$ observations is given by~\footnote{%
% Eq.~\ref{eq:lambdaMultiple} includes the following implicit factor to normalize
% its right-hand side: $\dfrac{\prod_i p(d_i|H)}{\int p(\lambdans)\,p(d_1, d_2, \cdots, d_N | \lambdans; H)\,\D\lambdans}$}
% \begin{equation}\label{eq:lambdaMultiple}
%  p(\lambdans | d_1, d_2, \cdots, d_N; H) = p(\lambdans | H)^{1-N}\prod_{i=1}^N p(\lambdans | d_i, H),
% \end{equation}

% 
In the second set of terms in Eq.~\ref{eq:lambdaMultiple} (of the form 
$p(\lambdans | d_i, H)$), each is the probability distribution for $\lambdans$
inferred {\it a posteriori} from the \textit{i}-th observation by marginalizing
\begin{equation}\label{eq:margpost}
 p(\lambdans | d_i, H) = \int\, p(\vec{\theta}, \lambdans | d_i, H)\, \D \vec{\theta},
\end{equation}
where $p(\vec{\theta}, \lambdans | d_i, H)$ is the inferred joint probability 
distribution of all source parameters $\vec{\theta}\cup\{\lambdans\}$ for the 
$i$-th event, as given by Eq.~\ref{eq:postprob}. We note that 
Fig.~\ref{fig:SingleSystemLambdaPDFvsSNR} illustrates $p(\lambdans | d_i, H)$
for three individual events. By substituting
Eq.~\ref{eq:lprior}-\ref{eq:margpost} into Eq.~\ref{eq:lambdaMultiple}, we
calculate the probability distribution for $\lambdans$ as measured using $N$
independent events.




Our goal is to understand the improvement in our measurement of $\lambdans$
with the number of recorded events. To do so, we simulate a population~\footnote{%
A population here is an ordered set of events, and an event itself is the 
set of parameters describing one astrophysical NSBH binary.}
of $N$ events, and quantify what we learn from each successive observation 
using Eq.~\ref{eq:lambdaMultiple}. This allows us to quantify how
rapidly our median estimate for $\lambdans$ converges to the true value,
and how rapidly our credible intervals for the same shrink, with increasing
$N$. Finally, we generate and analyze an ensemble of populations in order to
average over the stochastic process of population generation itself.


In order to generate each population, the first step is to fix
the NS properties: (i) NS mass $\mns=1.35M_\odot$, (ii) NS spin $\chins=0$
and (iii) NS tidal deformability $\lambdans=$ fixed value chosen from
$\{500,800,1000,1500,2000\}$. Next, we generate events, by sampling BH mass
(uniformly) from $\mbh\in[3M_\odot,6.75M_\odot]$, BH spin (uniformly)
from $\chibh\in[0, 1]$, orbital inclination from $\iota\in[0,\pi]$, and 
source location uniform in spatial volume\footnote{with a minimum SNR 
$\rho_\mathrm{min}=10$}.
% 
We restrict ourselves to positive aligned BH spins, since binaries with
anti-aligned spins have very little information to add at realistic SNRs,
as demonstrated in Fig.~\ref{fig:TT_LambdaCIWidths90_0_Lambda_SNR}. This is
to be taken into account when the number of observations is related to detector
operation time. 
% 
We repeat this process till we have an ordered set of $N$ events.
% 
Since we want to analyze not just a single realization of an astrophysical
population, but an ensemble of them, we make an additional approximation to
mitigate computational cost. Complete Bayesian parameter estimation is
performed for a set of simulated signals whose parameters are the vertices
of a regular hyper-cubic grid (henceforth ``G'') in the space of
$\{q\}\times\{\chibh\}\times\{\rho\}$, with each sampled at $q=\{2,3,4,5\}$,
$\chibh=\{-0.5,0,0.5,0.75\}$, and $\rho=\{10,20,30,50,70\}$.
All events in each population draw are substituted by their respective nearest
neighbours on the grid G.
% 
Our chosen signal parameter distribution is different from some other studies
in literature, which often sample from more astrophysically motivated population
distribution functions~\cite{Mandel:2009pc}. We chose one that is sufficiently
agnostic in absence of actual known NSBHs, and pragmatic enough for generating
population ensembles.




In Fig.~\ref{fig:TT_Lambda_vs_N_L800_CI90_0} we show illustrative results
for a single population with neutron star deformability $\lambdans=800$.
In the top panel, each curve shows the 
probability distributions for $\lambdans$ as inferred from $N$ events, with $N$
ranging from $1-80$. We also mark the $90\%$ credible intervals associated
with each of the probability distribution curves. The first few observations
do not have enough information to bound $\lambdans$ much more than
our prior from Eq.~\ref{eq:lprior} does. 
% 
In the bottom panel, we present information derived from the top panel.
The line-circle curve shows the measured median value from $N$ observations.
The pair of dashed (dotted) horizontal lines mark
$\pm25\%$ ($\pm50\%$) error bars. At each $N$, the range spanned by the 
filled region is the $90\%$ credible interval deduced from the same 
events. This figure somewhat quantifies the qualitative deductions we made
from the left panel. We find that the median does track the true value quickly,
reaching within its $10\%$ with $10-15$ observations. This is as one expects of 
injections in zero noise where random fluctuations are unable to shift the
median away from the true value, so long as the measurement is not restricted
by the prior. With the same information, our credible intervals also shrink to $\pm 25\%$.
% 
In Fig.~\ref{fig:TT_Lambda_vs_N_CI90_0} we show further results from four
independent populations for $\lambdans=\{500,1000,1500,2000\}$. As in the 
right panel of Fig.~\ref{fig:TT_Lambda_vs_N_L800_CI90_0}, the line-circle curves
track the median $\lambdans$, while the filled regions show
the associated $90\%$ credible intervals. From the figure, we observe
the following: (i) the shrinkage of credible interval widths with increasing
$N$ happens in a similar manner for each $\lambdans$, 
% (ii) within $\sim 10$ observations, median $\lambdans$ for all populations lie
% within $10\%$ of the true values, 
and (ii) it takes
approximately $20$ events to distinguish definitively (with $90\%$ credibility)
between deformable NSs with $\lambdans=2000$ and compact NSs with 
$\lambdans=500$, or equivalently to distinguish between hard, moderate and soft
nuclear equations of state. This is comparable to what has been found for
binary neutron stars~\cite{DelPozzo:13,Chatziioannou:2015uea,Agathos:2015a}.



So far we have discussed individual realizations of NSBH populations. The 
underlying stochasticity of the population generation process makes it
difficult to draw generalized inferences (from a single realization of an
NSBH population) about the measurability of $\lambdans$. In order to mitigate
this, we discuss ensembles of population draws next. In
Fig.~\ref{fig:TT_LambdaMedian_vs_N_AllInOne} we show the median
$\lambdans$ as a function of the number of observed
events, for four population ensembles, with a hundred population draws
in each ensemble. Lets focus on the {\it top left} panel first. In it, we
show the median $\lambdans$ for all populations in four ensembles,
with true $\lambdans=\{2000,1500,1000,500\}$ from top to bottom.
Populations highlighted in color are simply those that we discussed in
Fig.~\ref{fig:TT_Lambda_vs_N_CI90_0}. Dash-dotted horizontal lines
demarcate $\pm10\%$ error intervals around the true $\lambdans$ values.
The panel just below it shows the range of $\lambdans$ that encloses the
median $\lambdans$ for $90\%$ of the populations in {\it each}
ensemble. In other words, this panel shows the range of $\lambdans$ within which
the median $\lambdans$ value for $90\%$ of NSBH populations is
expected to lie. From these panels, we observe that our median $\lambdans$
values will be within $10\%$ of the {\it true} value after $\sim 25$
detections of less deformable neutron stars ($\lambdans\leq 1000$), or
after as few as $15$ detections of more deformable neutron stars
($\lambdans\geq 1500$). This is not surprising because we inject simulated
signals in {\it zero} noise, which ensures that the median not be shifted away
from the true value. That it takes $15+$ events for the median to approach
the true value is a manifestation of the fact that the measurement is limited
by the prior on $\lambdans$ when we have fewer than $15$ events.
% 
The results discussed in Fig.~\ref{fig:TT_Lambda_vs_N_L800_CI90_0},
\ref{fig:TT_Lambda_vs_N_CI90_0} and the left two panels
of Fig.~\ref{fig:TT_LambdaMedian_vs_N_AllInOne} apply to the parameter
distribution spanned by the grid G. This distribution allows for $\mbh$
as low as $2.7M_\odot$ (i.e. $q=2$).
Given that disruptive signatures are strongest for small $\mbh$, we now
investigate an alternate paradigm in which no black hole masses fall within
the mass gap $2-5M_\odot$ suggested by astronomical
observations~\cite{Bailyn:1997xt,Kalogera:1996ci,Kreidberg:2012,
Littenberg:2015tpa}. We will henceforth denote our standard paradigm, which
does not respect the mass-gap, as paradigm A; with paradigm B being
this alternate scenario.
% 
Both right panels of the figure are identical to their corresponding left
panels, but drawn under population paradigm B. Under this paradigm, we
expectedly find that information accumulation is much slower. It would
take $25-40$ detections with $\rho\geq10$ under this paradigm, for our median
$\lambdans$ to converge within $10\%$ of its true value.
% % 
% This is a promising deduction, as one might expect to see $\mathcal{O}(10)$
% events over a few years' timescale with design aLIGO~\cite{Abadie:2010cfa}.



Finally, we investigate the statistical uncertainties associated with
$\lambdans$ measurements. We use $90\%$ credible intervals as our measure of
the same. First, we draw an ensemble of a hundred populations each for
$\lambdans=\{500,1000,1500,2000\}$. For each population $i$ in each ensemble,
we construct
its $90\%$ credible interval $[{\lambdans^{90\%}}_{i-},{\lambdans^{90\%}}_{i+}]$.
Next, we construct the interval $[X^-,Y^-]$ that contains ${\lambdans^{90\%}}_{i-}$
for $90\%$ of the populations in each ensemble; and similarly $[X^+,Y^+]$
for ${\lambdans^{90\%}}_{i+}$. Finally, in the left panel of 
Fig.~\ref{fig:TT_LambdaError_vs_N_L500_2000_CI90_0_AllInOne}, we show the
conservative width $|Y^+ - X^-|$ that contains the $90\%$ credible
intervals for $90\%$ of all populations in each ensemble~\footnote{Drawn
under paradigm A.}.
% 
% In each panel
% we show the width of the $90\%$ credible interval measured from $N$ events
% (with $N$ on the $x$-axis), normalized by the true value of $\lambdans$ for
% the population. In gray are shown results from each of $100$ population
% realizations, with the single population highlighted in color corresponding
% to the one we focused on in 
% Fig.~\ref{fig:TT_Lambda_vs_N_L800_CI90_0}-\ref{fig:TT_Lambda_vs_N_CI90_0}.
% The inset shows the same, except that the ordinate is {\it not} normalized by
% the true value of $\lambdans$. 
% 
From top to bottom, the population $\lambdans$
decreases from $\lambdans=2000\rightarrow 500$, corresponding to decreasingly
deformable NSs with softer equations of state. We observe the following:
(i) for moderately-hard to hard equations of state with $\lambdans\geq 1000$,
we can constrain $\lambdans$ within $\pm 50\%$ using only $10-20$ events, and
within $\pm 25\%$ (marked by black circles) with $25-40$ events; (ii) for softer 
equations of state with
$\lambdans<1000$, we will achieve the same accuracy with $20-30$ and $50+$ 
events, respectively; and (iii) for the first $5$ or so observations, our
measurement spans the entire prior allowed range:
$\lambdans\in[0,4000]$, as shown by the plateauing of the $90\%$ 
credible intervals towards the left edge to $90\%$ of $4000$, i.e. $3600$.
% 
The right panel in Fig.~\ref{fig:TT_LambdaError_vs_N_L500_2000_CI90_0_AllInOne}
is identical to the left one, with the difference that populations are drawn
under paradigm B, which does {\it not} allow for BH masses to fall within
the mass-gap. We find that
for NSs with $\lambdans\leq 1000$, it would take $25-40$ events to
constrain $\lambdans$ within $\pm 50\%$ and $50+$ events to constrain it
within $\pm 25\%$. This is somewhat slower than paradigm A, as is to be
expected since here we preclude the lowest mass-ratios, which correspond to
signals with largest tidal signatures. For $\lambdans>1000$ we find that we
can constrain $\lambdans$ within $\pm 50\%$ with a similar number of events as
for paradigm A, but will need more ($30-40$, as compared to $25-40$) events
to further constrain it to within $\pm 25\%$ of the true value.
% 
Under either paradigms, we find that measuring $\lambdans$ better than
$25\%$ will require $\mathcal{O}(10^2)$ observations of disruptive NSBH
mergers.


% Finally, we quantitatively explore the dependence of our statistical
% uncertainties for $\lambdans$ on the number of events, as well as on the true
% NS deformability itself. First, we will focus on the dependence on $N$. We
% assume a power-law dependence of the form
% $\delta\lambdans\propto\ 1/N^\alpha$. For each of the $100$ populations 
% for each of $\lambdans=500-2000$, we compute the exponent $\alpha$ as a
% function of the number of observed events $N$, and show it in 
% Fig.~\ref{fig:TT_PowerLawLambdaErrorVsN}. There are $100\times5=500$ curves
% on the figure, with one highlighted for each value of population's $\lambdans$.
% These highlighted values are only special in the sense that they correspond to
% populations discussed earlier in this section (c.f.
% Fig.~\ref{fig:TT_Lambda_vs_N_L800_CI90_0}-\ref{fig:TT_Lambda_vs_N_CI90_0}).
% We immediately observe two things, (i) there is a globally similar dependence
% on $N$ for all populations, and (ii) information accumulates {\it faster} than
% $1/\sqrt{N}$. In fact, we find that if
% $\delta\lambdans\propto\frac{1}{N^\alpha}$, $\alpha$ lines in the range
% $0.7_{-0.2}^{+0.2}$.
% % 
% Next, we focus on the dependence of $\delta\lambdans$ on $\lambdans$ of the
% population itself. As suggested by Fisher-matrix studies~\cite{Lackey:2013axa},
% and as for $N$, we assume the form $\delta\lambdans\propto\lambdans^\beta$.
% From each set of $100$ populations with a given $\lambdans$ value, we draw one
% at random, and form a set of $5$ similarly drawn populations, one for each of
% $\lambdans=\{500,800,1000,1500,2000\}$. With each set, we determine $\beta$
% for different number of observed events $N$. In all, we make $100$ independent
% $5-$population sets and show the value of $\beta$ measured from each in 
% Fig.~\ref{fig:TT_PowerLawLambdaErrorVsLambda}. We find that the assumed
% relation $\delta\lambdans\propto\lambdans^\beta$ gets fairly robust for 
% larger values of $N$, with $\beta$ converging to $\beta=0.5^{+0.33}_{-0.33}$.
% The fact that $0<\beta<1$ implies that the relative error
% $\delta\lambdans/\lambdans$ {\it decreases} with increasing $\lambdans$, while
% the absolute error {\it increases}.
% % 
% From these results, we conclude that the measurement uncertainty for
% $\lambdans$ after $N$ observations is
% \begin{equation}
%  \delta\lambdans\propto \dfrac{\lambdans^{0.5^{+0.33}_{-0.33}}}{N^{0.7_{-0.2}^{+0.2}}}.
% \end{equation}
% We also find that while these results are inferred from paradigm A populations,
% paradigm B gives very similar results.



To summarize, in this section we study the improvement in our measurement of 
NS deformability parameter $\lambdans$ with an increasing number of events. We
do so by simulating plausible populations of disrupting NSBH binaries (with
$\rho\geq 10$). We find that:
(i) for more deformable neutron stars (harder equation of states), the median
value of $\lambdans$ comes within $10\%$ of the true value with as 
few as $10$ events, while achieving the same accuracy for softer equations of 
state will take $15-20$ source detections; (ii) the statistical uncertainty
associated with $\lambdans$ measurement shrinks to within $\pm50\%$ with
$10-20$ events, and to within $\pm 25\%$ with $50+$ events, when source 
$\lambdans\geq 1000$; (iii) for softer equations of state, the same could take
$25-40$ and $50+$ events, respectively for the two uncertainty thresholds;
and (iv) if BHs really do observe the astrophysical mass-gap, the information
accumulation is somewhat slower than if they do not. We conclude that within
$20-30$ observations, aLIGO would begin to place very interesting bounds on 
the NS deformability, which would allow us to rule out or rank different
equations of state for neutron star matter. Our key findings are 
summarized in Fig.~\ref{fig:TT_LambdaMedian_vs_N_AllInOne} and
\ref{fig:TT_LambdaError_vs_N_L500_2000_CI90_0_AllInOne}.





%%%%%%%%%%%%%%%%%%%%%%%%%%%%%%%%%%%%%%%%%%%%%%%%%%%%%%%%%%%%%%%%%%%%%%%%%%%%%%%
\section{Discussion}\label{s1:discussion}
%%%%%%%%%%%%%%%%%%%%%%%%%%%%%%%%%%%%%%%%%%%%%%%%%%%%%%%%%%%%%%%%%%%%%%%%%%%%%%%

The pioneering terrestrial observation of gravitational waves by Advanced LIGO
harbingers the dawn of an era of gravitational-wave astronomy where observations
would finally drive scientific discovery~\cite{Abbott:2016blz}. As confirmed by
the first observations~\cite{Abbott:2016blz,Abbott:2016nmj,Abbott:2016nhf},
stellar-mass compact binary mergers emit GWs right in the sensitive frequency
band of the LIGO observatories, and are their primary targets.
Neutron star black hole binaries form a physically distinct sub-class of
compact binaries. We expect to detect the first of them in the upcoming
observing runs~\cite{Abbott:2016ymx}, and subsequently at a healthy rate of
$0.2-300$ mergers a year when aLIGO detectors reach design
sensitivity~\cite{Abadie:2010cf}.

NSBH binaries are interesting for various reasons. Unlike BBHs, the presence of
matter allows for richer phenomena to occur alongside the strong-field
gravitational dynamics. The quadrupolar moment of the NS changes during the
course of inspiral, which increases the inspiral rate of the binary and alters the
form of the emitted gravitational waves. Close to merger, under restricted but
plausible conditions, the neutron star is disrupted by the tidal field of its 
companion black hole and forms an accretion disk around it. This disruption
reduces the quadrupolar moment of the system, and decreases the amplitude of
the emitted GWs from the time of disruption through to the end of ringdown.
Both of these phenomena are discernible in their gravitational-wave signatures
alone. In addition, if the neutron star matter is magnetized, the magnetic
winding above the remnant black hole poles can build up magnetic fields
sufficiently to power short gamma-ray bursts (SGRB)~\cite{Foucart:2015a,
Lovelace:2013vma,Deaton2013,Foucart2012,Shibata:2005mz,Paschalidis2014}.
Therefore a coincident observation of gravitational waves from an NSBH merger
and a SGRB can potentially confirm the hypothesis that the former is a
progenitor of the latter~\cite{eichler:89,1992ApJ...395L..83N,moch:93,
Barthelmy:2005bx,2005Natur.437..845F,2005Natur.437..851G,Shibata:2005mz,
Tanvir:2013,Paschalidis2014}.


In this paper we study the observability of tidal signatures in the
gravitational-wave spectrum of NSBH binaries. More specifically, we investigate
three questions. First, what is the effect of not including tidal effects in 
templates while characterizing NSBH signals? Second, if we do include tidal 
effects, how well can we measure the tidal deformability of the NS
(parameterized by $\lambdans$) from individual NSBH signals? And third, as we
observe more and more signals, how does our knowledge of $\lambdans$ improve?
In the following, we summarize our main findings.





First, we study the effects of not including tidal terms in our search
templates while characterizing NSBH signals. We expect that the waveform
template that best fits the signal would compensate for the reduced number of
degrees of freedom in the template model by moving away from the true
parameters of the binary. This should result in a {\it systematic} bias in 
the recovered values of non-tidal source parameters, such as its masses 
and spins. In order to quantify it, we inject tidal signals into zero noise,
and perform a Bayesian parameter estimation analysis on them using templates
{\it without} tidal terms.
% 
We use the LEA+ model (c.f. Sec.~\ref{s2:waveforms}) to produce tidal waveforms
that incorporate the effect
of NS distortion during inspiral, and of its disruption close to merger. Our
injected signals sample the region of NSBH parameter space where NS disruption
prior to binary merger is likely {\it and} can be modeled using LEA+. Their
parameters are given by combinations of $q=\mbh/\mns=\{2,3,4,5\},
\chibh=\{-0.5, 0, 0.5, 0.75\}$ and $\lambdans=\{500,800,1000,1500,2000\}$.
Other parameters, such as source location and orientation, that factor out of
$h(t)$ as amplitude scaling are co-sampled by varying $\rho=\{20,30,50,70\}$.



At low to moderate SNRs ($\rho\lesssim 30$), we find that using BBH templates
does not significantly hamper our estimation of non-tidal parameters for NSBH
signals. In the worst case, when the BH mass is within the astrophysical 
mass-gap~\cite{Bailyn:1997xt,Kalogera:1996ci,Kreidberg:2012,Littenberg:2015tpa}
and its spin is positive aligned, the systematic biases in $\eta$ and $\chibh$
measurements do become somewhat comparable to statistical errors (ratio
$\sim 0.5-0.8$) under very restrictive conditions~\footnote{requiring a
companion BH with mass $\mbh\lesssim 4.5M_\odot$ (i.e. in the astrophysical
mass-gap), and the
hardest NS EoS considered (with $\lambdans\simeq 2000$).}, but never exceed them.
At high SNRs ($\rho\gtrsim 50$), systematic biases in $\mchirp$ become larger
than the statistical uncertainties. For $\eta$ and $\chibh$ the difference
is more drastic with the systematics reaching up to $4\times$ the statistical
errors. We therefore conclude that $\rho\simeq 30-50$ is loud enough to
motivate the use of tidal templates for even the estimation of non-tidal
parameters from NSBH signals.
% 
We also conclude that low-latency parameter estimation algorithms, designed to
classify GW signals into electromagnetically active (NSBH and NSNS) and
inactive (BBH) sources, can use BBH templates to trigger GRB 
alerts~\cite{2012A&A...541A.155A,Singer:2014qca,Singer:2015ema,Pankow:2015cra,
Abbott:2016wya,Abbott:2016gcq} for NSBH signals with low to moderate SNRs
($\rho\lesssim 30$).
This is so because the primary requirement of identifying NS-X binaries (X =
\{NS, BH\}) can be achieved just as easily with BBH templates, on the basis of
the smaller component's mass\footnote{The smaller component mass is unlikely
to be significantly biased by missing tidal effects in filter templates below
$\rho\simeq 30$, as we show above.}.
% 
We also speculate that NSBH detection searches are unlikely to be
affected by the choice of ignoring tidal effects in matched-filtering
templates, if these effects are too subtle to manifest in parameter estimation
below $\rho\simeq 30$.



% 
% % On the known NS and BH properties
% The number of neutron stars observed using conventional astronomical methods has
% grown rapidly in the recent past, both in isolated and two-body
% systems~\cite{Demorest:2010bx,Lyne:2004cj,2013Sci...340..448A,atnfcatalog,
% mcgillmagnetarcatalog,stellarcollapsemass}. Their masses span the range $1.2-2M_\odot$
% with an average of $\sim 1.35M_\odot$~\cite{Lattimer:2004sa,stellarcollapsemass}.
% On the other hand, NS spins have been observed to have magnitudes below
% $|\vec{S}_\mathrm{NS}|/\mns^2 < 0.01$~\cite{Miller:2014aaa}.
% % 
% On the other hand, indirect observations of stellar-mass BHs place their
% masses between $5-35M_\odot$, with their spin angular momenta 
% $|\vec{\chi}_\mathrm{BH}|$ ranging from small to nearly extremal (Kerr) values
% (see, e.g., Refs.~\cite{McClintockEtAl:2006,Miller:2009cw,Gou:2014una} for 
% examples of nearly extremal estimates of BH spins, Refs.~\cite{McClintock:2013vwa,
% Reynolds:2013qqa} for recent reviews of astrophysical BH spin measurements,
% and Figure 5 of Ref.~\cite{Miller:2014aaa} for a comparison of NS and BH spins).


Second, we turn the question around to ask: can we measure the tidal effects if
our template models did account for them? Tidal effects in our waveform model
are parameterized using a single deformability parameter 
$\lambdans\propto (R/M)_\mathrm{NS}^5$. In order to quantify the 
measurability of $\lambdans$, we inject the same tidal signals as before, and
this time perform a Bayesian analysis on them using {\it tidal} templates. 
The results are detailed in Sec.~\ref{s1:PEwithNS}.
% 
At low SNRs ($\rho\simeq 20$), we find that the best we can do is to constrain
$\lambdans$ within $\pm 75\%$ of its true value at $90\%$ credible level. This
too only if the BH is spinning sufficiently rapidly, with $\chibh\gtrsim +0.7$,
and the NS has $\lambdans\gtrsim 1000$. At moderate SNRs ($\rho\simeq 30$), we
can constrain $\lambdans$ a little better, i.e. within $\pm 50\%$ of its true
value. This level of accuracy, however, again requires that BH spin
$\chibh\gtrsim+0.7$ and $\lambdans\gtrsim 1000$. Binaries with smaller BH spins
and/or softer NS EoSs will furnish worse than $\pm 75\%-\pm 100\%$ errors for
$\lambdans$. This trend continues as we increase the SNR from $\rho=30-50$. It
is not before we reach an SNRs as high as $\rho\simeq 70$ that we can shrink
$\lambdans$ errors substantially with a single observation (i.e. within
$\pm 25\%$ of its true value).
% 
In summary, we find that with a single but moderately loud NSBH signal,
Advanced LIGO can begin to put a factor of $1-2\times$ constraints on NS tidal
deformability parameter. These constraints can subsequently be used to assess
the likelihood of various candidate equations of state for nuclear matter, and
possibly to narrow the range they span.




% \prayush{\bf Sec 5 summary}\\
% \prayush{\bf Tidal measurements versus observation time - in how many years can
% we constrain $\lambdans$ reasonably?}\\
% % 
Third, knowing that single observations can furnish only so much information
about the NS equation of state, we move on to investigate how well we do with
multiple signals. In order to quantify how $\lambdans$ measurement improves
with the number of observed events $N$, we generate populations of NSBH signals
and combine the information extracted from each event.
% 
The population generation procedure is as follows. The neutron star mass is
held fixed at $1.35M_\odot$, its spin at $\chins=0$, and its tidal
deformability is fixed to each of $\lambdans=\{500,1000,1500,2000\}$. Black hole
mass is sampled uniformly from the range $[2,5]\times 1.35=[2.7, 6.75]M_\odot$,
and spin from $\chibh\in[0,1]$. As before, our parameter choice here is given
by the intersection set of the mass range that allows for neutron star disruption
and the range supported by LEA+~\cite{Foucart2012,Foucart:2013a,Lackey:2013axa}.
In order to keep the computational cost reasonable, we make an additional
approximation. For every population generated, we replace the parameters of each
event by their nearest neighbor on the uniform grid G, which has vertices
at: $q=\{2,3,4,5\}\times\chibh=\{-0.5,0,0.5,0.75\}\times\lambdans=\{500,800,
1000,1500,2000\}\times\rho=\{10,20,30,50,70\}$.
% 
This way, we only have to run full Bayesian parameter estimation analysis on
this fixed set of signals. 
% 
There are two sources of error that enter the deductions we make from
a single population generated in the manner described above. First, since the
injection parameters are pushed to their nearest neighbor on a grid, we
find discrete jumps in $\lambdans$ errors as a function of $N$. And second, an
individual population is one particular realization of a stochastic process and
could have excursions that may never be found in another population. To
account for both of these limitations, we generate an ensemble of populations,
and conservatively combine information from all of them\footnote{See 
Sec.~\ref{s1:multiple_observations} for further details.}.




We probe two astrophysical paradigms, one that allows for BH masses to lie
within the astrophysical mass-gap (paradigm A), and one that does not (paradigm
B).
% 
{\it For paradigm A}, we find the following: (i) for the softer equations of
state that result in less deformable neutron stars, $15-20$ detections bring
the measured probability distribution for $\lambdans$ entirely within the prior,
which ensures that the median $\lambdans$ tracks the true value to within $10\%$.
(ii) For NSBH populations with more deformable NSs ($\lambdans> 1000$),
the same is achievable within as few as $10$ (or $15$ at most) realistic
observations. (iii) The statistical uncertainty associated with $\lambdans$
measurement can be restricted to be within $\pm50\%$ using $10-20$ observations
when $\lambdans> 1000$), and using $25-40$ observations for softer equations
of state. All of the above is possible within a few years of design
aLIGO operation~\cite{Abadie:2010cfa}, if astrophysical BHs are allowed
masses $< 5M_\odot$ (i.e. in the mass-gap). However, further
restricting $\lambdans$ will require $50+$ NSBH observations.
% 
{\it For paradigm B}, we find the information accumulation to be somewhat slower.
While the quantitative inferences for populations with $\lambdans>1000$ are
not affected significantly, we find that $\lambdans< 1000$ populations require 
$10-20\%$ more events to attain the same measurement accuracy as under
paradigm A. In either case, the accumulation of information from NSBH signals
is likely slower than from BNS inspirals~\cite{Mandel:2009pc,Lackey2014,
Wade:2014vqa}, although template models for the latter may be more
uncertain due to missing point-particle PN terms at orders comparable to
the tidal terms.
% 
We conclude that within as few as $20-30$ observations of disruptive NSBH
mergers, aLIGO will begin to place interesting bounds on NS deformability.
This, amongst other things, will allow us to rank different equations of 
state for neutron star matter from most to least likely, within a few years'
detector operation.
% 
Our methods and results are detailed in Sec.~\ref{s1:multiple_observations}.









Finally, we note that the underlying numerical simulations used to calibrate
the waveform model used here have not been verified against
independent codes so far.
% 
It is therefore difficult to assess the combined modeling error of LEA+ and its
effect
on our results. Our results here are, therefore, limited by the limitations of
our waveform model, and presented with this caveat. However, we do expect the
combined effect of modeling errors to {\it not} affect our {\it qualitative}
conclusions, especially since the underlying point-particle component of LEA+
includes all high-order terms, unlike past BNS studies~\cite{Lackey2014,
Wade:2014vqa}
% 
In future, we plan to further the results presented here by using more recent
tidal models~\cite{Pannarale:2015jka,Hinderer:2016eia}, that
may improve upon LEA+\footnote{One of them~\cite{Pannarale:2015jka} is only an
amplitude model though, which has to be augmented with a compatible phase model
first.}.




%%%%%%%%%%%%%%%%%%%%%%%%%%%%%%%%%%%%%%%%%%%%%%%%%%%%%%%%%%%%%%%%%%%%%%%%%%%%%%%
% Acknowledgments
%%%%%%%%%%%%%%%%%%%%%%%%%%%%%%%%%%%%%%%%%%%%%%%%%%%%%%%%%%%%%%%%%%%%%%%%%%%%%%%
\begin{acknowledgments}
We thank Ben Lackey, Francesco Pannarale, Francois Foucart, and Duncan Brown
    for helpful discussions. We gratefully acknowledge support
  for this research at CITA from NSERC of Canada, the Ontario Early 
  Researcher Awards Program, the Canada Research
  Chairs Program, and the Canadian Institute for Advanced Research; at
  Caltech from the Sherman Fairchild Foundation and NSF grants
  PHY-1404569 and AST-1333520; at Cornell from the
  Sherman Fairchild Foundation and NSF grants PHY-1306125 and
  AST-1333129; and at Princeton from NSF grant PHY-1305682 and the
  Simons Foundation.  Calculations were performed at the Vulcan
  supercomputer at the Albert Einstein Institute;
  H.P. and P.K. thank the Albert-Einstein Institute,
  Potsdam, for hospitality during part of the time where this research
  was completed. M.P. thanks CITA for hospitality where part of the work
  was carried out.
\end{acknowledgments}

%%%%%%%%%%%%%%%%%%%%%%%%%%%%%%%%%%%%%%%%%%%%%%%%%%%%%%%%%%%%%%%%%%%%%%%%%%%%%%%
%%%%%%%%%%%%%%%%%%%%%%%%%%%%%%%%%%%%%%%%%%%%%%%%%%%%%%%%%%%%%%%%%%%%
\begin{appendix}

\section{Statistical uncertainty in measuring non-tidal parameters}\label{as1:nontidalerrors}
% 
% #################
\begin{figure*}
\centering 
\textbf{Statistical measurement uncertainty for NSBH parameters $\mchirp$ (top),
$\eta$ (middle) and $\chibh$ (bottom);\newline ignoring tidal effects.}\par\medskip
\includegraphics[trim = {2cm 0 0 0},width=2.\columnwidth]{plots/TNMchirpCIWidths90_0_Lambda_SNR}\\
\includegraphics[trim = {2cm 0 0 0},width=2.\columnwidth]{plots/TNEtaCIWidths90_0_Lambda_SNR}\\
\includegraphics[trim = {2cm 0 0 0},width=2.\columnwidth]{plots/TNChiBHCIWidths90_0_Lambda_SNR}
\caption{
We show here the statistical uncertainty associated with our measurement of
non-tidal parameters $\mchirp, \eta,$ and $\chibh$ (at $90\%$ credibility),
over the signal parameter space. Individual panels show the same as a function
of BH mass and spin. Across each row, we see the effect of increasing signal
strength (i.e. SNR) with the tidal deformability of the NS $\lambdans$ fixed.
Down each column, we see the effect of increasing $\lambdans$, at fixed SNR.
Tidal effects are ignored in templates.
}
\label{fig:CIWidths90_Lambda_SNR}
\end{figure*}
% #################
%%
In Fig.~\ref{fig:CIWidths90_Lambda_SNR}, we show how {\it precisely} can we
measure non-tidal NSBH parameters $X=\{\mchirp,\eta,\chibh\}$ using BBH templates.
The three panels correspond to $\mchirp$ (top), $\eta$ (middle), and $\chibh$
(bottom), and show the width of these credible intervals $(\Delta X)^{90\%}$
as a function of BH mass/spin (within each sub-panel), and NS properties, i.e.
$\lambdans$ (downwards in each column)~\footnote{We restrict NS mass to
$1.35M_\odot$ and its spin to zero. Varying its tidal deformability $\lambdans$
does not significantly change the measurement uncertainties for non-tidal
binary parameters, as is evident from comparing the two rows in each panel of
Fig.~\ref{fig:CIWidths90_Lambda_SNR}.}.
% 
From the left-most column, we find that: (i) at $\rho=20$ the chirp mass is
measured remarkably well - to a precision of $0.16\%$ of its true value, and
(ii) so is $\chibh$. (iii) The dimensionless mass-ratio $\eta$ is determined
more loosely, with $25+\%$ uncertainty. If the signal is even louder
($\rho\geq 30$), all three measurements gain further precision, especially
$\eta$, for which the relative errors shrink down to single-digit percents.
% % 
We remind ourselves that these results do not tell the full story since the
precision of a measurement is only meaningful if the measurement is accurate 
to begin with. In our case there are tidal effects that have not been
incorporated into our search (BBH) templates, which can lead to a systematic
bias in parameter recovery. We refer the reader to Sec.~\ref{s1:PEwithnoNS} for
a comparative study of both systematic and statistical errors.




\section{Illustrations of Bayesian posteriors}\label{as1:illustrations}
% 
\begin{figure*}
\centering
\textbf{Illustrative posterior probability distributions for NSBH parameters,}\par
\textbf{for signals at different SNR values; $\rho = 30$ (left) and $50$ (right).}\par\medskip
\includegraphics[width=1.05\columnwidth,trim=2cm 0 0 0]{plots/AllParamsMcEtPDF1D2D_q4_mc2_25_chi0_50_snr30}%\\
\includegraphics[width=1.05\columnwidth,trim=2cm 0 0 0]{plots/AllParamsMcEtPDF1D2D_q4_mc2_25_chi0_50_snr50}%\\
% \includegraphics[width=0.5\columnwidth]{plots/McEta2D_q4_mc2_25_chi0_50_snr30}
% \includegraphics[width=0.5\columnwidth]{plots/EtaChiBH2D_q4_mc2_25_chi0_50_snr30}
% \includegraphics[width=0.5\columnwidth]{plots/McChiBH2D_q4_mc2_25_chi0_50_snr30}\\
% \includegraphics[width=0.5\columnwidth]{plots/LambdaMc2D_q4_mc2_25_chi0_50_snr30}
% \includegraphics[width=0.5\columnwidth]{plots/LambdaEta2D_q4_mc2_25_chi0_50_snr30}
% \includegraphics[width=0.5\columnwidth]{plots/LambdaChiBH2D_q4_mc2_25_chi0_50_snr30}\\
% \includegraphics[width=0.47\columnwidth]{plots/LambdaMc2D_q4_mc2_25_chi0_50_snr50}
% \includegraphics[width=0.47\columnwidth]{plots/LambdaEta2D_q4_mc2_25_chi0_50_snr50}
% \includegraphics[width=0.47\columnwidth]{plots/LambdaChiBH2D_q4_mc2_25_chi0_50_snr50}
\caption{%
% We illustrate here two-dimensional joint probability distributions, as
% recovered for the intrinsic parameters of a binary with
% $q = \mbh/\mns = 5.4M_\odot/1.35M_\odot = 4$, $\chibh=+0.5$,
% $\lambdans=2000$, and $\rho=30$ (same as in
% Fig.~\ref{fig:SingleSystemLambdaPDFvsSNR}).
% The top row shows the same for combinations of 
% non-tidal parameters, i.e. binary masses and spins, while the middle row 
% shows combinations of $\lambdans$ with non-tidal parameters. The bottom row
% is similar to the middle, with the only difference being that the SNR has been
% turned up to $\rho=50$.
We illustrate here two sets of two-dimensional joint probability distributions,
differing only in signal strength, with $\rho=30$ in the left panel, and
$\rho=50$ in the right. The injected parameters are 
$q = \mbh/\mns = 5.4M_\odot/1.35M_\odot = 4$, $\chibh=+0.5$, and 
$\lambdans=2000$. Contours are shown for $\{1-,2-,3-\}\sigma$ confidence levels.
Templates include tidal effects, as evident in the bottom rows
of both panels which show the correlation of $\lambdans$ with non-tidal 
parameters. Contrasting the two panels illustrates the effect of increasing the
SNR on various parameter measurements.
}
\label{fig:SingleSystemLambda2DPDFs}
\end{figure*}
% #################


% #################
% \begin{figure*}
% \centering    
% \textbf{Improvement in $\lambdans$ measurement accuracy with multiple observations}\par\medskip
% \includegraphics[width=1.05\columnwidth,trim=1cm 0 0 0]{plots/pdfLambda_vs_N_L800.pdf}
% \includegraphics[width=1.\columnwidth,trim=0 0 1cm 0]{plots/FillBetweenErrorBarsLambda_vs_N_L800.pdf}
% \caption{%
% These panels show the accumulation of information from a population whose
% $\lambdans=800$.
% {\it Left:} Posterior probability distributions for $\lambdans$ (colored 
% curves), and associated $90\%$ credible intervals (grey vertical lines),
% shown for different number of accumulated observations $N$. Distributions are
% normalized to unit area.
% {\it Right:} Measured median value of $\lambdans$ (as solid circles) and the 
% associated $90\%$ credible intervals (as the vertical extent of filled
% region), shown as a function of number of observations $N$. Solid horizontal
% line indicates the true value of $\lambdans$. Dashed and dotted horizontal
% lines (a pair for each line-style) demarcate $\pm 25\%$ and $\pm 50\%$ error
% bounds.
% }
% \label{fig:TT_Lambda_vs_N_L800_CI90_0}
% \end{figure*}
% %
% #################





In Fig.~\ref{fig:SingleSystemLambda2DPDFs} we show the correlation of
mass, spin, and tidal parameter measurements. We keep the binary parameters
as in Fig.~\ref{fig:SingleSystemLambdaPDFvsSNR}, with $\lambdans=2000$, and
set $\rho=30$ (left panel) or $\rho=50$ (right panel).
%
We find that the measurement of $\lambdans$ is weakly degenerate with
other parameters, and at realistic SNRs it would improve by a few tens of 
percent if we knew non-tidal parameters to better accuracy. The predominant
factor that would enhance the measurement accuracy for $\lambdans$ is 
nevertheless the signal strength. Only when $\rho\gtrsim 50$ can we
expect $\lambdans$ measurement to be limited by its degeneracy with 
non-tidal parameters (at a factor of few level), as also reported by previous
studies~\cite{Lackey:2013axa}.







% ###########################################################
% ###########################################################
\section{Phenomenology of $\lambdans$ measurement errors}
% ###########################################################
% 
\begin{figure}
\centering    
\includegraphics[width=1.05\columnwidth]{plots/PowerLawCoefficient_LambdaErrorvsN_vs_N.pdf}
\caption{%
Assuming a power-law dependence of the measurement error on the number of
events: $\delta\lambdans\propto 1/N^\alpha$, we show $\alpha$ in this figure
as a function of the number of observed events $N$. Shown are five families
of $100$ population draws each, with each family corresponding to one of
$\lambdans=\{500,800,1000,1500,2000\}$. Each grey curve corresponds to one
of these $100\times5 = 500$ populations. The thicker curves, one from each
family, shows the population we discussed in
Fig.~\ref{fig:TT_Lambda_vs_N_L800_CI90_0}-\ref{fig:TT_Lambda_vs_N_CI90_0}.
We find that a power-law is a good approximation for the concerned dependence,
and information accumulates {\it faster} than $1/\sqrt{N}$. We estimate
$\alpha\simeq 0.7^{+0.2}_{-0.2}$.
}
\label{fig:TT_PowerLawLambdaErrorVsN}
\end{figure}
%
% 
\begin{figure}
\centering    
% \includegraphics[width=1.03\columnwidth]{plots/LambdaRelErrorBars_vs_Lambda_AllPopulations_N80_Log.pdf}
\includegraphics[width=\columnwidth]{plots/PowerLawCoefficient_LambdaErrorvsLambda_vs_N_AllPopulations.pdf}
\caption{%
In this figure, which is similar to Fig.~\ref{fig:TT_PowerLawLambdaErrorVsN},
we quantify the dependence of $\delta\lambdans$ on $\lambdans$ itself. Of 
the five families of simulated NSBH populations, we construct $100$
independent sets taking one population from each family. With each of 
these $100$ sets, and assuming a power-law dependence:
$\delta\lambdans\propto\lambdans^\beta$, we estimate $\beta$ and show it in
this figure as a function of the number of observed events $N$. The thicker
curve corresponds to the populations discussed in
Fig.~\ref{fig:TT_Lambda_vs_N_CI90_0}.
% 
We find that $\beta$ can be estimated to lie within $[1/6,5/6]$ with a
likely value close to $1/2$. Since $0<\beta<1$, the relative error
$\delta\lambdans/\lambdans$ {\it decreases} as the star gets more 
deformable, while the absolute error $\delta\lambdans$ {\it increases}.
}
\label{fig:TT_PowerLawLambdaErrorVsLambda}
\end{figure}
% 
Here, we quantitatively explore the dependence of our statistical
uncertainties for $\lambdans$ on the number of events, as well as on the true
NS deformability itself. First, we will focus on the dependence on $N$. We
assume a power-law dependence of the form
$\delta\lambdans\propto\ 1/N^\alpha$. For each of the $100$ populations 
for each of $\lambdans=500-2000$, we compute the exponent $\alpha$ as a
function of the number of observed events $N$, and show it in 
Fig.~\ref{fig:TT_PowerLawLambdaErrorVsN}. There are $100\times5=500$ curves
on the figure, with one highlighted for each value of population's $\lambdans$.
These highlighted values are only special in the sense that they correspond to
populations discussed earlier in this section (c.f.
Fig.~\ref{fig:TT_Lambda_vs_N_L800_CI90_0}-\ref{fig:TT_Lambda_vs_N_CI90_0}).
We immediately observe two things, (i) there is a globally similar dependence
on $N$ for all populations, and (ii) information accumulates {\it faster} than
$1/\sqrt{N}$. In fact, we find that if
$\delta\lambdans\propto\frac{1}{N^\alpha}$, $\alpha$ lines in the range
$0.7_{-0.2}^{+0.2}$.
% 
Next, we focus on the dependence of $\delta\lambdans$ on $\lambdans$ of the
population itself. As suggested by Fisher-matrix studies~\cite{Lackey:2013axa},
and as for $N$, we assume the form $\delta\lambdans\propto\lambdans^\beta$.
From each set of $100$ populations with a given $\lambdans$ value, we draw one
at random, and form a set of $5$ similarly drawn populations, one for each of
$\lambdans=\{500,800,1000,1500,2000\}$. With each set, we determine $\beta$
for different number of observed events $N$. In all, we make $100$ independent
$5-$population sets and show the value of $\beta$ measured from each in 
Fig.~\ref{fig:TT_PowerLawLambdaErrorVsLambda}. We find that the assumed
relation $\delta\lambdans\propto\lambdans^\beta$ gets fairly robust for 
larger values of $N$, with $\beta$ converging to $\beta=0.5^{+0.33}_{-0.33}$.
The fact that $0<\beta<1$ implies that the relative error
$\delta\lambdans/\lambdans$ {\it decreases} with increasing $\lambdans$, while
the absolute error {\it increases}.
% 
From these results, we conclude that the measurement uncertainty for
$\lambdans$ after $N$ observations is
\begin{equation}
 \delta\lambdans\propto \dfrac{\lambdans^{0.5^{+0.33}_{-0.33}}}{N^{0.7_{-0.2}^{+0.2}}}.
\end{equation}
We also find that while these results are inferred from paradigm A populations,
paradigm B gives very similar results.


\end{appendix}
%%%%%%%%%%%%%%%%%%%%%%%%%%%%%%%%%%%%%%%%%%%%%%%%%%%%%%%%%%%%%%%%%%%%%%%%%%%%%%%




%%%%%%%%%%%%%%%%%%%%%%%%%%%%%%%%%%%%%%%%%%%%%%%%%%%%%%%%%%%%%%%%%%%%%%%%%%%%%%%
\section*{References}
%%%%%%%%%%%%%%%%%%%%%%%%%%%%%%%%%%%%%%%%%%%%%%%%%%%%%%%%%%%%%%%%%%%%%%%%%%%%%%%
\bibliography{References/References}

\end{document}
