\documentclass[aps,prd,amsmath,floats,floatfix, twocolumn,
superscriptaddress,nofootinbib,showpacs]{revtex4-1}

\usepackage[colorlinks, pdfborder={0 0 0}, plainpages=false]{hyperref}
\usepackage{graphicx}
\usepackage{xspace}
\usepackage[usenames,dvipsnames]{color}
\usepackage{amssymb}

\newcommand{\roughly}{\mathchar"5218\relax} % Different from \sim in spacing

% Macros for text changes
\newcommand{\red}{\textcolor{red}}
\newcommand{\dan}[1]{\textcolor{WildStrawberry}{#1}}
\newcommand{\Mark}[1]{\textcolor{Cerulean}{#1}}
\newcommand{\larry}[1]{\textcolor{OliveGreen}{#1}}
\newcommand{\bela}[1]{\textcolor{Blue}{#1}}
\newcommand{\saul}[1]{\textcolor{Orange}{#1}}

% Macros for text notes and comments
\newcommand{\Note}[1]{\textcolor{blue}{\textbf{[#1]}}}


\newcommand{\Caltech}{\affiliation{Theoretical Astrophysics 350-17,
    California Institute of Technology, Pasadena, CA 91125, USA}}
\newcommand{\Cornell}{\affiliation{Center for Radiophysics and Space
    Research, Cornell University, Ithaca, New York 14853, USA}}
\newcommand{\CITA}{\affiliation{Canadian Institute for Theoretical
    Astrophysics, 60 St.~George Street, University of Toronto,
    Toronto, ON M5S 3H8, Canada}} %
\newcommand{\GWPAC}{\affiliation{Gravitational Wave Physics and
    Astronomy Center, California State University Fullerton,
    Fullerton, California 92834, USA}} %


%%%%%%%%%%%%%%%%%%%%%%%%%%%%%%%%%%%%%%%%%%%%%%%%%%%%%%%%%%%%%%%

\begin{document}

\title{
Measuring matter effects in Neutron star - Black hole binaries with Advanced LIGO
}

\author{People across the Atlantic}

\date{\today}

\begin{abstract}
Abstract
\end{abstract}

\pacs{}
% 04.25.D- Numerical relativity
% 04.25.dg Numerical studies of black holes and black-hole binaries
% 04.25.Nx Post-Newtonian approximation; perturbation theory; related approximations 
% 04.30.-w Gravitational waves (see also 04.80.Nn Gravitational wave detectors and experiments)
% 04.30.Db Wave generation and sources 
% 02.70.Hm Spectral methods

\maketitle

%%%%%%%%%%%%%%%%%%%%%%%%%%%%%%%%%%%%%%%%%%%%%%%%%%%%%%%%%%%%%%%%%%%%%%%%%%%%%%%
\section{Introduction}
%%%%%%%%%%%%%%%%%%%%%%%%%%%%%%%%%%%%%%%%%%%%%%%%%%%%%%%%%%%%%%%%%%%%%%%%%%%%%%%
Current status:

\begin{enumerate}
\item We choose the following values of different NS-BH binary parameters:
\begin{enumerate}
\item $m_\mathrm{NS}=1.35M_\odot$;
\item $q=m_\mathrm{BH}/\mathrm{NS}=\{2,3\}$;
\item $\chi_\mathrm{NS}=0$;
\item $\chi_\mathrm{BH}=\{-0.5, 0, +0.5\}$;
\item $\Lambda =\{500,1000,2000\}$.
\end{enumerate}

\item For each system comprised of all unique combinations of the above 
parameter choices, we carry out three parameter estimation tests: 
\begin{enumerate}
\item BH-BH signals injected, recovered with BH-BH waveforms (control),
\item NS-BH signals injected, recovered with BH-BH waveforms, (current approach) and
\item NS-BH signals injected, recovered with NS-BH waveforms.
\end{enumerate}
Injections were made at different SNR values: 
\begin{enumerate}
\item $\rho = \{20, 30, 50, 70, 90, 120\}$.
\end{enumerate}
Choice of emcee parameters:
\begin{enumerate}
\item $N_\mathrm{samples}=150,000$; 
\item $N_\mathrm{walkers}=100$; 
\item $N_\mathrm{burn-in}=500$; 
\end{enumerate}
and choice of prior boundaries:
\begin{enumerate} 
\item $m_\mathrm{BH}\in [1.2, 25]M_\odot$;
\item $m_\mathrm{NS}\in [1.2, 15]M_\odot$;
\item $\Lambda\leq 4000$; 
\item $\sigma\left(\Lambda\right) = 100$ (for chains where templates are NS-BH waveforms).
\end{enumerate}

\item Results are shown in
Fig.~\ref{fig:TNT_chirpMassBias_vs_SNR_q23}-~\ref{fig:TT_NSLambdaCIWidth90_vs_SNR_q23}.

\end{enumerate}
%%%%%%%%%%%%%%%%%%%%%%%%%%%%%%%%%%%%%%%%%%%%%%%%%%%%%%%%%%%%%%%%%%%%%%%%%%%%%%%
\section{Techniques}
%%%%%%%%%%%%%%%%%%%%%%%%%%%%%%%%%%%%%%%%%%%%%%%%%%%%%%%%%%%%%%%%%%%%%%%%%%%%%%%
\subsection{Modeling tidal deformation during inspiral}
Describe the PN terms used here

\subsection{Modeling tidal disruption near merger}
Describe the Lackey et al model

\subsection{MCMC methods}
Describe the emcee-based PE code

%%%%%%%%%%%%%%%%%%%%%%%%%%%%%%%%%%%%%%%%%%%%%%%%%%%%%%%%%%%%%%%%%%%%%%%%%%%%%%%
\section{How are detection searches affected by not including NS matter effects}
%%%%%%%%%%%%%%%%%%%%%%%%%%%%%%%%%%%%%%%%%%%%%%%%%%%%%%%%%%%%%%%%%%%%%%%%%%%%%%%
See what is the loss in SNR if we recover NSBH signals with a template bank of
BH-BH waveforms? It is very likely that the answer would be: $<1\%$. But it
is very interesting to show that for spinning BHs!

%%%%%%%%%%%%%%%%%%%%%%%%%%%%%%%%%%%%%%%%%%%%%%%%%%%%%%%%%%%%%%%%%%%%%%%%%%%%%%%
\section{How is PE affected by not including NS matter effects?}
%%%%%%%%%%%%%%%%%%%%%%%%%%%%%%%%%%%%%%%%%%%%%%%%%%%%%%%%%%%%%%%%%%%%%%%%%%%%%%%
\begin{figure*}
\centering 
% \textbf{BH-BH Injection; BH-BH Templates}
% \includegraphics[width=1.7\columnwidth]{plots/NN_chirpMassBias_vs_SNR_q23.pdf}\\ 
\textbf{NS-BH Injection; BH-BH Templates}
\includegraphics[width=1.7\columnwidth]{plots/TN_chirpMassBias_vs_SNR_q23.pdf}\\ 
\textbf{NS-BH Injection; NS-BH Templates}
\includegraphics[width=1.7\columnwidth]{plots/TT_chirpMassBias_vs_SNR_q23.pdf}%\quad
\caption{These figures show the (fractional) difference between the chirp mass 
value corresponding to the median of its posterior probability distribution,
and the actual injected chirp mass,
as a function of the signal-to-noise-ratio (SNR) of the injected signal.
% In the top row, NS matter effects on the inspiral and merger waveform
% are neithed included in templates nor in the injected signal. 
In the top row, NS matter effects (inspiral and merger) are included in the
injections but not in templates. This corresponds to the current 
plan for aLIGO parameter estimation studies. The bottom row shows the effect
of additionally including NS matter effects in the templates.
% 
In each row, the left and right panel correspond to 
$q=m_\mathrm{BH}/m_\mathrm{NS}=\{2,3\}$, respectively. 
In each panel, the color of each curve corresponds to the value of NS 
deformability parameter $(\Lambda=\{500,1000,2000\})$, 
and line-style corresponds to the value of
the dimensionless spin on the BH ($\chi_\mathrm{BH}=\{-0.5,0,+0.5\}$).
}
\label{fig:TNT_chirpMassBias_vs_SNR_q23}
\end{figure*}
% 
\begin{figure*}
\centering 
% \textbf{BH-BH Injection; BH-BH Templates}
% \includegraphics[width=1.7\columnwidth]{plots/NN_chirpMassCIWidth90_vs_SNR_q23.pdf}\\ 
\textbf{NS-BH Injection; BH-BH Templates}
\includegraphics[width=1.7\columnwidth]{plots/TN_chirpMassCIWidth90_vs_SNR_q23.pdf}\\ 
\textbf{NS-BH Injection; NS-BH Templates}
\includegraphics[width=1.7\columnwidth]{plots/TT_chirpMassCIWidth90_vs_SNR_q23.pdf}%\quad
\caption{These figures are similar to Fig.~\ref{fig:TNT_chirpMassBias_vs_SNR_q23}
with the difference that here we show the width of the $90\%$ confidence 
interval (recovered) for chirp mass, as a function of the injected SNR. 
Note that the confidence interval's width is normalized by the injected
parameter value.
}
\label{fig:TNT_chirpMassCIWidth90_vs_SNR_q23}
\end{figure*}
% 
\newpage
\newpage

\begin{figure*}[!t]
\centering    
\textbf{NS-BH Injection; BH-BH Templates}
\includegraphics[width=1.7\columnwidth]{plots/TN_EtaBias_vs_SNR_q23.pdf}\\ 
\textbf{NS-BH Injection; NS-BH Templates}
\includegraphics[width=1.7\columnwidth]{plots/TT_EtaBias_vs_SNR_q23.pdf}\\%\quad
\textbf{BH-BH Injection; BH-BH Templates}
\includegraphics[width=1.7\columnwidth]{plots/NN_EtaBias_vs_SNR_q23.pdf} 
\caption{This figure (top two rows) is similar to Fig.~\ref{fig:TNT_chirpMassBias_vs_SNR_q23},
with the difference that we consider the systematic bias in the dimensionless
mass-ratio $\eta$ here. In the bottom row we show control runs where both
signal and template waveforms lack matter effects.}
\label{fig:TNT_EtaBias_vs_SNR_q23}
\end{figure*}
% 
\begin{figure*}
\centering    
\textbf{NS-BH Injection; BH-BH Templates}
\includegraphics[width=1.7\columnwidth]{plots/TN_EtaCIWidth90_vs_SNR_q23.pdf}\\ 
\textbf{NS-BH Injection; NS-BH Templates}
\includegraphics[width=1.7\columnwidth]{plots/TT_EtaCIWidth90_vs_SNR_q23.pdf}\\%\quad
\textbf{BH-BH Injection; BH-BH Templates}
\includegraphics[width=1.7\columnwidth]{plots/NN_EtaCIWidth90_vs_SNR_q23.pdf} 
\caption{This figure (top two rows) is similar to Fig.~\ref{fig:TNT_chirpMassCIWidth90_vs_SNR_q23},
  with the difference that we consider the confidence intervals in the recovery
  of the dimensionless mass-ratio $\eta$ here. The bottom row here is similar to 
  Fig.~\ref{fig:TNT_EtaBias_vs_SNR_q23} with the difference that width of the $90\%$
  confidence interval is shown here.}
\label{fig:TNT_EtaCIWidth90_vs_SNR_q23}
\end{figure*}
% 
\begin{figure*}
\centering    
\textbf{NS-BH Injection; BH-BH Templates}
\includegraphics[width=1.7\columnwidth]{plots/TN_BHspinBias_vs_SNR_q23.pdf}\\ 
\textbf{NS-BH Injection; NS-BH Templates}
\includegraphics[width=1.7\columnwidth]{plots/TT_BHspinBias_vs_SNR_q23.pdf}\\%\quad
\textbf{BH-BH Injection; BH-BH Templates}
\includegraphics[width=1.7\columnwidth]{plots/NN_BHspinBias_vs_SNR_q23.pdf}
\caption{The top two rows in this figure are
similar to Fig.~\ref{fig:TNT_chirpMassBias_vs_SNR_q23},
with the differences that we consider the systematic bias in the dimensionless
spin on the BH here, and that the differences shown are absolute and not
fractions of the injected spin. In the bottom row, we show a control run where
both injected and template waveforms correspond to binary black holes.}
\label{fig:TNT_BHspinBias_vs_SNR_q23}
\end{figure*}
% 
\begin{figure*}
\centering    
\textbf{BH-BH Injection; BH-BH Templates}
\includegraphics[width=1.7\columnwidth]{plots/NN_BHspinCIWidth90_vs_SNR_q23.pdf}\\ 
\textbf{NS-BH Injection; BH-BH Templates}
\includegraphics[width=1.7\columnwidth]{plots/TN_BHspinCIWidth90_vs_SNR_q23.pdf}\\ 
\textbf{NS-BH Injection; NS-BH Templates}
\includegraphics[width=1.7\columnwidth]{plots/TT_BHspinCIWidth90_vs_SNR_q23.pdf}
\caption{The top two rows in the figure are similar to
Fig.~\ref{fig:TNT_chirpMassCIWidth90_vs_SNR_q23},
with the difference that we consider the confidence intervals in the recovery
of the dimensionless spin on the BH $\chi_\mathrm{BH}$ here.
As in Fig.~\ref{fig:TNT_BHspinBias_vs_SNR_q23}, we show here the absolute 
width of the confidence interval without normalizing with the injected spin.
The bottom row show a control run where matter effects are ignored both 
in the injected signal and the template waveforms.}
\label{fig:TNT_BHspinCIWidth90_vs_SNR_q23}
\end{figure*}
% 
\begin{figure*}
\centering
\textbf{NS-BH Injection; NS-BH Templates}
\includegraphics[width=1.7\columnwidth]{plots/TT_NSLambdaBias_vs_SNR_q23.pdf}
\caption{This figure is similar to the bottom row of
Fig.~\ref{fig:TNT_chirpMassBias_vs_SNR_q23}, with the difference that  we
consider the NS tidal parameter $\Lambda$ here.}
\label{fig:TT_NSLambdaBias_vs_SNR_q23}
\end{figure*}
% 
\begin{figure*}
\centering
\textbf{NS-BH Injection; NS-BH Templates}
\includegraphics[width=1.7\columnwidth]{plots/TT_NSLambdaCIWidth90_vs_SNR_q23.pdf}
\caption{This figure is similar to the bottom row of
Fig.~\ref{fig:TNT_chirpMassCIWidth90_vs_SNR_q23}, with the difference that  
we consider the NS tidal parameter $\Lambda$ here.}
\label{fig:TT_NSLambdaCIWidth90_vs_SNR_q23}
\end{figure*}
% 
% 
Advanced LIGO searches and parameter estimation efforts aimed at binaries 
containing NS and stellar-mass BHs are poised to use waveform models that do not
include the effects of the NS tidal deformability~\cite{Canton:2014ena}.
While this is not expected to
be the dominant source of error at low SNRs, it will likely introduce a
systematic bias in the physical parameters that we recover from GW observations.
In this section, we study the impact of the same for binaries where the BH spins
are aligned to the orbital angular momentum.

\begin{enumerate}
\item Above what SNR values, below what mass-ratios, above what BH spins, etc, 
do we begin to care about NS deformability?\newline
``one interesting plot to make would be the parameter bias (max-Likelihood
parameters minus true parameters) vs SNR, as functionals of q/sBH/Lambda, when
using T signals recovered with N templates.''
\end{enumerate}


Could have subsections for aLIGO and ET ?

%%%%%%%%%%%%%%%%%%%%%%%%%%%%%%%%%%%%%%%%%%%%%%%%%%%%%%%%%%%%%%%%%%%%%%%%%%%%%%%
\section{What do we gain by using templates that include NS matter effects?}
%%%%%%%%%%%%%%%%%%%%%%%%%%%%%%%%%%%%%%%%%%%%%%%%%%%%%%%%%%%%%%%%%%%%%%%%%%%%%%%
Having shown in the previous section that we begin to care for NS tidal effects
for signals in the XXX corner of the paraemter space, with SNRs above YY, here
we investigate the effect of the improvement in the accuracy of the recovered
parameters when tidal effects are included in the templates.

\begin{enumerate}
\item What is the reduction in the bias of the maximum likelihood parameters
when using tidal templates?\newline
``Plot the ratio of the bias between N templates and T templates (against T
signals), as a function of SNR.''
\item What is the reduction in the uncertainty in binary mass and spin, if
any?\newline
``show how the the $90\%$ confidence intervals shrink, as a function of SNR, 
when we go from using N templates to T templates.''
\end{enumerate}


\red{Could have subsections for aLIGO and ET ?}


%%%%%%%%%%%%%%%%%%%%%%%%%%%%%%%%%%%%%%%%%%%%%%%%%%%%%%%%%%%%%%%%%%%%%%%%%%%%%%%
\section{Discussion}
%%%%%%%%%%%%%%%%%%%%%%%%%%%%%%%%%%%%%%%%%%%%%%%%%%%%%%%%%%%%%%%%%%%%%%%%%%%%%%%
Discussion

%%%%%%%%%%%%%%%%%%%%%%%%%%%%%%%%%%%%%%%%%%%%%%%%%%%%%%%%%%%%%%%%%%%%%%%%%%%%%%%
% Acknowledgments
%%%%%%%%%%%%%%%%%%%%%%%%%%%%%%%%%%%%%%%%%%%%%%%%%%%%%%%%%%%%%%%%%%%%%%%%%%%%%%%
\begin{acknowledgments}
Acknowledgments
\end{acknowledgments}

%%%%%%%%%%%%%%%%%%%%%%%%%%%%%%%%%%%%%%%%%%%%%%%%%%%%%%%%%%%%%%%%%%%%%%%%%%%%%%%
\section*{References}
%%%%%%%%%%%%%%%%%%%%%%%%%%%%%%%%%%%%%%%%%%%%%%%%%%%%%%%%%%%%%%%%%%%%%%%%%%%%%%%
\bibliography{References/References}

\end{document}
