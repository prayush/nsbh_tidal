\documentclass[aps,prd,amsmath,floats,floatfix, twocolumn,
superscriptaddress,nofootinbib,showpacs]{revtex4-1}

\usepackage[colorlinks, pdfborder={0 0 0}, plainpages=false]{hyperref}
\usepackage{graphicx}
\usepackage{xspace}
\usepackage[usenames,dvipsnames]{color}
\usepackage{amssymb}
\usepackage[dvipsnames]{xcolor}

\newcommand{\roughly}{\mathchar"5218\relax} % Different from \sim in spacing

% Macros for text changes
\newcommand{\red}{\textcolor{red}}
\newcommand{\dan}[1]{\textcolor{WildStrawberry}{#1}}
\newcommand{\Mark}[1]{\textcolor{Cerulean}{#1}}
\newcommand{\larry}[1]{\textcolor{OliveGreen}{#1}}
\newcommand{\bela}[1]{\textcolor{Blue}{#1}}
\newcommand{\saul}[1]{\textcolor{Orange}{#1}}
\newcommand{\prayush}{\textcolor{red!40!black}}

% Macros for text notes and comments
\newcommand{\Note}[1]{\textcolor{blue}{\textbf{[#1]}}}
\newcommand{\lambdans}{\Lambda_\mathrm{NS}}
\newcommand{\chibh}{\chi_\mathrm{BH}}
\newcommand{\chins}{\chi_\mathrm{NS}}
\newcommand{\mbh}{m_\mathrm{BH}}
\newcommand{\mns}{m_\mathrm{NS}}
\newcommand{\mchirp}{\mathcal{M}_c}

\newcommand{\Caltech}{\affiliation{Theoretical Astrophysics 350-17,
    California Institute of Technology, Pasadena, CA 91125, USA}}
\newcommand{\Cornell}{\affiliation{Center for Radiophysics and Space
    Research, Cornell University, Ithaca, New York 14853, USA}}
\newcommand{\CITA}{\affiliation{Canadian Institute for Theoretical
    Astrophysics, 60 St.~George Street, University of Toronto,
    Toronto, ON M5S 3H8, Canada}} %
\newcommand{\GWPAC}{\affiliation{Gravitational Wave Physics and
    Astronomy Center, California State University Fullerton,
    Fullerton, California 92834, USA}} %


    
\newcommand{\NS}{\mathrm{NS}}
%%%%%%%%%%%%%%%%%%%%%%%%%%%%%%%%%%%%%%%%%%%%%%%%%%%%%%%%%%%%%%%

\begin{document}

\title{
Measuring matter effects in Neutron star - Black hole binaries with Advanced LIGO
}

\author{People across the Atlantic}

\date{\today}

\begin{abstract}
We attempt to answer the following questions: 
\begin{enumerate}
 \item Will ignoring the effect of NS deformability during inspiral and NS 
 disruption during merger (in an NSBH binary) change the efficiency of our 
 searches?
 \item What systematic biases are introduced by neglecting the NS tidal
 disruption close to merger? 
 \item At what SNR do the systematic biases dominate over the statistical 
 errors in the recovery of binary parameters?
 \item By using templates which models NS deformation (over inspiral) and 
 NS disruption (close to merger), what information can we recover about
 the $\Lambda_\mathrm{BH}\sim (R/M)^5_\mathrm{NS}$ parameter?
\end{enumerate}
\end{abstract}

\pacs{}
% 04.25.D- Numerical relativity
% 04.25.dg Numerical studies of black holes and black-hole binaries
% 04.25.Nx Post-Newtonian approximation; perturbation theory; related approximations 
% 04.30.-w Gravitational waves (see also 04.80.Nn Gravitational wave detectors and experiments)
% 04.30.Db Wave generation and sources 
% 02.70.Hm Spectral methods

\maketitle

%%%%%%%%%%%%%%%%%%%%%%%%%%%%%%%%%%%%%%%%%%%%%%%%%%%%%%%%%%%%%%%%%%%%%%%%%%%%%%%
\section{Introduction}
%%%%%%%%%%%%%%%%%%%%%%%%%%%%%%%%%%%%%%%%%%%%%%%%%%%%%%%%%%%%%%%%%%%%%%%%%%%%%%%

The Advanced LIGO (aLIGO) observatories began their first observing run ``O1''
mid-September 2015, operating at a factor of $3-4$ higher gravitational-wave (GW) 
strain sensitivity than their first-generation counterparts~\cite{Shoemaker2009}.
Coalescing binaries of
compact objects, i.e. Black Holes (BH) and/or Neutron Stars (NS), are the most
likely targets for GW detection with these ground-based instruments. Over a
$3$-month O1, these detectors would have access to as
many binary mergers as initial-LIGO detectors did over approximately $6$ years
of coincident operation. Both of these instruments will be further upgraded in
phases to reach their design sensitivity by $2018-19$, at which point we expect to
observe $\sim 70$ binary mergers a year~\cite{Abadie:2010cf}.

NS-BH binaries are one of the most important sources of gravitational waves (GW)
with these ground-based instruments, and we expect to detect $\mathcal{O}(10)$
of such systems every year~\cite{Abadie:2010cf}. 
%
One question that aLIGO will likely probe is `what is the nature of
matter at nuclear densities', i.e. what is the equation of state (EoS) of 
neutron stars. The EoS determines the compactness of NSs, which determines
their deformation over the course of inspiral (albeit weakly, at $5$PN order in 
phasing~\prayush{CITE PN TIDAL RESULTS}), as well as how early in the coalescence
process will the NS be disrupted by the tidal field of the BH. Once disrupted,
the NS forms an accretion disk around the BH. Most of this matter disk spirals 
into the BH within a few milliseconds, with a small fraction becoming unbounded
and gets ejected~\prayush{CITE BHNS NR}. It is believed that an accompaning
short (lasting less than $2$ seconds) Gamma-ray Burst (sGRB) might be present
under certain physical conditions~\prayush{CITE NSBH-GRB},
making these systems one of the prime targets for joint electromagnetic-GW
astronomy searches~\prayush{CITE EM-GW}.
In such a situtation, the emitted GW signal
is heavily damped post-disruption, and the quasi-normal modes of the resulting
BH are not excited. Such a GW signal is qualitatively very different from a
BH-BH merger, and one of the main goals of this work is to determine the 
signal strength at which this difference will be quantifiabe by aLIGO.
%
However, studies have shown that for mass-ratio $\mbh/\mns\gtrsim 5$ the BH
will need to have fairly large and positively-aligned (with the orbit) spins
in order for its tidal field to be strong enough to disrupt the NS before the
innermost stable circular orbit (ISCO) is reached~\cite{Foucart:2013psa}.


Conventional astronomical 
methods have, to date, made observations of NSs with masses between 
$XX-YY M_\odot$ and spins $|\vec{\chi}_\mathrm{NS}|\lesssim 0.05$. On the other hand, 
indirect observations of stellar-mass BHs place their masses between 
$5-35M_\odot$, with their spin angular momenta $|\vec{\chi}_\mathrm{BH}|$ 
ranging from small to nearly extremal (Kerr) values (see,
e.g.~\cite{McClintockEtAl:2006,Miller:2009cw,Gou:2014una} for examples of 
nearly extremal estimates of BH spins, and see
Refs.~\cite{McClintock:2013vwa,Reynolds:2013qqa} for recent reviews of
astrophysical BH spin measurements). 



The difficulty in discerning the matter effects in an NSBH merger GW signal
lies in distinguishing it reliably from a BHBH merger signal. The earlier the
NS disrupts, the more orbits there are over which the effect of NS disruption
is visible. Where exactly does disruption happen is governed both by the 
mass-ratio of the binary, as well as the spin on the BH. Larger BHs (with
$q=m_\mathrm{BH}/m_\mathrm{NS}\gtrsim 7$), as well as BHs with large retrograde
spins ($\chi_\mathrm{BH} \lesssim 0$), tend to swallow the NS whole due to a
rapid merger (i.e. without it disrupting before the inner-most stable circular
orbit (ISCO) is reached). On the other hand, binaries with mass-ratios 
$q=2-5$ with large pro-grade $\chi_\mathrm{BH}$ can disrupt much before 
merger, leaving their imprint in the GW signal emitted close to $1.7-2$~kHz.
% 

The goal of this paper is to probe the distinguishability of BHBH and NSBH
mergers, focusing at the effects of the BH mass, BH spin, and NS tidal 
deformability. We also investigate the accuracy with which aLIGO can measure
the tidal deformability parameter $\Lambda_\NS\sim(R/M)^5_\NS$. For both of 
these studies, we will use the zero-detuning high-power noise curve for the
aLIGO detectors. \prayush{In addition, we also study the effect of tuning aLIGO
over a narrow frequency range aimed at NSBH mergers.}
We use the models published in Ref.~\cite{Lackey:2013axa,Pannarale:2015jka} to
simulate NSBH merger waveforms.



\vspace{2cm}

\textit{\prayush{Current status:}}

Below we explain the outline the different calculations performed in this 
paper:

\begin{enumerate}
\item We choose the following values of different NSBH binary parameters:
\begin{enumerate}
\item $m_\mathrm{NS}=1.35M_\odot$;
\item $q=m_\mathrm{BH}/\mathrm{NS}=\{2,3\}$;
\item $\chi_\mathrm{NS}=0$;
\item $\chi_\mathrm{BH}=\{-0.5, 0, +0.5\}$;
\item $\Lambda =\{500,1000,2000\}$ ($\Lambda=0\implies \mathrm{BHBH}$).
\end{enumerate}

\item For each system, corresponding to one of all possible unique combinations
of the above parameter choices, we carry out three parameter estimation tests: 
\begin{enumerate}
\item BH-BH signals injected, recovered with BH-BH waveforms (control),
\item NS-BH signals injected, recovered with BH-BH waveforms, (current approach) and
\item NS-BH signals injected, recovered with NS-BH waveforms.
\end{enumerate}
Injections were made at different SNR values: 
\begin{enumerate}
\item $\rho = \{20, 30, 50, 70, 90, 120\}$.
\end{enumerate}
Choice of emcee parameters:
\begin{enumerate}
\item $N_\mathrm{samples}=150,000$; 
\item $N_\mathrm{walkers}=100$; 
\item $N_\mathrm{burn-in}=500$; 
\end{enumerate}
and choice of prior boundaries:
\begin{enumerate} 
\item $m_\mathrm{BH}\in [1.2, 25]M_\odot$;
\item $m_\mathrm{NS}\in [1.2, 15]M_\odot$;
\item $\Lambda\leq 4000$; 
\item $\sigma\left(\Lambda\right) = 100$ (for chains where templates are NS-BH waveforms).
\end{enumerate}

\item Results are shown in
Fig.~\ref{fig:TNT_chirpMassBias_vs_SNR_q23}-~\ref{fig:TT_NSLambdaCIWidth90_vs_SNR_q23}.

\end{enumerate}
%%%%%%%%%%%%%%%%%%%%%%%%%%%%%%%%%%%%%%%%%%%%%%%%%%%%%%%%%%%%%%%%%%%%%%%%%%%%%%%
\section{Techniques}
%%%%%%%%%%%%%%%%%%%%%%%%%%%%%%%%%%%%%%%%%%%%%%%%%%%%%%%%%%%%%%%%%%%%%%%%%%%%%%%
\subsection{Modeling tidal deformation during inspiral}
Describe the PN terms used here

\subsection{Modeling tidal disruption near merger}
Describe the Lackey et al model

\subsection{MCMC methods}
Describe the emcee-based PE code

%%%%%%%%%%%%%%%%%%%%%%%%%%%%%%%%%%%%%%%%%%%%%%%%%%%%%%%%%%%%%%%%%%%%%%%%%%%%%%%
\section{How are detection searches affected by not including NS matter effects}
%%%%%%%%%%%%%%%%%%%%%%%%%%%%%%%%%%%%%%%%%%%%%%%%%%%%%%%%%%%%%%%%%%%%%%%%%%%%%%%

Near future detection searches with aLIGO are poised to search for NSBH
signals using BHBH templates. In this section, we examine the effect of ignoring
the effects of tidal deformation and disruption of the NS during inspiral and 
merger phases of search templates. We construct a template bank $\mathcal{B}$ 
restricted to the NSBH parameter space with $\mbh\in[2.5M_\odot, 20.5M_\odot]$, 
$\chibh\in[-0.5, +0.75]$, $\mns\in[1M_\odot,3M_\odot]$ and $\chi_\mathrm{NS}=0$,
using a PN-based geometric lattice placement method~\cite{Harry:2013tca}. 


Before investigating its SNR recovery against tidal NSBH signals, we test the 
effectualness of this bank to point-particle NSBH signals,
in order to make sure that the placement method does not leave any region of
the binary parameter-space under-covered. We simulate $50,000$ signals with 
component-mass and BH spin values
sampled uniformly over their allowed ranges, and filter each of them against
the entire bank. For both signals and templates, we use the frequency-domain 
reduced-order model (ROM) of the SEOBNRv2 approximant~\cite{Purrer:2014} 
implemented within the LIGO algorithms Library~\cite{lal}. We record the 
fitting-factors~\cite{FittingFactorApostolatos} recovered by the bank for all
signals, which measures the fraction of optimal SNR that the bank is able
to recover. Fig.~\ref{} shows these FFs as a function of the injected
BH mass and BH spin, and Fig.~\ref{} provides a histogram of the recovered FF 
values. From these figures, we conclude that our bank is effectual as intended.


Next, tn order to establish the effect of NS matter effects on the SNR recovery of 
detection searches, we perform a similar calculation as above, with the
difference that the signals are now modelled using the model of tidal 
corrections to IMR amplitude and phasing published in 
Ref.~\cite{Lackey:2013axa}, applied to SEOBNRv2 as the underlying BHBH model.




%%%%%%%%%%%%%%%%%%%%%%%%%%%%%%%%%%%%%%%%%%%%%%%%%%%%%%%%%%%%%%%%%%%%%%%%%%%%%%%
\section{How is PE affected by not including NS matter effects?}
%%%%%%%%%%%%%%%%%%%%%%%%%%%%%%%%%%%%%%%%%%%%%%%%%%%%%%%%%%%%%%%%%%%%%%%%%%%%%%%
% \begin{figure*}
% \centering 
% % \textbf{BH-BH Injection; BH-BH Templates}
% % \includegraphics[width=1.7\columnwidth]{plots/NN_chirpMassBias_vs_SNR_q23.pdf}\\ 
% \textbf{NS-BH Injection; BH-BH Templates}
% \includegraphics[width=1.7\columnwidth]{plots/TN_chirpMassBias_vs_SNR_q23.pdf}\\ 
% \textbf{NS-BH Injection; NS-BH Templates}
% \includegraphics[width=1.7\columnwidth]{plots/TT_chirpMassBias_vs_SNR_q23.pdf}%\quad
% \caption{These figures show the (fractional) difference between the chirp mass 
% value corresponding to the median of its posterior probability distribution,
% and the actual injected chirp mass,
% as a function of the signal-to-noise-ratio (SNR) of the injected signal.
% % In the top row, NS matter effects on the inspiral and merger waveform
% % are neithed included in templates nor in the injected signal. 
% In the top row, NS matter effects (inspiral and merger) are included in the
% injections but not in templates. This corresponds to the current 
% plan for aLIGO parameter estimation studies. The bottom row shows the effect
% of additionally including NS matter effects in the templates.
% % 
% In each row, the left and right panel correspond to 
% $q=m_\mathrm{BH}/m_\mathrm{NS}=\{2,3\}$, respectively. 
% In each panel, the color of each curve corresponds to the value of NS 
% deformability parameter $(\Lambda=\{500,1000,2000\})$, 
% and line-style corresponds to the value of
% the dimensionless spin on the BH ($\chi_\mathrm{BH}=\{-0.5,0,+0.5\}$).
% }
% \label{fig:TNT_chirpMassBias_vs_SNR_q23}
% \end{figure*}
% % 
% \begin{figure*}
% \centering 
% % \textbf{BH-BH Injection; BH-BH Templates}
% % \includegraphics[width=1.7\columnwidth]{plots/NN_chirpMassCIWidth90_vs_SNR_q23.pdf}\\ 
% \textbf{NS-BH Injection; BH-BH Templates}
% \includegraphics[width=1.7\columnwidth]{plots/TN_chirpMassCIWidth90_vs_SNR_q23.pdf}\\ 
% \textbf{NS-BH Injection; NS-BH Templates}
% \includegraphics[width=1.7\columnwidth]{plots/TT_chirpMassCIWidth90_vs_SNR_q23.pdf}%\quad
% \caption{These figures are similar to Fig.~\ref{fig:TNT_chirpMassBias_vs_SNR_q23}
% with the difference that here we show the width of the $90\%$ confidence 
% interval (recovered) for chirp mass, as a function of the injected SNR. 
% Note that the confidence interval's width is normalized by the injected
% parameter value.
% }
% \label{fig:TNT_chirpMassCIWidth90_vs_SNR_q23}
% \end{figure*}
% % 
% 
% \begin{figure*}[!t]
% \centering    
% \textbf{NS-BH Injection; BH-BH Templates}
% \includegraphics[width=1.7\columnwidth]{plots/TN_EtaBias_vs_SNR_q23.pdf}\\ 
% \textbf{NS-BH Injection; NS-BH Templates}
% \includegraphics[width=1.7\columnwidth]{plots/TT_EtaBias_vs_SNR_q23.pdf}\\%\quad
% \textbf{BH-BH Injection; BH-BH Templates}
% \includegraphics[width=1.7\columnwidth]{plots/NN_EtaBias_vs_SNR_q23.pdf} 
% \caption{This figure (top two rows) is similar to Fig.~\ref{fig:TNT_chirpMassBias_vs_SNR_q23},
% with the difference that we consider the systematic bias in the dimensionless
% mass-ratio $\eta$ here. In the bottom row we show control runs where both
% signal and template waveforms lack matter effects.}
% \label{fig:TNT_EtaBias_vs_SNR_q23}
% \end{figure*}
% % 
% \begin{figure*}
% \centering    
% \textbf{NS-BH Injection; BH-BH Templates}
% \includegraphics[width=1.7\columnwidth]{plots/TN_EtaCIWidth90_vs_SNR_q23.pdf}\\ 
% \textbf{NS-BH Injection; NS-BH Templates}
% \includegraphics[width=1.7\columnwidth]{plots/TT_EtaCIWidth90_vs_SNR_q23.pdf}\\%\quad
% \textbf{BH-BH Injection; BH-BH Templates}
% \includegraphics[width=1.7\columnwidth]{plots/NN_EtaCIWidth90_vs_SNR_q23.pdf} 
% \caption{This figure (top two rows) is similar to Fig.~\ref{fig:TNT_chirpMassCIWidth90_vs_SNR_q23},
%   with the difference that we consider the confidence intervals in the recovery
%   of the dimensionless mass-ratio $\eta$ here. The bottom row here is similar to 
%   Fig.~\ref{fig:TNT_EtaBias_vs_SNR_q23} with the difference that width of the $90\%$
%   confidence interval is shown here.}
% \label{fig:TNT_EtaCIWidth90_vs_SNR_q23}
% \end{figure*}
% \begin{figure*}
% \centering    
% \textbf{NS-BH Injection; BH-BH Templates}
% \includegraphics[width=1.7\columnwidth]{plots/TN_BHspinBias_vs_SNR_q23.pdf}\\ 
% \textbf{NS-BH Injection; NS-BH Templates}
% \includegraphics[width=1.7\columnwidth]{plots/TT_BHspinBias_vs_SNR_q23.pdf}\\%\quad
% \textbf{BH-BH Injection; BH-BH Templates}
% \includegraphics[width=1.7\columnwidth]{plots/NN_BHspinBias_vs_SNR_q23.pdf}
% \caption{The top two rows in this figure are
% similar to Fig.~\ref{fig:TNT_chirpMassBias_vs_SNR_q23},
% with the differences that we consider the systematic bias in the dimensionless
% spin on the BH here, and that the differences shown are absolute and not
% fractions of the injected spin. In the bottom row, we show a control run where
% both injected and template waveforms correspond to binary black holes.}
% \label{fig:TNT_BHspinBias_vs_SNR_q23}
% \end{figure*}
% % 
% \begin{figure*}
% \centering    
% \textbf{BH-BH Injection; BH-BH Templates}
% \includegraphics[width=1.7\columnwidth]{plots/NN_BHspinCIWidth90_vs_SNR_q23.pdf}\\ 
% \textbf{NS-BH Injection; BH-BH Templates}
% \includegraphics[width=1.7\columnwidth]{plots/TN_BHspinCIWidth90_vs_SNR_q23.pdf}\\ 
% \textbf{NS-BH Injection; NS-BH Templates}
% \includegraphics[width=1.7\columnwidth]{plots/TT_BHspinCIWidth90_vs_SNR_q23.pdf}
% \caption{The top two rows in the figure are similar to
% Fig.~\ref{fig:TNT_chirpMassCIWidth90_vs_SNR_q23},
% with the difference that we consider the confidence intervals in the recovery
% of the dimensionless spin on the BH $\chi_\mathrm{BH}$ here.
% As in Fig.~\ref{fig:TNT_BHspinBias_vs_SNR_q23}, we show here the absolute 
% width of the confidence interval without normalizing with the injected spin.
% The bottom row show a control run where matter effects are ignored both 
% in the injected signal and the template waveforms.}
% \label{fig:TNT_BHspinCIWidth90_vs_SNR_q23}
% \end{figure*}

% #################
\begin{figure*}
\centering 
\includegraphics[trim={{0.2\columnwidth} 0 0 0},width=2.1\columnwidth]{plots-TN/TN_MchirpBiases_Lambda_SNR.pdf}
\caption{Systematic fractional biases in recovered chirp-mass $\mchirp$, as a
fraction of the injected (true) value, for NSBH binaries, shown here as a 
function of the mass and spin of the BH (each panel), $\lambdans$ (across 
rows), and injection's SNR $\rho$ (across columns). Tidal effects during
inspiral and the effects of NS disruption close to merger are modelled in the
signal, but \textit{not} in the templates. 
}
\label{fig:TNT_chirpMassBias_vs_SNR_q23}
\end{figure*}
% 
\begin{figure*}
\centering
\includegraphics[trim={{0.4\columnwidth} 0 0 0},width=2.2\columnwidth]{plots-TN/TN_MchirpCIWidths90_0_Lambda_SNR.pdf}
\caption{$90\%$ confidence intervals for the recovered chirp-mass $\mchirp$, as a
fraction of the injected (true) value, for NSBH binaries, shown here as a 
function of the mass and spin of the BH (each panel), $\lambdans$ (across 
rows), and injection's SNR $\rho$ (across columns). Tidal effects during
inspiral and the effects of NS disruption close to merger are modelled in the
signal, but \textit{not} in the templates. 
}
\label{fig:TNT_chirpMassCIWidth90_vs_SNR_q23}
\end{figure*}
% 
\newpage
\newpage

\begin{figure*}[!t]
\centering    
\includegraphics[trim={{0.4\columnwidth} 0 0 0},width=2.2\columnwidth]{plots-TN/TN_EtaBiases_Lambda_SNR.pdf}
\caption{Systematic fractional biases in the recovered dimensionless mass-ratio
$\eta = m_1 m_2 / (m_1+m_2)^2$, as a
fraction of the injected (true) value, for NSBH binaries, shown here as a 
function of the mass and spin of the BH (each panel), $\lambdans$ (across 
rows), and injection's SNR $\rho$ (across columns). Tidal effects during
inspiral and the effects of NS disruption close to merger are modelled in the
signal, but \textit{not} in the templates. }
\label{fig:TNT_EtaBias_vs_SNR_q23}
\end{figure*}
% 
\begin{figure*}
\centering    
\includegraphics[trim={{0.4\columnwidth} 0 0 0},width=2.2\columnwidth]{plots-TN/TN_EtaCIWidths68_3_Lambda_SNR.pdf}
\caption{$90\%$ confidence intervals for the recovered dimensionless mass-ratio
$\eta = m_1 m_2 / (m_1+m_2)^2$, as a
fraction of the injected (true) value, for NSBH binaries, shown here as a 
function of the mass and spin of the BH (each panel), $\lambdans$ (across 
rows), and injection's SNR $\rho$ (across columns). Tidal effects during
inspiral and the effects of NS disruption close to merger are modelled in the
signal, but \textit{not} in the templates. }
\label{fig:TNT_EtaCIWidth90_vs_SNR_q23}
\end{figure*}
% 
\begin{figure*}
\centering
\includegraphics[trim={{0.4\columnwidth} 0 0 0},width=2.2\columnwidth]{plots-TN/TN_ChiBHBiases_CI90_0_Lambda_SNR.pdf}
\caption{Systematic fractional biases in the recovered value of the BH spin $\chibh$,
as a fraction of the injected (true) value, for NSBH binaries, shown here as a
function of the mass and spin of the BH (each panel), $\lambdans$ (across 
rows), and injection's SNR $\rho$ (across columns). Tidal effects during
inspiral and the effects of NS disruption close to merger are modelled in the
signal, but \textit{not} in the templates. }
\label{fig:TNT_BHspinBias_vs_SNR_q23}
\end{figure*}
% 
\begin{figure*}
\centering    
\includegraphics[trim={{0.4\columnwidth} 0 0 0},width=2.2\columnwidth]{plots-TN/TN_ChiBHCIWidths90_0_Lambda_SNR.pdf}
\caption{$90\%$ confidence intervals for the recovered value of the BH spin $\chibh$,
as a fraction of the injected (true) value, for NSBH binaries, shown here as a 
function of the mass and spin of the BH (each panel), $\lambdans$ (across 
rows), and injection's SNR $\rho$ (across columns). Tidal effects during
inspiral and the effects of NS disruption close to merger are modelled in the
signal, but \textit{not} in the templates.}
\label{fig:TNT_BHspinCIWidth90_vs_SNR_q23}
\end{figure*}
% 
% 
Advanced LIGO searches and parameter estimation efforts aimed at binaries 
containing NS and stellar-mass BHs are poised to use waveform models that do not
include the effects of the NS tidal deformability~\cite{Canton:2014ena}.
While this is not expected to
be the dominant source of error at low SNRs, it will likely introduce a
systematic bias in the physical parameters that we recover from GW observations.
In this section, we study the impact of the same for binaries where the BH spins
are aligned to the orbital angular momentum.

\begin{enumerate}
\item Above what SNR values, below what mass-ratios, above what BH spins, etc, 
do we begin to care about NS deformability?\newline
``one interesting plot to make would be the parameter bias (max-Likelihood
parameters minus true parameters) vs SNR, as functionals of q/sBH/Lambda, when
using T signals recovered with N templates.''\newline
`` another interesting plot to make would be the fitting-factors recovered with
a discrete (search oriented) template bank of BBH waveforms, for NSBH signals of
loudness, deformability, spins, mass-ratios, etc. Which parameter plays the 
dominant role in paramter regions where the bank does not suffice to recover
the desired threshold of optimal SNR?''
\end{enumerate}


\prayush{Could have subsections for different frequency-optimized aLIGO noise curves ?}

\hspace{5mm}



%%%%%%%%%%%%%%%%%%%%%%%%%%%%%%%%%%%%%%%%%%%%%%%%%%%%%%%%%%%%%%%%%%%%%%%%%%%%%%%
\section{What do we gain by using templates that include NS matter effects?}
%%%%%%%%%%%%%%%%%%%%%%%%%%%%%%%%%%%%%%%%%%%%%%%%%%%%%%%%%%%%%%%%%%%%%%%%%%%%%%%
% % 
% \begin{figure*}
% \centering
% \textbf{NS-BH Injection; NS-BH Templates}
% \includegraphics[width=1.7\columnwidth]{plots/TT_NSLambdaBias_vs_SNR_q23.pdf}
% \caption{This figure is similar to the bottom row of
% Fig.~\ref{fig:TNT_chirpMassBias_vs_SNR_q23}, with the difference that  we
% consider the NS tidal parameter $\Lambda$ here.}
% \label{fig:TT_NSLambdaBias_vs_SNR_q23}
% \end{figure*}
% % 
% \begin{figure*}
% \centering
% \textbf{NS-BH Injection; NS-BH Templates}
% \includegraphics[width=1.7\columnwidth]{plots/TT_NSLambdaCIWidth90_vs_SNR_q23.pdf}
% \caption{This figure is similar to the bottom row of
% Fig.~\ref{fig:TNT_chirpMassCIWidth90_vs_SNR_q23}, with the difference that  
% we consider the NS tidal parameter $\Lambda$ here.}
% \label{fig:TT_NSLambdaCIWidth90_vs_SNR_q23}
% \end{figure*}
% % 
% \begin{figure*}
% \centering    
% \includegraphics[width=2\columnwidth]{plots-TT/TT_LambdaBiases_CI90_0_Lambda_SNR.png}
% \caption{This figure shows.}
% \label{fig:TT_LambdaBiases_CI90_0_Lambda_SNR}
% \end{figure*}
% 
\begin{figure*}
\centering    
\includegraphics[width=2.1\columnwidth]{plots-TT/TT_LambdaCIWidths68_3_Lambda_SNR.png}\\
\includegraphics[width=2.1\columnwidth]{plots-TT/TT_LambdaCIWidths90_0_Lambda_SNR.png}
\caption{This figure shows.}
\label{fig:TT_LambdaCIWidths90_0_Lambda_SNR}
\end{figure*}
% 
\begin{figure*}
\centering    
\includegraphics[width=.65\columnwidth]{plots-TT/TT_SNRThresholdForLambdaMeasurement_BHspin_MassRatio_Lambda500_0_CI68_3.png}
\includegraphics[width=.65\columnwidth]{plots-TT/TT_SNRThresholdForLambdaMeasurement_BHspin_MassRatio_Lambda800_0_CI68_3.png}
\includegraphics[width=.65\columnwidth]{plots-TT/TT_SNRThresholdForLambdaMeasurement_BHspin_MassRatio_Lambda1000_0_CI68_3.png}\\
\includegraphics[width=.65\columnwidth]{plots-TT/TT_SNRThresholdForLambdaMeasurement_BHspin_MassRatio_Lambda500_0_CI90_0.png}
\includegraphics[width=.65\columnwidth]{plots-TT/TT_SNRThresholdForLambdaMeasurement_BHspin_MassRatio_Lambda800_0_CI90_0.png}
\includegraphics[width=.65\columnwidth]{plots-TT/TT_SNRThresholdForLambdaMeasurement_BHspin_MassRatio_Lambda1000_0_CI90_0.png}
\caption{This figure shows.}
\label{fig:TT_SNRThresholdForLambdaMeasurement_BHspin_MassRatio_CI90_0}
\end{figure*}
% 
\begin{figure*}
\centering    
\includegraphics[width=.65\columnwidth]{plots-TT/TT_LambdaThresholdForLambdaMeasurement_BHspin_BHmass_SNR30_0_CI68_3.png}
\includegraphics[width=.65\columnwidth]{plots-TT/TT_LambdaThresholdForLambdaMeasurement_BHspin_BHmass_SNR50_0_CI68_3.png}
\includegraphics[width=.65\columnwidth]{plots-TT/TT_LambdaThresholdForLambdaMeasurement_BHspin_BHmass_SNR90_0_CI68_3.png}\\
\includegraphics[width=.65\columnwidth]{plots-TT/TT_LambdaThresholdForLambdaMeasurement_BHspin_BHmass_SNR50_0_CI90_0.png}
\includegraphics[width=.65\columnwidth]{plots-TT/TT_LambdaThresholdForLambdaMeasurement_BHspin_BHmass_SNR90_0_CI90_0.png}
\caption{This figure shows.}
\label{fig:TT_LambdaThresholdForLambdaMeasurement_BHspin_BHmass}
\end{figure*}

Having shown in the previous section that we begin to care for NS tidal effects
for signals in the XXX corner of the paraemter space, with SNRs above YY, here
we investigate the effect of the improvement in the accuracy of the recovered
parameters when tidal effects are included in the templates.

\begin{enumerate}
\item Where in the parameter space can we actually make a statement about 
$\Lambda_\mathrm{NS}$ ?
\item What is the reduction in the bias of the maximum likelihood parameters
when using tidal templates?\newline
``Plot the ratio of the bias between N templates and T templates (against T
signals), as a function of SNR.''
\item What is the reduction in the uncertainty in binary mass and spin, if
any?\newline
``show how the the $90\%$ confidence intervals shrink, as a function of SNR, 
when we go from using N templates to T templates.''
\end{enumerate}


\subsection{Constraining the tidal deformability of the neutron star}\label{s2:measuring_lambda}

With the enhanced sensitivity of aLIGO, we are likely to be able to measure or
constrain deviations in the observed GW signals, especially close to binary
merger, from their expected form under the point-particle approximation.
The tidal disruption of the NS close to merger, and outside of the 
inner-most circular orbit, will drastically change the morphology of the emitted
GW signal, the measurement of which will allow us to constrain the equation 
of state (EoS) of nuclear matter.
The earlier in the orbit the NS disrupts, the sharper is the drop in the emitted
GW signal for frequencies above the disruption frequency. In this section we
study the ability of aLIGO to constrain the 
$\lambdans\sim \left(R/M\right)_\mathrm{NS}^5$ parameter, which
the combination that parametrizes the deviation in the emitted GW signal.
Past studies~\cite{}
have studied a limited sample of systems with distinguishability criterion that
are typically applicable in the very high signal-to-noise ratio cases. 
In what follows, we will use stochastic parameter estimation algorithms to show
aLIGO's capability of extracting EoS information out from GW signals with 
more realistic (likely) SNRs.

As before we focus on binaries with mass-ratios $q=\{[2,3,4,5\}$, with the NS mass
fixed at $1.35M_\odot$. We allow BH spin to take the values 
$\chibh=\{-0.5,0,+0.5,+0.75\}$, and $\lambdans=\{500, 800, 1000\}$. For each of these
parameter combinations, we inject a tidal-corrected waveform using the 
Lackey et al~\cite{} model , and use MCMC with ensemble sampling to construct the 
posterior probability distribution for different binary parameters with (a) BHBH
templates, and (b) NSBH templates. We will discuss the results of (b) here.


















%%%%%%%%%%%%%%%%%%%%%%%%%%%%%%%%%%%%%%%%%%%%%%%%%%%%%%%%%%%%%%%%%%%%%%%%%%%%%%%
\section{Discussion}
%%%%%%%%%%%%%%%%%%%%%%%%%%%%%%%%%%%%%%%%%%%%%%%%%%%%%%%%%%%%%%%%%%%%%%%%%%%%%%%
Discussion

%%%%%%%%%%%%%%%%%%%%%%%%%%%%%%%%%%%%%%%%%%%%%%%%%%%%%%%%%%%%%%%%%%%%%%%%%%%%%%%
% Acknowledgments
%%%%%%%%%%%%%%%%%%%%%%%%%%%%%%%%%%%%%%%%%%%%%%%%%%%%%%%%%%%%%%%%%%%%%%%%%%%%%%%
\begin{acknowledgments}
Acknowledgments
\end{acknowledgments}

%%%%%%%%%%%%%%%%%%%%%%%%%%%%%%%%%%%%%%%%%%%%%%%%%%%%%%%%%%%%%%%%%%%%%%%%%%%%%%%
\section*{References}
%%%%%%%%%%%%%%%%%%%%%%%%%%%%%%%%%%%%%%%%%%%%%%%%%%%%%%%%%%%%%%%%%%%%%%%%%%%%%%%
\bibliography{References/References}

\end{document}
