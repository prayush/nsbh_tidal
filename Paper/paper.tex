\documentclass[aps,prd,amsmath,floats,floatfix, twocolumn,
superscriptaddress,nofootinbib,showpacs]{revtex4-1}

\usepackage[colorlinks, pdfborder={0 0 0}, plainpages=false]{hyperref}
\usepackage{graphicx}
\usepackage{xspace}
\usepackage[usenames,dvipsnames]{color}
\usepackage{amssymb}
\usepackage[dvipsnames]{xcolor}
\usepackage{placeins}

\newcommand{\roughly}{\mathchar"5218\relax} % Different from \sim in spacing

% Macros for text changes
\newcommand{\red}{\textcolor{red}}
\newcommand{\dan}[1]{\textcolor{WildStrawberry}{#1}}
\newcommand{\Mark}[1]{\textcolor{Cerulean}{#1}}
\newcommand{\larry}[1]{\textcolor{OliveGreen}{#1}}
\newcommand{\bela}[1]{\textcolor{Blue}{#1}}
\newcommand{\saul}[1]{\textcolor{Orange}{#1}}
\newcommand{\prayush}{\textcolor{red!40!black}}

% Macros for text notes and comments
\newcommand{\Note}[1]{\textcolor{blue}{\textbf{[#1]}}}
\newcommand{\D}{\mathrm{d}}
\newcommand{\lambdans}{\Lambda_\mathrm{NS}}
\newcommand{\chibh}{\chi_\mathrm{BH}}
\newcommand{\chins}{\chi_\mathrm{NS}}
\newcommand{\mbh}{m_\mathrm{BH}}
\newcommand{\mns}{m_\mathrm{NS}}
\newcommand{\mchirp}{\mathcal{M}_c}
\newcommand{\LL}{\mathcal{L}}
\newcommand{\deff}{D_\mathrm{eff}}


\newcommand{\Caltech}{\affiliation{Theoretical Astrophysics 350-17,
    California Institute of Technology, Pasadena, CA 91125, USA}}
\newcommand{\Cornell}{\affiliation{Center for Radiophysics and Space
    Research, Cornell University, Ithaca, New York 14853, USA}}
\newcommand{\CITA}{\affiliation{Canadian Institute for Theoretical
    Astrophysics, 60 St.~George Street, University of Toronto,
    Toronto, ON M5S 3H8, Canada}} %
\newcommand{\GWPAC}{\affiliation{Gravitational Wave Physics and
    Astronomy Center, California State University Fullerton,
    Fullerton, California 92834, USA}} %


    
\newcommand{\NS}{\mathrm{NS}}
%%%%%%%%%%%%%%%%%%%%%%%%%%%%%%%%%%%%%%%%%%%%%%%%%%%%%%%%%%%%%%%

\begin{document}

\title{
Measuring matter effects in Neutron star - Black hole binaries with Advanced LIGO
}

\author{People across the Atlantic}

\date{\today}

\begin{abstract}
We attempt to answer the following questions: 
\begin{enumerate}
 \item What systematic biases are introduced by neglecting the NS tidal
 disruption close to merger? 
 \item At what SNR do the systematic biases dominate over the statistical 
 errors in the recovery of binary parameters?
 \item By using templates which models NS deformation (over inspiral) and 
 NS disruption (close to merger), what information can we recover about
 the $\Lambda_\mathrm{BH}\sim (R/M)^5_\mathrm{NS}$ parameter?
\end{enumerate}
\end{abstract}

\pacs{}
% 04.25.D- Numerical relativity
% 04.25.dg Numerical studies of black holes and black-hole binaries
% 04.25.Nx Post-Newtonian approximation; perturbation theory; related approximations 
% 04.30.-w Gravitational waves (see also 04.80.Nn Gravitational wave detectors and experiments)
% 04.30.Db Wave generation and sources 
% 02.70.Hm Spectral methods

\maketitle

%%%%%%%%%%%%%%%%%%%%%%%%%%%%%%%%%%%%%%%%%%%%%%%%%%%%%%%%%%%%%%%%%%%%%%%%%%%%%%%
\section{Introduction}
%%%%%%%%%%%%%%%%%%%%%%%%%%%%%%%%%%%%%%%%%%%%%%%%%%%%%%%%%%%%%%%%%%%%%%%%%%%%%%%

The Advanced LIGO (aLIGO) observatories began their first observing run ``O1''
mid-September 2015, operating at a factor of $3-4$ higher gravitational-wave (GW) 
strain sensitivity than their first-generation counterparts~\cite{Shoemaker2009}.
Coalescing binaries of
compact objects, i.e. Black Holes (BH) and/or Neutron Stars (NS), are the most
likely targets for GW detection with these ground-based instruments. Over a
$3$-month O1, these detectors would have access to as
many binary mergers as initial-LIGO detectors did over approximately $6$ years
of coincident operation. Both of these instruments will be further upgraded in
phases to reach their design sensitivity by $2018-19$, at which point we expect to
observe $\sim 70$ binary mergers a year~\cite{Abadie:2010cf}.

NS-BH binaries are one of the most important sources of gravitational waves (GW)
with these ground-based instruments, and we expect to detect $\mathcal{O}(10)$
of such systems every year~\cite{Abadie:2010cf}. 
%
One question that aLIGO will likely probe is `what is the nature of
matter at nuclear densities', i.e. what is the equation of state (EoS) of 
neutron stars. The EoS determines the compactness of NSs, which determines
their deformation over the course of inspiral (albeit weakly, at $5$PN order in 
phasing~\prayush{CITE PN TIDAL RESULTS}), as well as how early in the coalescence
process will the NS be disrupted by the tidal field of the BH~\prayush{[CITE]}.
Once disrupted,
the NS forms an accretion disk around the BH. Most of this matter disk spirals 
into the BH within a few milliseconds, with a small fraction becoming unbounded
and gets ejected~\prayush{CITE BHNS NR}. It is believed that an accompanying
short (lasting less than $2$ seconds) Gamma-ray Burst (sGRB) might be present
under certain physical conditions~\prayush{CITE NSBH-GRB},
making these systems one of the prime targets for joint electromagnetic-GW
astronomy searches~\prayush{CITE EM-GW}.
In such a situtation, the emitted GW signal
is heavily damped post-disruption, and the quasi-normal modes of the resulting
BH are not excited. Such a GW signal is qualitatively very different from a
BH-BH merger, and one of the main goals of this work is to determine the 
signal strength at which this difference will be quantifiabe by aLIGO.
%
However, studies have shown that for mass-ratio $\mbh/\mns\gtrsim 5$ the BH
will need to have fairly large and positively-aligned (with the orbit) spins
in order for its tidal field to be strong enough to disrupt the NS before the
innermost stable circular orbit (ISCO) is reached~\cite{Foucart:2013psa}.


Conventional astronomical methods have, to date, made observations of ~2500
NSs with masses between $1.25-2.1 M_\odot$~\cite{Lyne:2004cj,
Demorest:2010bx,2013Sci...340..448A,atnfcatalog,mcgillmagnetarcatalog,
stellarcollapsemass}, although
tightly clustered within $1.35\pm0.5M_\odot$~\cite{stellarcollapsemass}.
%
Observations of over 2500 NSs have shown that the spin of NSs
$|\vec{\chi}_\mathrm{NS}|$ have magnitudes that are $<1\%$~\cite{Miller:2014aaa}.
%
On the other hand, indirect observations of stellar-mass BHs place their
masses between $5-35M_\odot$, with their spin angular momenta 
$|\vec{\chi}_\mathrm{BH}|$ ranging from small to nearly extremal (Kerr) values
(see, e.g., Refs.~\cite{McClintockEtAl:2006,Miller:2009cw,Gou:2014una} for 
examples of nearly extremal estimates of BH spins, Refs.~\cite{McClintock:2013vwa,
Reynolds:2013qqa} for recent reviews of astrophysical BH spin measurements,
and Figure 5 of Ref.~\cite{Miller:2014aaa} for a comparison of NS and BH spins).



The difficulty in discerning the matter effects in an NSBH merger GW signal
lies in distinguishing it reliably from a BHBH merger signal. The earlier the
NS disrupts, the more orbits there are over which the effect of NS disruption
is visible. Where exactly does disruption happen is governed both by the 
mass-ratio of the binary, as well as the spin on the BH. Larger BHs (with
$q=m_\mathrm{BH}/m_\mathrm{NS}\gtrsim 7$), as well as BHs with large retrograde
spins ($\chi_\mathrm{BH} \lesssim 0$), tend to swallow the NS whole due to a
rapid merger (i.e. without it disrupting before the inner-most stable circular
orbit (ISCO) is reached). On the other hand, binaries with mass-ratios 
$q=2-5$ with large pro-grade $\chi_\mathrm{BH}$ can disrupt much before 
merger, leaving their imprint in the GW signal emitted close to $1.7-2$~kHz.
% 

The goal of this paper is to probe the distinguishability of BHBH and NSBH
mergers, focusing at the effects of the BH mass, BH spin, and the tidal 
deformability parameter $\Lambda_\NS\sim(R/M)^5_\NS$. We also investigate the
accuracy with which aLIGO can measure $\Lambda_\NS$ itself. For both of 
these studies, we will use the zero-detuning high-power noise curve for the
aLIGO detectors. \prayush{In addition, we also study the effect of tuning aLIGO
over a narrow frequency range aimed at NSBH mergers.}
We use the models published in Ref.~\cite{Lackey:2013axa,Pannarale:2015jka} to
simulate NSBH merger waveforms.



\vspace{2cm}

\textit{\prayush{Current status:}}

Below we explain the outline the different calculations performed in this 
paper:

\begin{enumerate}
\item We choose the following values of different NSBH binary parameters:
\begin{enumerate}
\item $m_\mathrm{NS}=1.35M_\odot$;
\item $q=m_\mathrm{BH}/\mathrm{NS}=\{2,3\}$;
\item $\chi_\mathrm{NS}=0$;
\item $\chi_\mathrm{BH}=\{-0.5, 0, +0.5\}$;
\item $\Lambda =\{500,1000,2000\}$ ($\Lambda=0\implies \mathrm{BHBH}$).
\end{enumerate}

\item For each system, corresponding to one of all possible unique combinations
of the above parameter choices, we carry out three parameter estimation tests: 
\begin{enumerate}
\item BH-BH signals injected, recovered with BH-BH waveforms (control),
\item NS-BH signals injected, recovered with BH-BH waveforms, (current approach) and
\item NS-BH signals injected, recovered with NS-BH waveforms.
\end{enumerate}
Injections were made at different SNR values: 
\begin{enumerate}
\item $\rho = \{20, 30, 50, 70, 90, 120\}$.
\end{enumerate}
Choice of emcee parameters:
\begin{enumerate}
\item $N_\mathrm{samples}=150,000$; 
\item $N_\mathrm{walkers}=100$; 
\item $N_\mathrm{burn-in}=500$; 
\end{enumerate}
and choice of prior boundaries:
\begin{enumerate} 
\item $m_\mathrm{BH}\in [1.2, 25]M_\odot$;
\item $m_\mathrm{NS}\in [1.2, 15]M_\odot$;
\item $\Lambda\leq 4000$; 
\item $\sigma\left(\Lambda\right) = 100$ (for chains where templates are NS-BH waveforms).
\end{enumerate}

\item Results are shown in
Fig.~\ref{fig:TNT_chirpMassBias_vs_SNR_q23}-~\ref{fig:TT_NSLambdaCIWidth90_vs_SNR_q23}.

\end{enumerate}
%%%%%%%%%%%%%%%%%%%%%%%%%%%%%%%%%%%%%%%%%%%%%%%%%%%%%%%%%%%%%%%%%%%%%%%%%%%%%%%
\section{Techniques}
%%%%%%%%%%%%%%%%%%%%%%%%%%%%%%%%%%%%%%%%%%%%%%%%%%%%%%%%%%%%%%%%%%%%%%%%%%%%%%%
\subsection{Modeling tidal deformation during inspiral}
Describe the PN terms used here

\subsection{Modeling tidal disruption near merger}
Describe the Lackey et al model

\subsection{Bayesian methods}
% % Describe the emcee-based PE code

The monte carlo algorithm samples the parameter space in proportion with 
the likelihood $\LL(\vec{\theta}) = p(d|\vec{\theta})$, with a 
uniform prior in parameters $p(\vec{\theta}) = \Theta(\vec{\theta}-\vec{\theta}_L) + \Theta(\vec{\theta}_H-\vec{\theta})$
\begin{equation}
 p(\vec{\theta}|d) = p(d|\vec{\theta}) p(\vec{\theta})
\end{equation}



%%%%%%%%%%%%%%%%%%%%%%%%%%%%%%%%%%%%%%%%%%%%%%%%%%%%%%%%%%%%%%%%%%%%%%%%%%%%%%%%
%\section{How are detection searches affected by not including NS matter effects}
%%%%%%%%%%%%%%%%%%%%%%%%%%%%%%%%%%%%%%%%%%%%%%%%%%%%%%%%%%%%%%%%%%%%%%%%%%%%%%%%
%
%Near future detection searches with aLIGO are poised to search for NSBH
%signals using BHBH templates. In this section, we examine the effect of ignoring
%the effects of tidal deformation and disruption of the NS during inspiral and 
%merger phases of search templates. We construct a template bank $\mathcal{B}$ 
%restricted to the NSBH parameter space with $\mbh\in[2.5M_\odot, 20.5M_\odot]$, 
%$\chibh\in[-0.5, +0.75]$, $\mns\in[1M_\odot,3M_\odot]$ and $\chi_\mathrm{NS}=0$,
%using a PN-based geometric lattice placement method~\cite{Harry:2013tca}. 
%
%
%Before investigating its SNR recovery against tidal NSBH signals, we test the 
%effectualness of this bank to point-particle NSBH signals,
%in order to make sure that the placement method does not leave any region of
%the binary parameter-space under-covered. We simulate $50,000$ signals with 
%component-mass and BH spin values
%sampled uniformly over their allowed ranges, and filter each of them against
%the entire bank. For both signals and templates, we use the frequency-domain 
%reduced-order model (ROM) of the SEOBNRv2 approximant~\cite{Purrer:2014} 
%implemented within the LIGO algorithms Library~\cite{lal}. We record the 
%fitting-factors~\cite{FittingFactorApostolatos} recovered by the bank for all
%signals, which measures the fraction of optimal SNR that the bank is able
%to recover. Fig.~\ref{} shows these FFs as a function of the injected
%BH mass and BH spin, and Fig.~\ref{} provides a histogram of the recovered FF 
%values. From these figures, we conclude that our bank is effectual as intended.
%
%
%Next, tn order to establish the effect of NS matter effects on the SNR recovery of 
%detection searches, we perform a similar calculation as above, with the
%difference that the signals are now modelled using the model of tidal 
%corrections to IMR amplitude and phasing published in 
%Ref.~\cite{Lackey:2013axa}, applied to SEOBNRv2 as the underlying BHBH model.

%%%%%%%%%%%%%%%%%%%%%%%%%%%%%%%%%%%%%%%%%%%%%%%%%%%%%%%%%%%%%%%%%%%%%%%%%%%%%%%
\section{How is PE affected by not including NS matter effects?}
%%%%%%%%%%%%%%%%%%%%%%%%%%%%%%%%%%%%%%%%%%%%%%%%%%%%%%%%%%%%%%%%%%%%%%%%%%%%%%%

% #################
\begin{figure*}
\centering 
%\includegraphics[trim={11cm 0 0 0},width=2.7\columnwidth]{plots/TNMchirpBiasesOverCIWidths_CI68_3_Lambda_SNR.pdf}\\
% \includegraphics[trim={3.6cm 0 0 0},width=2.25\columnwidth]{plots/TNMchirpBiasesOverCIWidths_CI90_0_Lambda_SNR}
\includegraphics[trim={1.3cm 0 0 0},width=2.\columnwidth]{plots/TNMchirpBiasesOverCIWidths_CI90_0_Lambda_SNR30_70_linear}
\caption{We show here the ratio of systematic and statistical
measurement uncertainties for the binary chirp mass over the NSBH parameter 
space. Each panel shows the same as a function of BH mass and spin. Across
each row, we can see the effect of increasing the signal strength (i.e. SNR)
with the NS's deformability fixed. Down each column, we can see 
the effect of the increasing NS's tidal deformability at fixed SNR. We also
show dashed contours for where the ratio equals $10\%$ (labelled ``-1'')
or $100\%$ (labelled ``0'').
% 
Generally, these two sources of uncertainty are different and for BBHs, the
statistical errors dominate systematic ones~\cite{Kumar:2016dhh}. We find
that for BHNS binaries, its not much different until we get to high SNRs
$\rho\gtrsim 70$ for which if the BH is spinning rapidly enough, the 
measurement uncertainty for $\mchirp$ could become dominated by the exclusion
of tidal effects in filtering templates.
}
\label{fig:TN_chirpMassBias_vs_Lambda_SNR}
\end{figure*}
%
\begin{figure*}[!t]
\centering    
%\includegraphics[trim={11cm 0 0 0},width=2.7\columnwidth]{plots/TNEtaBiasesOverCIWidths_CI68_3_Lambda_SNR.pdf}\\
% \includegraphics[trim={3.6cm 0 0 0},width=2.25\columnwidth]{plots/TNEtaBiasesOverCIWidths_CI90_0_Lambda_SNR.pdf}\\
\includegraphics[trim={1.3cm 0 0 0},width=2.\columnwidth]{plots/TNEtaBiasesOverCIWidths_CI90_0_Lambda_SNR30_70_linear002}
\caption{This figure is similar to Fig.~\ref{fig:TN_chirpMassBias_vs_Lambda_SNR}
with the difference that the quantity in question here is the symmetric 
mass-ratio $\eta$. We find that for fairly loud GW signals, with $\rho\simeq 50$,
not including the effects of NS's tidal deformation on GW emission can become the dominant
source of error for astrophysical searches with Advanced LIGO. However,
for quieter signals with $\rho\leq 30$, it will have minimal effects on the measurement
of $\eta$. We remind the reader that the SNRs here are all single detector values.
}
\label{fig:TN_EtaBias_vs_Lambda_SNR}
\end{figure*}
% 
% 
\begin{figure*}
\centering
\includegraphics[trim={1.3cm 0 0 0},width=2.\columnwidth]{plots/TNChiBHBiasesOverCIWidths_CI90_0_Lambda_SNR_linear}
% \includegraphics[width=\columnwidth]{plots/TNChiBHBiasesOverCIWidthsVsSNR_All_CI68_3.pdf}
% \includegraphics[width=\columnwidth]{plots/TNChiBHBiasesOverCIWidthsVsSNR_All_CI90_0.pdf}
\caption{This figure shows the ratio of the systematic and statistical
measurement errors for BH spins, with other attributed identical to 
Fig.~\ref{fig:TN_chirpMassBias_vs_Lambda_SNR}, and~\ref{fig:TN_EtaBias_vs_Lambda_SNR}.
Similar to the case of mass parameters, we find that below
$\rho\approx 30$, ignoring tidal effects in templates introduces minor systematic effects,
which remain subdominant to the statistical measurement uncertainties.
}
\label{fig:TNT_BHspinBias_vs_SNR_q23}
\end{figure*}
% 
% 
Advanced LIGO searches and parameter estimation efforts aimed at binaries 
containing NS and stellar-mass BHs are poised to use waveform models that do not
include the effects of the NS tidal deformability~\cite{Canton:2014ena}.
While this is not expected to
be the dominant source of error at low SNRs, it will likely introduce a
systematic bias in the physical parameters that we recover from GW observations.
In this section, we study the impact of the same for binaries where the BH spins
are aligned to the orbital angular momentum.

\begin{enumerate}
\item Above what SNR values, below what mass-ratios, above what BH spins, etc, 
do we begin to care about NS deformability?\newline
``one interesting plot to make would be the parameter bias (max-Likelihood
parameters minus true parameters) vs SNR, as functionals of q/sBH/Lambda, when
using T signals recovered with N templates.''\newline
\end{enumerate}





\hspace{5mm}






%%%%%%%%%%%%%%%%%%%%%%%%%%%%%%%%%%%%%%%%%%%%%%%%%%%%%%%%%%%%%%%%%%%%%%%%%%%%%%%
\section{What do we gain by using templates that include NS matter effects?}
%%%%%%%%%%%%%%%%%%%%%%%%%%%%%%%%%%%%%%%%%%%%%%%%%%%%%%%%%%%%%%%%%%%%%%%%%%%%%%%
%\begin{figure*}
%\centering    
%\includegraphics[width=.65\columnwidth]{plots-TT/TT_SNRThresholdForLambdaMeasurement_BHspin_MassRatio_Lambda500_0_CI68_3.png}
%\includegraphics[width=.65\columnwidth]{plots-TT/TT_SNRThresholdForLambdaMeasurement_BHspin_MassRatio_Lambda800_0_CI68_3.png}
%\includegraphics[width=.65\columnwidth]{plots-TT/TT_SNRThresholdForLambdaMeasurement_BHspin_MassRatio_Lambda1000_0_CI68_3.png}\\
%\includegraphics[width=.65\columnwidth]{plots-TT/TT_SNRThresholdForLambdaMeasurement_BHspin_MassRatio_Lambda500_0_CI90_0.png}
%\includegraphics[width=.65\columnwidth]{plots-TT/TT_SNRThresholdForLambdaMeasurement_BHspin_MassRatio_Lambda800_0_CI90_0.png}
%\includegraphics[width=.65\columnwidth]{plots-TT/TT_SNRThresholdForLambdaMeasurement_BHspin_MassRatio_Lambda1000_0_CI90_0.png}
%\caption{This figure shows.}
%\label{fig:TT_SNRThresholdForLambdaMeasurement_BHspin_MassRatio_CI90_0}
%\end{figure*}
% 
\begin{figure*}
\centering    
\includegraphics[trim={3cm 0 0 0},width=2.3\columnwidth]{plots/TTLambdaCIWidths90_0_Lambda_SNR.pdf}
%\includegraphics[trim={3cm 0 0 0},width=2.3\columnwidth]{plots/TTLambdaCIWidths90_0_Lambda_SNR.pdf}
\caption{This figure shows the statistical uncertainty in the measurement of
$\lambdans$, as a percentage of the injected signal's $\lambdans$ value. In each panel,
the same is shown as a function of the BH mass and spin, keeping the 
$\lambdans$ and injection's SNR $\rho$ fixed (given in the panel). Each row contains
panels with the same value of $\lambdans$, with $\rho$ increasing from left to right.
Each column contains panels with the same value of $\rho$, with $\lambdans$ 
increasing from top to bottom.
% 
Contours are drawn in each panel demarkating regions where we can constrain the
$\lambdans$ parameter well (within a factor of two of the injected value).
% 
We note that, as expected, the measurement accuracy for $\lambdans$ improves with (i) increasing
SNR, (ii) decreasing BH mass, (iii) increasing BH spin, and 
(iv) increasing $\lambdans$, i.e. the tidal deformability of the neutron star.
}
\label{fig:TT_LambdaCIWidths90_0_Lambda_SNR}
\end{figure*}
% 
\begin{figure*}
\centering    
% \includegraphics[width=\columnwidth]{plots/TTLambdaErrorCurves_BHspin_BHmass_SNR20_CI90_0.pdf}
\includegraphics[width=1.025\columnwidth]{plots/TTSNRThresholdFor100LambdaMeasurement_BHspin_BHmass_Lambda1500_0_CI90_0}
\includegraphics[width=1.025\columnwidth]{plots/TTSNRThresholdFor100LambdaMeasurement_BHspin_BHmass_Lambda2000_0_CI90_0}\\
% \includegraphics[width=.68\columnwidth]{plots/TTLambdaErrorCurves_BHspin_BHmass_SNR70_CI90_0.pdf}
% \includegraphics[width=.68\columnwidth]{plots/TTLambdaErrorCurves_BHspin_BHmass_SNR90_CI90_0.pdf}
% \includegraphics[width=.68\columnwidth]{plots/TTLambdaErrorCurves_BHspin_BHmass_SNR120_CI90_0.pdf}
\caption{In these figures we highlight the regions of BH mass and spin plane where we can 
constrain the $\lambdans$ of its companion NS within a factor of $1-2$ of its true value.
Panels correspond to different signal strength, with $\rho$ increasing from left to right.
Within each panel, $\lambdans$ measurement uncertainty contours are shown as functions of BH 
mass and spin, with line-types corresponding to different uncertainty levels, and
colors showing the injected $\lambdans$.
% We note that this information is, in principle, contained in 
% Fig.~\ref{fig:TT_LambdaCIWidths90_0_Lambda_SNR}, which we gather here to better understand
% the effect of NS's deformability itself on its measurement. 
As hinted at in Fig.~\ref{fig:TT_LambdaCIWidths90_0_Lambda_SNR}, we note that the measurability
of NS matter
effects improves with all factors that enhances the signature of the NS's disruption on
the \textit{detectable} portion of the emitted GW signal (implying, within a frequency band
set by the detectors).
}
\label{fig:TT_LambdaErrorCurves_BHspin_BHmass_CI90_0}
\end{figure*}
%
\begin{figure*}
\centering    
% \includegraphics[width=\columnwidth]{plots/TTLambdaErrorCurves_BHspin_BHmass_SNR20_CI90_0.pdf}
\includegraphics[width=1.025\columnwidth]{plots/TTLambdaErrorCurves_BHspin_BHmass_SNR30_CI90_0.pdf}
\includegraphics[width=1.025\columnwidth]{plots/TTLambdaErrorCurves_BHspin_BHmass_SNR50_CI90_0.pdf}\\
% \includegraphics[width=.68\columnwidth]{plots/TTLambdaErrorCurves_BHspin_BHmass_SNR70_CI90_0.pdf}
% \includegraphics[width=.68\columnwidth]{plots/TTLambdaErrorCurves_BHspin_BHmass_SNR90_CI90_0.pdf}
% \includegraphics[width=.68\columnwidth]{plots/TTLambdaErrorCurves_BHspin_BHmass_SNR120_CI90_0.pdf}
\caption{In these figures we highlight the regions of BH mass and spin plane where we can 
constrain the $\lambdans$ of its companion NS within a factor of $1-2$ of its true value.
Panels correspond to different signal strength, with $\rho$ increasing from left to right.
Within each panel, $\lambdans$ measurement uncertainty contours are shown as functions of BH 
mass and spin, with line-types corresponding to different uncertainty levels, and
colors showing the injected $\lambdans$.
% We note that this information is, in principle, contained in 
% Fig.~\ref{fig:TT_LambdaCIWidths90_0_Lambda_SNR}, which we gather here to better understand
% the effect of NS's deformability itself on its measurement. 
As hinted at in Fig.~\ref{fig:TT_LambdaCIWidths90_0_Lambda_SNR}, we note that the measurability
of NS matter
effects improves with all factors that enhances the signature of the NS's disruption on
the \textit{detectable} portion of the emitted GW signal (implying, within a frequency band
set by the detectors).
}
\label{fig:TT_LambdaErrorCurves_BHspin_BHmass_CI90_0}
\end{figure*}
%
% \begin{figure*}
% \centering    
% \includegraphics[trim={0 0 0 0},width=\columnwidth]{plots/TT_q2-00_mNS1-35_chiBH0-75_Lambda1500-0_SNR20-0_corner_eta_Mc_chi2_Lambda.pdf}
% \includegraphics[trim={0 0 0 0},width=\columnwidth]{plots/TT_q2-00_mNS1-35_chiBH0-75_Lambda1500-0_SNR20-0_corner_m1_m2_chi2_Lambda.pdf}
% \caption{These figure shows the $1-$ and $2-$dimensional 
% \textit{a posteriori} probability distributions 
% for different physical parameters describing a BHNS system including NS matter 
% effects, as measured by a Bayesian Markov-chain Monte Carlo using tidal templates.
% The true parameters of the injected signal are: 
% $\mathcal{M}=1.6406M_\odot, \eta=0.2222, \chibh=\chi_2=0.75$ and $\lambdans=1500$,
% with the injected SNR $\rho=20$. We show this case as typical of those with the largest
% tidal signatures imprinted on the emitted GWs (due to the system being comparable
% mass-ratio with a large-aligned spin on the BH and a fairly deformable NS), yet 
% with low SNR.
% % 
% We note a few things:
% (i) chirp mass is well constrained, with a nearly-Gaussian probability distribution;
% while (ii) the other $3$ less constrained parameters rail against
% prior boundaries. The specific boudary conditions that we run into are (a) for $\eta$,
% we are limited by the requirement that $m_\mathrm{NS}\geq 1.2M_\odot$ 
% of our templates; (b) for BH spin, we are limited by the $\chibh\leq+0.75$ requirement
% that results from a template model limitation~\cite{Lackey:2013axa}; and (c) for 
% $\lambdans$, we
% restrict it to $[0, 4000]$ for the same reason as (b). These prior boundaries
% limit the exploration of the entire likelihood surfaces. While (a) is astrophysically
% justified, given that we have not observed NSs with masses below 
% $1.25M_\odot$, (b) and (c) will be removed in future studies as more evolved template
% models become available. See, also, Fig.~\ref{fig:TT_logLikeProb_q2_chi75_L1500_SNR20}.
% % 
% Finally, we note that the left panel corresponds to the actual sampling parameters,
% while the right panel to derived ones.
% }
% \label{fig:TT_pdf_q2_chi75_L1500_SNR20}
% \end{figure*}
% %
% \begin{figure*}
% \centering    
% % \includegraphics[width=.66\columnwidth]{plots/TT_q2-00_mNS1-35_chiBH0-75_Lambda1500-0_SNR20-0_corner_eta_Mc_chi2_Lambda.pdf}
% \includegraphics[width=\columnwidth]{plots/TT_logLike_vs_m1_m2_eta0-22_Mc1-64_chi0-00_chi0-50_L1000-0_SNR30.pdf}
% \includegraphics[width=\columnwidth]{plots/TT_logProb_vs_m1_m2_eta0-22_Mc1-64_chi0-00_chi0-50_L1000-0_SNR30.pdf}
% \caption{These figures show the effect of limitation (a) (described in the caption
% of Fig.~\ref{fig:TT_pdf_q2_chi75_L1500_SNR20}) on the likelihood surface portion
% that is actually accessible to our Bayesian inferencing algorithm. On the left we show the
% entire likelihood surface in the $\mbh-\mns$ plane, while on the left is the 
% portion not forbidden by the requirement $\mns\geq 1.2M_\odot$. Similarly, our
% priors on BH spin and NS deformability further curtail the portion of sampled
% parameter space, albeit to much less impact than this mass cut.}
% \label{fig:TT_logLikeProb_q2_chi75_L1500_SNR20}
% \end{figure*}


%
%% 
%\begin{figure*}
%\centering    
%\includegraphics[width=.65\columnwidth]{plots-TT/TT_LambdaThresholdForLambdaMeasurement_BHspin_BHmass_SNR30_0_CI68_3.png}
%\includegraphics[width=.65\columnwidth]{plots-TT/TT_LambdaThresholdForLambdaMeasurement_BHspin_BHmass_SNR50_0_CI68_3.png}
%\includegraphics[width=.65\columnwidth]{plots-TT/TT_LambdaThresholdForLambdaMeasurement_BHspin_BHmass_SNR90_0_CI68_3.png}\\
%\includegraphics[width=.65\columnwidth]{plots-TT/TT_LambdaThresholdForLambdaMeasurement_BHspin_BHmass_SNR50_0_CI90_0.png}
%\includegraphics[width=.65\columnwidth]{plots-TT/TT_LambdaThresholdForLambdaMeasurement_BHspin_BHmass_SNR90_0_CI90_0.png}
%\caption{This figure shows.}
%\label{fig:TT_LambdaThresholdForLambdaMeasurement_BHspin_BHmass}
%\end{figure*}

Having shown in the previous section that we begin to care for NS tidal effects
for signals in the XXX corner of the paraemter space, with SNRs above YY, here
we investigate the effect of the improvement in the accuracy of the recovered
parameters when tidal effects are included in the templates.

\begin{enumerate}
\item Where in the parameter space can we actually make a statement about 
$\Lambda_\mathrm{NS}$ ?
\item What is the reduction in the bias of the maximum likelihood parameters
when using tidal templates?\newline
``Plot the ratio of the bias between N templates and T templates (against T
signals), as a function of SNR.''
\item What is the reduction in the uncertainty in binary mass and spin, if
any?\newline
``show how the the $90\%$ confidence intervals shrink, as a function of SNR, 
when we go from using N templates to T templates.''
\end{enumerate}


\subsection{Constraining the tidal deformability of the neutron star}\label{s2:measuring_lambda}

With the enhanced sensitivity of aLIGO, we are likely to be able to measure or
constrain deviations in the observed GW signals, especially close to binary
merger, from their expected form under the point-particle approximation.
The tidal disruption of the NS close to merger, and outside of the 
inner-most circular orbit, will drastically change the morphology of the emitted
GW signal, the measurement of which will allow us to constrain the equation 
of state (EoS) of nuclear matter.
The earlier in the orbit the NS disrupts, the sharper is the drop in the emitted
GW signal for frequencies above the disruption frequency. In this section we
study the ability of aLIGO to constrain the 
$\lambdans\sim \left(R/M\right)_\mathrm{NS}^5$ parameter, which
the combination that parametrizes the deviation in the emitted GW signal.
Past studies~\cite{}
have studied a limited sample of systems with distinguishability criterion that
are typically applicable in the very high signal-to-noise ratio cases. 
In what follows, we will use stochastic parameter estimation algorithms to show
aLIGO's capability of extracting EoS information out from GW signals with 
more realistic (likely) SNRs.

As before we focus on binaries with mass-ratios $q=\{[2,3,4,5\}$, with the NS mass
fixed at $1.35M_\odot$. We allow BH spin to take the values 
$\chibh=\{-0.5,0,+0.5,+0.75\}$, and $\lambdans=\{500, 800, 1000\}$. For each of these
parameter combinations, we inject a tidal-corrected waveform using the 
Lackey et al~\cite{} model , and use MCMC with ensemble sampling to construct the 
posterior probability distribution for different binary parameters with (a) BHBH
templates, and (b) NSBH templates. We will discuss the results of (b) here.




\pagebreak
\newpage
\FloatBarrier
\section{Combining observations: looking forward with Advanced LIGO}\label{s1:multiple_observations}
% 
% 
\begin{figure*}
\centering    
\includegraphics[width=1.\columnwidth]{plots/pdfLambda_vs_N_L800.pdf}
\includegraphics[width=1.\columnwidth]{plots/FillBetweenNormErrorBarsLambda_vs_N_L800.pdf}
\caption{{\it Left}:  Posterior probability distributions for $\lambdans$
and associated $90\%$ confidence intervals, shown as a function of number of events 
accumulated $N$.
{\it Right}: Filled-region plots showing the median and $90\%$ confidence intervals
for $\lambdans$, as a function of the number of observed events $N$.
 Shown are two independent filled regions. 
%  
 One in green shows the recovered confidence interval
 for $\lambdans$ (on the left y-axis), normalized by its true value,
 as a function of the 
 number of observed events $N$. The recovered median is shown by the
 green line-circled curve. The pair of horizontal green dashed (dotted)
 lines show the $\pm 50\%$ ($\pm 25\%$) symmetric error bounds around
 the true $\lambdans$ value, which is shown at y-value $=1$ by definition.
% 
 The second filled region, in grey, shows the recovered
 confidence interval for neutron star compactness, without normalizing.
 The corresponding y-values are shown on the *right* y-axis. The horizontal grey
 dashed line traces the true value of $\lambdans$, and is what we expect
 the filled region to approach with increasing number of observations.
%  
 Finally, the grey dotted lines are contours of $1/\sqrt{N}$, drawn to aid the eye.
}
\label{fig:TT_Lambda_vs_N_L800_CI90_0}
\end{figure*}
%
% 
\begin{figure*}
\centering    
\includegraphics[width=1.\columnwidth]{plots/FillBetweenNormErrorBarsLambda_vs_N_L2000.pdf}
\includegraphics[width=1.\columnwidth]{plots/FillBetweenNormErrorBarsLambda_vs_N_L1500.pdf}\\
\includegraphics[width=1.\columnwidth]{plots/FillBetweenNormErrorBarsLambda_vs_N_L1000.pdf}
\includegraphics[width=1.\columnwidth]{plots/FillBetweenNormErrorBarsLambda_vs_N_L500.pdf}
\caption{Filled-region plots showing the median and $90\%$ confidence intervals
for $\lambdans$ measurement, as a function of the number of observed events $N$.
Different panels correspond to different injected $\lambdans$ values, which is shown
in their title. In each panel are shown two independent filled regions. 
%  
 One in green shows the recovered confidence interval
 for $\lambdans$ (on the left y-axis), normalized by its true value,
 as a function of the 
 number of observed events $N$. The recovered median is shown by the
 green line-circled curve. The pair of horizontal green dashed (dotted)
 lines show the $\pm 50\%$ ($\pm 25\%$) symmetric error bounds around
 the true $\lambdans$ value, which is shown at y-value $=1$ by definition.
% 
 The second filled region, in grey, shows the recovered
 confidence interval for neutron star compactness, without normalizing.
 The corresponding y-values are shown on the *right* y-axis. The horizontal grey
 dashed line traces the true value of $\lambdans$, and is what we expect
 the filled region to approach with increasing number of observations.
%  
 Finally, the grey dotted lines are contours of $1/\sqrt{N}$, drawn to aid the eye.
}
\label{fig:TT_Lambda_vs_N_L500_2000_CI90_0}
\end{figure*}
%
% 
% \begin{figure}
% \centering    
% \includegraphics[width=1.\columnwidth]{plots/FillBetweenRelErrorBarsLambda_vs_NShifted_AllLambda.pdf}
% \caption{HELLO
% }
% \label{fig:TT_Lambda_vs_N_L500_2000_CI90_0_AllInOne}
% \end{figure}
%
% 
\begin{figure*}
\centering    
\includegraphics[width=1.\columnwidth]{plots/LambdaCIWidths_vs_N.pdf}
\includegraphics[width=1.\columnwidth]{plots/RelErrorLambdaMedian_vs_N.pdf}
\caption{{\it Left}: In the main panel is shown the width of the measured $90\%$ 
confidence interval for $\lambdans$, normalized by its true value,
 as a function of the number of events observed $N$. In the inset, we
 show the same measurement uncertainty, but not normalized, and on a 
 logarithmic axis for $N$ as well.
% 
 From both the inset and the main plot, we find that for the first $3-4$ events
 our measurement is prior limited. As $N$ increases, the uncertainty falls 
 roughly as $1/\sqrt{N}$ with intermittent loud signals causing more rapid 
 (than $1/\sqrt{N}$) narrowing of the confidence interval.
%  
 {\it Right}: This figure shows the fractional difference between the median
 recovered value from the posterior probability distribution for $\lambdans$
 as a function of $N$. Different curves correspond to different injected values
 of $\lambdans$, and are shaded from light to dark in direct proportion. 
 The horizontal dashed brown line marks $10\%$ deviation of the median from the 
 true value.
%  
We find that within $\approx 10$ detections, the median measured value for
$\lambdans$ would estimate the true value for the star with the remarkable 
accuracy of $10\%$.
}
\label{fig:TT_LambdaError_vs_N_L500_2000_CI90_0}
\end{figure*}
%

% 
\begin{figure}
\centering    
\includegraphics[width=1.\columnwidth]{plots/LambdalErrorBars_vs_Lambda_N49_Log.pdf}
\caption{This figure shows the relative uncertainty in the measurement of $\lambdans$,
as a function of $\lambdans$ itself. The drawn curves are cumulative, in the sense 
that the $N=n$ curve includes information from all of the first $n$ observed events.
Those corresponding to the first $N\leq 5$ events are shown in fading shades of red,
with the ones for higher value of $N$ being shown in grey (going from light to dark in
proportion to $N$).
Green dash-dotted lines are contours of $y=x^{-(1+1/5)}$.
% 
\prayush{We note that the measurement uncertainty from a population follows the
equation for the green lines.}
% 
}
\label{fig:TT_Lambda_vs_Lambda_L500_2000_CI90_0_AllInOne}
\end{figure}
%



% 
\subsection{Multiple identical sources at low SNR}\label{s2:identical_multiple}
% 
% 
Here we show a crude calculation which tells us that approximately $8$ (or $16$) observations
of NSBH signals with $\rho\geq 10$ distributed uniformly in effective volume, would
give us similarly accurate a measurement of $\lambdans$ as would a single detection
with $\rho=50$ (or $\rho=70$). 


We want to distribute signals in effective distance $\deff$, which is a combination of 
luminosity distance, inclination angle of the binary's orbital angular momentum
with respect to the detector, and its various sky angles. The signal strength 
scales inversely with $\deff$, if it comprises only of the dominant $(2,2)$ mode, allowing
for various angle-dependent factors to scale out beside the luminosity distance.
% 
In order to consider multiple events, lets imagine a population uniformly distributed
in effective volume, i.e. within a sphere of radius $\deff^\mathrm{max}$. The 
radius of this sphere is set by the lowest SNR that is distinguishable from noise
by LIGO searches. Now, dividie the sphere into $I$ shells of equal width. The radius
of the $i$-th sphere would then be $D_i = \deff^\mathrm{max} (i - 1/2)/I$. The 
measurement uncertainty in $\lambdans$ scales inversely with the SNR, and hence
directly with the effective distance to the source. Therefore, if we have a 
measurement error $\sigma_0$ for a source located at $\deff = D_0$, the same error
for the same source located within the $i$-th shell would be 
$$
\sigma_i = \sigma_0 \frac{D_i}{D_0}.
$$
Independent measurements of $\lambdans$ for identical sources at different distances
would have their combined error given by
\begin{equation}\label{eq:1oversigma}
\frac{1}{\sigma^2} = \sum_{i=1}^I \frac{N_i}{\sigma_i^2} = \left(\frac{D_0}{\sigma_0}\right)^2 \sum_{i=1}^I\frac{N_i}{D_i^2},
\end{equation}
where $N_i$ is the number of sources detected in the $i$-th shell, and is a random
variable, with its probability proportional to the volume of the shell, i.e.
$$
p(N_i) \propto \frac{V_i}{V_\mathrm{total}} \propto \dfrac{\left(\deff^\mathrm{max} \frac{i}{I}\right)^3 - \left(\deff^\mathrm{max} \frac{i-1}{I}\right)^3}{(\deff^\mathrm{max})^3} \propto \frac{1}{I^3} [i^3 - (i-1)^3].
$$
Therefore, the expected number of detections in the $i$-th shell would be
$$
\langle N_i\rangle = \frac{N}{I^3} [i^3 - (i-1)^3],
$$
where $N=\sum_{i=1}^I N_i$ is the total number of NSBH detections. $N$ is expected
to be Poisson distributed around the mean detection rate
$\mathcal{R}\equiv\langle N\rangle$, where $\mathcal{R}$ can vary from $0.6-1000$ per
$\mathrm{Gpc}^3$ per year~\cite{Abadie:2010cfa}. If there are $n$ resulting detections
a year per unit effective volume, then
\begin{equation}
\langle N\rangle = \int_0^{\deff^\mathrm{max}} 4\pi n D^2 \D D = \frac{4\pi}{3} n (\deff^\mathrm{max})^3,
\end{equation}
and
\begin{eqnarray}
 \langle \frac{1}{\sigma^2}\rangle &=& \left(\frac{D_0}{\sigma_0}\right)^2 \int_0^{\deff^\mathrm{max}} \frac{4\pi n D^2 }{D^2}\D D\\
 &=& \left(\frac{D_0}{\sigma_0}\right)^2 4\pi n \deff^\mathrm{max},
\end{eqnarray}
where in the previous equation we have converted the summation in Eq.~\ref{eq:1oversigma} 
to an integral. So the root-mean-square averaged measurement error from $N$ sources 
distributed uniformly in effective volume within a sphere of radius $\deff^\mathrm{max}$
would be
\begin{equation}\label{eq:rmsSigmaIdenticalSources}
 \sigma_{avg} \equiv \frac{1}{\sqrt{\langle1/\sigma^{2}\rangle}} = \frac{\sigma_0}{D_0} \deff^\mathrm{max} \frac{1}{\sqrt{3\langle N\rangle}}.
\end{equation}
Let us analyze the implications of Eq.~\ref{eq:rmsSigmaIdenticalSources}. It is
straightforward to deduce that the same measurement certainty as afforded by a single
observation with a high SNR $\alpha$ can be obtained from $N = \alpha^2/300$ observations
uniformly distributed in effective volume with SNRs $\geq 10$. E.g., to get to the level 
of certainty afforded at SNR$=70$, we would need $49/3\approx 16-17$ realistic detections.

There are two main caveats to this rudimentary calculation, which was outlined in 
Ref.~\cite{Markakis:2010mp}, (i) it applies to identical sources,
which is astrophysically next to impossible to achieve exactly, but could perhaps happen 
approximately; (ii) the average measurement error $\sigma_{avg}$ has been defined as in 
the first equality in Eq.~\ref{eq:rmsSigmaIdenticalSources}. 

% 
\subsection{Astrophysical source population}\label{s2:astro_multiple}
% 
Lets try a respectable Bayesian approach. Imagine we have $N$ stretches of data,
$d_1, d_2, \cdots, d_N$, each containing a single observation of an NSBH binary. Let $K$
denote our collective prior knowledge, except for expectations on binary parameters
themselves, which will enter the following calculation explicitly. Let the gravitational wave
signal from each binary be characterized by the parameters 
$\vec{\theta} = \{\mbh, \mns, \chibh, \chins, \rho\}$ and $\lambdans$. Note that we have folded
in the luminosity distance, inclination angle of the binary's orbital angular momentum with
respect to the plane of the detector, and its sky location angles into a single parameter $\rho$,
which is the SNR of the received signal. Note also that $\vec{\theta}$ does not include
$\lambdans$ as the latter would be treated independently. In addition, we assume
that all observed NS's have the same equation of state, that uniquely determines their
$\lambdans$ parameter. In other words, we assume that we know its mass perfectly.
% 
Using Bayes' theorem, the measured probability distribution function for the $\lambdans$
parameter is given as
\begin{equation}\label{eq:lambdaMultiple}
 p(\lambdans | d_1, d_2, \cdots, d_N; K) = p(\lambdans | K)^{\alpha}\prod_{i=1}^N p(\lambdans | d_i, K),
\end{equation}
where $p(\lambdans|K)$ is the prior expectation for $\lambdans$, and we have assumed
that all of the $N$ observations are independent. [Note: $\alpha$ will be calculated 
soon, it should be a function only of N.]. A priori, we will assume that no particular value
of $\lambdans$ is more preferred over another, and $\lambdans\leq 4000$, ie 
$$
p(\lambdans | K) = \mathrm{Rect}\left(\frac{\lambdans-2000}{4000}\right)
$$
% 
The second set of terms $p(\lambdans | d_i, K)$ are the inferred posterior distribution
functions for $\lambdans$ from \textit{each} observation independently, and are 
calculated by marginalizing over all other binary parameters, i.e.
\begin{equation}
 p(\lambdans | d_n, K) = \int \D\, \vec{\theta} p(\vec{\theta}, \lambdans | d_n, K).
\end{equation}
The joint inferred probability distribution of all parameters $\vec{\theta}\cup\{\lambdans\}$ 
is given by
\begin{equation}
 p(\vec{\theta}, \lambdans | d_n, K) = \dfrac{p(d_n|\vec{\theta}, \lambdans, K)\,p(\vec{\theta}, \lambdans | K)}{p(d_n|K)}.
\end{equation}
Here, $p(\vec{\theta}, \lambdans | K)$ is the prior probability for binary parameters of taking 
particular values. In this paper, we use a uniform prior on the individual masses within 
pre-chosen ranges, as well as a uniform prior on the spin of the black hole. We set neutron star's
spin identically $=0$, and its tidal deformability parameter $\lambdans$ to be equally likely
between $[0, 4000]$. The prior probability of obtaining a particular realization of data
$p(d_n|K)$ is absorbed into the overall normalization. Finally, the first term in the numerator
$p(d_n|\vec{\theta}, \lambdans, K)$ is the likelihood of obtaining the given stretch of 
data $d_n$ if we assume that a signal parameterized by $\vec{\theta}\cup\{\lambdans\}$ is buried
in it, i.e.
\begin{equation}\label{eq:likelihood}
 \LL(\vec{\theta}, \lambdans) \equiv p(\vec{\theta}, \lambdans | K) = \mathcal{N} \mathrm{exp}[- \langle d_n - h | d_n - h\rangle ],
\end{equation}
where $h\equiv h(\vec{\theta}, \lambdans)$ is our waveform model, and the detector noise
weighted inner-product $\langle a|b\rangle$ is defined as 
$$
\langle a|b\rangle \equiv 4\,\mathrm{Re}\left[\int_0^\infty \dfrac{\tilde{a}(f) \tilde(b)(f)^*}{S_n(|f|)}\,\D f\right],
$$
where $\tilde{a}$ means the Fourier transform of the finite time series $a(t)$, and 
$S_n(|f|)$ is the one-sided amplitude spectrum of detector noise. The definition of $\LL$ in 
Eq.~\ref{eq:likelihood} assumes that the instrument noise is colored Gaussian. 


\prayush{%
As described in the previous sections, we have measured the $p(\lambdans | d_i, K)$ for a 
set of different binary parameters. Therefore, the problem of finding the cumulative measurement
constraint on $\lambdans$ using Eq.~\ref{eq:lambdaMultiple} reduces to being able to choose
an astrophysically motivated sample set of $N$ binaries over which the second term on the RHS 
can be computed.}

\prayush{%
Get posterior measurements for low SNRs, i.e. $\rho=\{10, 15, 20, 25, 30\}$. 
Fix black hole mass and spins. Compute the measurement certainty for $\lambdans$
as a function of the number of detections for a population astrophysically distributed 
in effective distance. Repeat for a few different combinations of black hole masses
and spins. This would be a sub-set of the overall detected population for NSBH binaries,
but what fraction of it would this comprise of is something that we can only probe with
LIGO itself. We do not know the real distribution of low-mass black holes in the field.
}



%%%%%%%%%%%%%%%%%%%%%%%%%%%%%%%%%%%%%%%%%%%%%%%%%%%%%%%%%%%%%%%%%%%%%%%%%%%%%%%
\section{Discussion}
%%%%%%%%%%%%%%%%%%%%%%%%%%%%%%%%%%%%%%%%%%%%%%%%%%%%%%%%%%%%%%%%%%%%%%%%%%%%%%%
Discussion

%%%%%%%%%%%%%%%%%%%%%%%%%%%%%%%%%%%%%%%%%%%%%%%%%%%%%%%%%%%%%%%%%%%%%%%%%%%%%%%
% Acknowledgments
%%%%%%%%%%%%%%%%%%%%%%%%%%%%%%%%%%%%%%%%%%%%%%%%%%%%%%%%%%%%%%%%%%%%%%%%%%%%%%%
\begin{acknowledgments}
Acknowledgments
\end{acknowledgments}

%%%%%%%%%%%%%%%%%%%%%%%%%%%%%%%%%%%%%%%%%%%%%%%%%%%%%%%%%%%%%%%%%%%%%%%%%%%%%%%
\section*{References}
%%%%%%%%%%%%%%%%%%%%%%%%%%%%%%%%%%%%%%%%%%%%%%%%%%%%%%%%%%%%%%%%%%%%%%%%%%%%%%%
\bibliography{References/References}

\end{document}
