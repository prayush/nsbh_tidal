\documentclass[aps,prd,amsmath,floats,floatfix, twocolumn,
superscriptaddress,nofootinbib,showpacs]{revtex4-1}

\usepackage[colorlinks, pdfborder={0 0 0}, plainpages=false]{hyperref}
\usepackage{graphicx}
\usepackage{xspace}
\usepackage[usenames,dvipsnames]{color}
\usepackage{amssymb}
\usepackage[dvipsnames]{xcolor}
\usepackage{placeins}

\newcommand{\roughly}{\mathchar"5218\relax} % Different from \sim in spacing

% Macros for text changes
\newcommand{\red}{\textcolor{red}}
\newcommand{\dan}[1]{\textcolor{WildStrawberry}{#1}}
\newcommand{\Mark}[1]{\textcolor{Cerulean}{#1}}
\newcommand{\larry}[1]{\textcolor{OliveGreen}{#1}}
\newcommand{\bela}[1]{\textcolor{Blue}{#1}}
\newcommand{\saul}[1]{\textcolor{Orange}{#1}}
\newcommand{\prayush}{\textcolor{red!40!black}}

% Macros for text notes and comments
\newcommand{\Note}[1]{\textcolor{blue}{\textbf{[#1]}}}
\newcommand{\D}{\mathrm{d}}
\newcommand{\lambdans}{\Lambda_\mathrm{NS}}
\newcommand{\dlambda}{\delta\Lambda}

\newcommand{\ii}{\mathrm{i}}
\newcommand{\chibh}{\chi_\mathrm{BH}}
\newcommand{\chins}{\chi_\mathrm{NS}}
\newcommand{\mbh}{m_\mathrm{BH}}
\newcommand{\mns}{m_\mathrm{NS}}
\newcommand{\mchirp}{\mathcal{M}_c}
\newcommand{\LL}{\mathcal{L}}
\newcommand{\deff}{D_\mathrm{eff}}


\newcommand{\Caltech}{\affiliation{Theoretical Astrophysics 350-17,
    California Institute of Technology, Pasadena, CA 91125, USA}}
\newcommand{\Cornell}{\affiliation{Center for Radiophysics and Space
    Research, Cornell University, Ithaca, New York 14853, USA}}
\newcommand{\CITA}{\affiliation{Canadian Institute for Theoretical
    Astrophysics, 60 St.~George Street, University of Toronto,
    Toronto, ON M5S 3H8, Canada}} %
\newcommand{\GWPAC}{\affiliation{Gravitational Wave Physics and
    Astronomy Center, California State University Fullerton,
    Fullerton, California 92834, USA}} %
\newcommand{\AEI}{\affiliation{Albert Einstein Institute,
Am M\"uhlenberg, G\"olm, Germany}} %



    
\newcommand{\NS}{\mathrm{NS}}
%%%%%%%%%%%%%%%%%%%%%%%%%%%%%%%%%%%%%%%%%%%%%%%%%%%%%%%%%%%%%%%

\begin{document}

\title{
Measuring matter effects in Neutron star - Black hole binaries with Advanced LIGO: a Bayesian study
}

\author{Prayush Kumar}\CITA\email{prkumar@cita.utoronto.ca}
\author{Michael P\"urrer}\AEI
\author{People across the Atlantic}

\date{\today}

\begin{abstract}
Very recently, the Advanced LIGO (aLIGO) observatories announced the first
terrestrial detection of gravitational waves. This heralds an era of 
observational gravitational wave (GW) astronomy. Neutron star - black hole
(NSBH) binaries are one of the primary sources targeted for observation
with the aLIGO detectors.
% 
GWs from these sources will carry signatures of the tidal distortion of
the NS by the BH during the inspiral, and of its tidal disruption (if that
happens) at merger. In the NS is disrupted, the merger and ringdown signal
is drastically suppressed, which distinguishes it from a BHBH merger.
% 
In this paper, we present the first Bayesian study of the measurability of
the tidal deformability of neutron stars, allowing the companion hole to take 
on a variety of masses and spins.
% 
First, we find that if BHBH templates are used to estimate source 
parameters for an NSBH signal, a significant bias is introduced in the 
inference if the signal-to-noise ratio (SNR) is greater than $\rho\simeq30$.
% 
Second, we find that if we are to get a loud signal with $\rho\geq 23$, 
we can put a factor of two bound on the NS tidal deformability parameter
$\lambdans\propto (R/M)^{5}$.
% 
Finally, we study the improvement in our measurement of $\lambdans$ from 
multiple observations. Using an approximately astrophysical population, 
we find that (i) the median measured $\lambdans$ catches up to within
$10\%$ of the true value in approximately $20$ observations; (ii) the $90\%$
confidence interval for $\lambdans$ shrinks to $\pm 50\%$ around the true 
value with the same number of observations.
% 
We find these results encouraging, and recommend that an effort to 
measure $\lambdans$ be taken within the LIGO-Virgo Collaboration once
we begin to detect NSBH coalescences.
\end{abstract}

\pacs{}
% 04.25.D- Numerical relativity
% 04.25.dg Numerical studies of black holes and black-hole binaries
% 04.25.Nx Post-Newtonian approximation; perturbation theory; related approximations 
% 04.30.-w Gravitational waves (see also 04.80.Nn Gravitational wave detectors and experiments)
% 04.30.Db Wave generation and sources 
% 02.70.Hm Spectral methods

\maketitle

%%%%%%%%%%%%%%%%%%%%%%%%%%%%%%%%%%%%%%%%%%%%%%%%%%%%%%%%%%%%%%%%%%%%%%%%%%%%%%%
\section{Introduction}\label{s1:introduction}
%%%%%%%%%%%%%%%%%%%%%%%%%%%%%%%%%%%%%%%%%%%%%%%%%%%%%%%%%%%%%%%%%%%%%%%%%%%%%%%

The Advanced LIGO (aLIGO) observatories began their first observing run ``O1''
mid-September 2015, operating at a factor of $3-4$ higher gravitational-wave
(GW) strain sensitivity than their first-generation 
counterparts~\cite{Shoemaker2009}.
% 
During O1, the first terrestrial detection of gravitational waves was made, as
was announced earlier in 2016~\cite{LIGOVirgo2016a}. Emitted more than a
billion year ago, these waves herald the era of observational gravitational
wave (GW) astronomy as they pass through Earth.
% 
Over the $3$-months of O1, aLIGO had access to more binary mergers than 
initial-LIGO over its entire coincident operation period. Both of aLIGO 
instruments will be further upgraded in phases to reach their design 
sensitivity by $2018-19$, at which point we may expect to observe $\sim 70$ 
binary mergers a year~\cite{Abadie:2010cf}.


Coalescing binaries of compact objects, i.e. black holes (BH) and/or neutron 
stars (NS), remain the most interestings targets for GW based observation with
aLIGO. NSBH binaries form an important sub-class of compact binaries, and we
expect to observe $\mathcal{O}(10)$ of these every year with design 
sensitivity aLIGO~\cite{Abadie:2010cf}.
% 
NSBH mergers are of interest for various reasons. For instance, they have 
been associated with (as the progenitors of) short Gamma-ray Bursts
(GRBs)~\cite{eichler:89,1992ApJ...395L..83N,moch:93,Barthelmy:2005bx,
2005Natur.437..845F,2005Natur.437..851G,Shibata:2005mz,Paschalidis2014,
Tanvir:2013}. Under favorable conditions, close to merger the neutron star
can get disrupted by the tidal field of its companion hole. Once disrupted,
the neutron star material may form an accretion disk around the black hole. 
Most of the disk matter spirals in within a few milliseconds, with a small 
fraction becoming unbounded and getting ejected, with a possibled simultaneous
burst of gamma-ray emission)~\cite{Foucart2015,Lovelace:2013vma,Deaton2013,
Foucart2012,Shibata:2005mz,Paschalidis2014}. A coincident detection of GWs
from an NSBH system with a GRB would therefore provide us insights into the
central engine powering GRBs. 
% 
Another question that NSBH mergers can help answer is `what is the 
nature of matter at nuclear densities', i.e. what is the equation of state
(EoS) of neutron stars. This is what we will focus on in this paper.
% 
The gravitational-waves emitted by NSBH binaries carry subtle hints of the 
matter structure of the neutron star. During early inspiral, the (weak) tidal
field of the companion hole deforms the star, thereby exciting its resonant
oscillation modes. The induced quadrupolar moment of the star depends on the
tidal field $\mathcal{E}_{ij}$ due to its companion through the relation
$Q_{ij} = -\lambda\mathcal{E}_{ij}$, where $\lambda$ is the EoS dependent
tidal deformability parameter related to the neutron star's dimensionless
Love number $k_2$ and radius $R_\mathrm{NS}$ as
$\lambda:=\frac{2}{3G}k_2 R_\mathrm{NS}^5$. This drains energy from the orbit
resulting in an increase of the inspiral rate. We know the leading and next-to
-leading order terms in post-Newtonian (PN) theory that capture this 
effect~\cite{Vines2011}, entering binary phasing at $5$PN order. Close to
merger, as the NS gets disrupted and forms long tails that feed into the 
companion black hole, the quadrupole moment of the system falls monotonically.
As a result, the emitted GWs are subdued significantly post-disruption
(as compared to what they would be for a BBH),
and the quasi-normal modes of the resulting black hole are not 
excited~\cite{Kyutoku:2010zd,Lackey:2013axa,Lovelace:2013vma,Foucart2015,
Pannarale:2015jia}.
% 
% 
The difficulty in discerning NS matter effects in an NSBH coalescence from its
GW signal lies in distinguishing it reliably from a BBH signal, since the 
difference between the two is either subtle (during inspiral) or short-lived
(disruption and merger). The earlier in inspiral the NS disrupts, the more
orbits there are over which its effect is visible. The mechanism of disruption
is governed by the nature of the BH, i.e. its mass and spin. More massive BHs 
(with $m_\mathrm{BH}\gtrsim 7\, m_\mathrm{NS}$), as well as BHs with
(large) retrograde spins $(\chi_\mathrm{BH}\in[-1. 0])$, tend to swallow the 
NS whole due to a rapid merger (i.e. without it disrupting before the
inner-most stable circular orbit (ISCO) is reached). The quasi-normal
oscillation modes of the final black hole's horizon are also 
excited~\cite{Foucart:2013psa}. Therefore, the only available information 
comes from the inspiral rate modification. On the other hand, less massive
black holes spinning in the prograde sense (with mass-ratio 
$q\in[2, 5]$, and $\chi_\mathrm{BH}\geq 0$) can cause their companion star to 
disrupt much before merger, leaving a strong imprint on the emitted GW 
signal~\cite{Shibata:2007zm,2010PhRvD..81f4026F,Lovelace:2013vma,
Kawaguchi:2015}.



A large fraction of past work aimed at measuring NS matter effects from GW
signals has consisted of inquiries about binary neutron stars (BNS)~\cite{
Lee1999a,Lee1999b,Lee2000,oechslin:07,Read:2008iy,Markakis:2010mp,Markakis:2011vd,
stergioulas:11,East:2011xa,Lackey2014,Wade:2014vqa,Bauswein:2014qla}. While single 
observations are not as informative, Bayesian studies have shown that with
$20-30$ BNS observations, aLIGO would begin to distinguish between hard,
moderate and soft candidate equations of state~\cite{DelPozzo:13,
Chatziioannou:2015uea,Agathos:2015a}. These studies rely on the PN 
description of BNS dynamics, which is reasonable since BNSs merge at very high
frequencies and most of their signal power resides in inspiral.
% 
On the other hand, NSBHs merge at relatively lower frequencies than BNS, 
and their disruption during late-inspiral occurs at frequencies where aLIGO
has relatively more resolving power. This, combined with a shorter inspiral,
means that the merger portion of NSBH signals is relatively more valuable for
measuring $\lambdans$. Some past NSBH studies aimed at measuring tidal effects,
however, have used PN inspiral-only waveforms~\cite{Maselli:2013rza}. In doing
so, however, (i) they ignore the merger signal which could contain significant
information for NSBHs, and (ii) the errors due to unknown vaccum terms in PN 
waveforms could easily dominate over the tidal terms 
themselves~\cite{Yagi:2014}, since there are unknown point-particle terms at 
orders {\it lower} than that at which tidal effects enter. Other studies that
use complete numerical simulations or derived models either use Fisher matrix 
to estimate $\lambdans$ measurement error~\cite{Lackey2011,Lackey:2013axa},
which is not reliable at realistic signal-to-noise ratios 
(SNR)~\cite{Vallisneri:2007ev}, or are limited in the binary parameter space
they sample~\cite{Foucart:2013psa}.


In this paper we study the measurability of neutron star's tidal deformability
from GWs emitted by {\it disruptive} NSBH mergers using aLIGO. We also probe how tidal 
effects affect the estimation of other binary parameters for the same class
of systems. This study improves upon previous work in many ways. First, we
include tidal effects during inspiral and merger in a consistent way, by using
the waveform model of Ref.~\cite{Lackey:2013axa} (henceforth ``LEA+''). Second,
we include the effect
of black hole spin on tidal GW signals, in addition to the effect of BH mass 
and NS's deformability itself. Third, we perform a fully Bayesian 
analysis, instead of using a Fisher matrix approximation. Fourth, we explore 
how our measurement errors decrease as we gain information from multiple 
(realistic) events.


We first probe the effect of not including tidal effects in parameter 
estimation templates on the recovery of non-tidal binary parameters, such as 
component masses and spins. This is the case of current and planned aLIGO
efforts.
To do so, we use the LEA+ model to generate a set of realistic signals, and
use that as data from which we estimate binary mass and spin parameters with
a Bayesian Monte-Carlo algorithm, using non-tidal (BBH) waveforms as filters.
% 
As in rest of the paper, we restrict here our parameter space to span 
mass-ratios $q:=\mbh/\mns\in[2,5]$, black hole spin $\chibh\in[-0.5, +0.75]$, 
and dimensionless neutron star tidal deformability 
$\lambdans:= G\left(\frac{c^2}{G \mns}\right)^5\lambda \in[500, 2000]$.
The mass of NS for signals (and not templates) is fixed to $1.35M_\odot$ 
throughout~\cite{stellarcollapsemass}, and is taken to be 
non-spinning~\cite{Miller:2014aaa}.
Each signal is rescaled to a range of SNR values $\rho\in[10, 70]$.
% 
We find that non-inclusion of tidal effects leads to systematic biases in
the measurement of non-tidal parameters. Consider first the binary chirp 
mass $\mchirp:=M\eta^{3/5}$ where $M$ is the binary total mass, and 
$\eta:=\mbh\mns/M^2$ is the dimensionless mass-ratio. We find that for
realistic (moderate) SNRs, i.e. $\rho\lesssim 30$, not including tidal
effects introduces a barely measurable bias that remains below $50\%$
of the statistical uncertainty of the $\mchirp$ measurement itself.
For higher signal strengths, we find that the underlying statistical
uncertainty decreases enough for the systematic bias to become comparable
to it for $\rho=50-70$ and higher. This is expected because chirp mass is fairly
accurately determined by the inspiral, where tidal effects are weak.
Next, we consider the mass-ratio $\eta$. We find that only if the NS is quite
deformable (with $\lambdans=2000$), {\it and} has a companion BH whose mass
is $\mbh\lesssim 4.5M_\odot$ (i.e. in the astrophysical ``mass-gap''~\cite{
Bailyn:1997xt,Kalogera:1996ci,Kreidberg:2012,Littenberg:2015tpa}),
will the systematic bias in $\eta$ measurement become comparable to its
statistical uncertainties at SNRs $\rho\leq 30$. For SNRs $\rho\gtrsim30$,
however, not including tidal effects is very likely to {\it significantly} bias
the mass-ratio estimate for NSBH binaries. Finally, we consider the spin on the
black hole $\chibh$. We find that $\chibh$ measurements are affected in a similar
way to $\eta$. By not including tidal effects, we compromise our measurement of
$\chibh$ significantly only if signal $\rho\gtrsim 30$. 
%
Overall, we learn an important thing. Bayesian parameter estimation algorithms
aimed at identifying NSBH binaries with low latency in order to alert 
electromagnetic (EM) observers in time to establish a possible coincident 
observation of a GRB~\cite{2012A&A...541A.155A,Singer:2014qca,Singer:2015ema,
Pankow:2015cra} can use BBH templates to look for NSBH systems with aLIGO.
This is so because the primary requirement of the EM follow-up effort is rapid
classification of a binary as an NSBH system, which can be achieved just as 
easily with BBH templates, purely on the basis of the smaller component's mass.
Another conclusion we draw is that detection searches are unaffected by the
choice of ignoring tidal effects in matched-filtering templates.


Next, we study the ability of aLIGO to constrain neutron star tidal 
deformability with a single observation of coalescing NSBHs. For this, we
use the same setup and parameter choices for signal waveforms as before, but
replace the template model with one that includes tidal effects from inspiral
through to merger~\cite{Lackey:2013axa} (i.e. LEA+). First of all, this reduces
the biases in the measurement of non-tidal parameters. Second, as we can see
from Fig.~\ref{fig:TT_LambdaCIWidths90_0_Lambda_SNR}, at signal-to-noise
ratios $\simeq 20-30$, only under the most favorable circumstance (including a
BH with mass within the ``mass-gap'') can we put a factor of $1.5-2$ bound on
$\lambdans$ at $90\%$ confidence level. For similar signals with 
$\rho\gtrsim 30$ we can constrain $\lambdans$ within $\pm 50\%$ of its true
value. For more realistic SNRs and BH parameters, a single measurement is 
for $\lambdans$ is accompanied with a $\sim\pm150+\%$. uncertainty. While these
results are expected, multiple realistic observations is the next window that 
we explore in order to obtain more information on $\lambdans$. These results 
are detailed in Sec.~\ref{s1:PEwithNS}.


Multiple realistic observations deliver far more information than what a single
observation furnishes, as has been shown the case to be for binary neutron
stars~\cite{DelPozzo:13}. As our final study, we perform a fully-Bayesian 
analysis of how our estimation of $\lambdans$ changes as we accumulate more and
more detections. The choices that determine the population are important here,
and were made as follows. Neutron star mass is fixed for signals at
$1.35M_\odot$, and its spin $\chins=0$. Black hole mass is sampled uniformly
from the range $[2,5]\times 1.35=[2.7, 6.75]M_\odot$. Our choice here is given
by the intersection set of the mass range that allows for neutron star disruption
and the range supported by the waveform model LEA+~\cite{Foucart2012,
Foucart:2013a,Lackey:2013axa}. Black hole spin is sampled uniformly from the
{\it aligned} range, i.e. $\chibh\in[0,+1]$. The true deformability is fixed
once per population, sampled uniformly from the range $\lambdans\in[0,2000]$.
In order to keep the computational cost tractable, we make a further 
approximation. We inject signals into zero noise at different SNRs and recover
their parameters from Bayesian inferencing for a fixed set of binaries,
whose parameters lie on a uniform grid given by:
$q=\{2,3,4,5\}\times\chibh=\{-0.5,0,0.5,0.75\}\times\lambdans=\{500,800,1000,1500,2000\}\times\rho=\{10,20,30,50,70\}$.
Once the population parameters been selected, each event's parameters are
replaced by their nearest neighbor in the above set. We make sure that this 
approximation is conservative by keeping our population sampling ranges within the
conservative side of the range spanned by the above set. With the caveats
laid out, we report our results in Sec.~\ref{s1:multiple_observations}.
% 
We find \prayush{[...]}



Finally, we note that our results are applicable for the design sensitivity
LIGO instruments, that are expected to come online in $2018-19$. Also note
that more a recent work that improves upon the waveform model we use in this
study has been published~\cite{Pannarale:2015jka}, but it provides only an 
amplitude model which needs to be (in future) augmented with a compatible
phase model. In addition, accuracy of the underlying numerical simulations 
used to calibrate the waveform model used here have not been verified against 
independent codes so far.
It is therefore difficult to assess the combined modeling error and its effect
on our results. they are, therefore, limited by the limitations of our
waveform model. However, we expect the combined effect of modeling errors to
not change our qualitative conclusions.




The remainder of the paper is organized as follows. 
Sec.~\ref{s1:techniques} discusses data analysis techniques and resources 
used in this paper, such as the waveform model, and parameter estimation 
algorithm.
Sec.~\ref{s1:PEwithnoNS} discusses the consequences of ignoring tidal 
effects in parameter estimation waveform models.
Sec.~\ref{s1:PEwithNS} discusses the measurability for the leading order
tidal parameter $\lambdans$ at plausible SNR values.
Sec.~\ref{s1:multiple_observations} discusses the improvement in our
measurement of $\lambdans$ with successive (multiple) observations of
NSBH mergers.
Finally, in Sec.~\ref{s1:discussion} we summarize our results and discuss
future prospects with Advanced LIGO.





%%%%%%%%%%%%%%%%%%%%%%%%%%%%%%%%%%%%%%%%%%%%%%%%%%%%%%%%%%%%%%%%%%%%%%%%%%%%%%%
\section{Techniques}\label{s1:techniques}
%%%%%%%%%%%%%%%%%%%%%%%%%%%%%%%%%%%%%%%%%%%%%%%%%%%%%%%%%%%%%%%%%%%%%%%%%%%%%%%

In this section we summarize various technical aspects of this paper. The model
used to obtain tidal waveforms is described in Sec.~\ref{s2:waveforms}. The 
process of inferring source parameters from a GW signal is summarized in 
Sec.~\ref{s2:bayesian}. The process of generating and combining multiple events
is instead detailed in the corresponding results section,
Sec.~\ref{s2:astro_multiple}.

\subsection{Waveform Models}\label{s2:waveforms}

In Ref.~\cite{Lackey:2013axa}, the authors presented a suite of $134$ numerical
relativity simulations of neutron stars inspiraling into spinning black holes. Amongst
the varied quantities were (a) the equation of state for the NS, (b) NS mass, 
(c) mass-ratio and (d) black hole spins, taken as aligned with the orbit.
A total of $21$ two-parameter EoSs were used for these simulations, which also 
had $q\in[2, 5]$, $\chibh\in[-0.5, +0.75]$, and 
$m_\mathrm{NS}\in[1.20, 1.45]M_\odot$. Based on this suite of simulations, the
authors calibrated a frequency-domain waveform model for the inspiral and merger phasing of
NSBH coalescences. The GW strain predicted by this model (henceforth ``LEA''),
can be written as
% 
\begin{equation}
 \tilde{h}_\mathrm{NSBH}(f, \vec{\theta}, \lambdans) = \tilde{h}_\mathrm{BBH}(f, \vec{\theta})\,A(f, \vec{\theta}, \lambdans)\,e^{\ii \Delta\Phi(f, \vec{\theta}, \lambdans)},
\end{equation}
where $\vec{\theta}:=\{\mbh, \mns, \chibh,\chins=0\}$ are the intrinsic binary
parameters. The first term on the right hand side is an underlying
BBH waveform model that is used as a base. In the original construction of LEA,
this was taken to be the SEOBNRv1 model~\cite{Taracchini:2012} of the 
Effective-one-body (EOB) family~\cite{Buonanno99}. The second term changes the 
amplitude of the BBH model
to match that of an NSBH merger of otherwise identical parameters, with NS-matter
effects parametrized by $\lambdans$. During inspiral this term is set to
unity, but is a sensitive function of $\lambdans$ close to merger. The third 
term does the same as the second, but to the waveform phasing. During inspiral,
$\Delta\Phi(f, \vec{\theta}, \lambdans)$ is set to the PN tidal phasing corrections,
at the leading and next-to-leading orders~\cite{Vines2011}. Close to merger,
additional phenomenological terms are added and calibrated to the $134$ available
NR simulations.

In this paper we use LEA for our signal and template modeling, but switch the 
underlying BBH model to SEOBNRv2~\cite{Taracchini:2013rva},
using its reduced-order frequency-domain version~\cite{Purrer:2015tud}. We
expect this change to make our conclusions more robust because: (a) the 
SEOBNRv2 model is more accurate~\cite{Kumar:2015tha,Kumar:2016dhh}, and (b)
the differences between the two EOB models are caused by the
inaccuracies of SEOBNRv1 during the {\it inspiral} phase, many orbits before 
merger~\cite{Kumar:2015tha}.
Since LEA only augments the inspiral phasing with PN tidal terms, our
change in the underlying BBH model does not change LEA's construction, but
increases the overall model accuracy during inspiral. Therefore the model used 
here (``LEA+'') is a more robust version of LEA.




\subsection{Bayesian methods}\label{s2:bayesian}
% % Describe the emcee-based PE code

We perform Bayesian parameter estimation for tidal signals, using
both tidal and non-tidal templates. The signal parameters are selected 
to lie on a grid given by:
$q=\{2,3,4,5\}\times\chibh=\{-0.5,0,0.5,0.75\}\times\lambdans=\{500,800,1000,1500,2000\}\times\rho=\{10,20,30,50,70\}$.
Corresponding signal waveforms are generated using the model described in the 
previous sub-section, and each injected into a separate zero noise data segment
$d_n$. Templates can be either tidal or non-tidal (LEA+ with $\lambdans=0$).

Inferred from $d_n$, the joint probability distribution for binary parameters
$\vec{\Theta}:=\vec{\theta}\cup\{\lambdans\}$ is given by
\begin{equation}\label{eq:postprob}
 p(\vec{\Theta} | d_n, K) = \dfrac{p(d_n|\vec{\Theta}, K)\,p(\vec{\Theta} | K)}{p(d_n|K)}.
\end{equation}
Here, $p(\vec{\Theta} | K)$ is the prior probability for binary parameters of
taking particular values. In this paper, we use a uniform prior on the
individual masses within pre-chosen ranges, as well as a uniform prior on the
spin of the black hole. We set neutron star's spin $=0$, and its tidal
deformability parameter $\lambdans$ is sampled uniformly from $[0, 4000]$. In
addition, we exclude mass-ratios between $[1,2]$ from our prior. This is done
because of model constraints for LEA+. The prior probability of obtaining a 
particular realization of data $p(d_n|K)$ is absorbed into the overall 
normalization. Finally, the first term in the numerator
$p(d_n|\vec{\theta}, \lambdans, K)$ is the likelihood of obtaining the given
stretch of data $d_n$ if we assume that a signal parameterized by
$\vec{\theta}\cup\{\lambdans\}$ is buried in it, i.e.
\begin{equation}\label{eq:likelihood}
 \LL(\vec{\Theta}) \equiv p(d_n| \vec{\Theta}, K) = \mathcal{N} \mathrm{exp}[- \langle d_n - h | d_n - h\rangle ],
\end{equation}
where $h\equiv h(\vec{\Theta})$ is a template given by our waveform model LEA+,
and the detector noise weighted inner-product $\langle a|b\rangle$ is defined 
as
\begin{equation}
\langle a|b\rangle \equiv 4\,\mathrm{Re}\left[\int_0^\infty \dfrac{\tilde{a}(f) \tilde(b)(f)^*}{S_n(|f|)}\,\D f\right],
\end{equation}
% 
where $\tilde{a}(f)$ is the Fourier transform of the finite time series $a(t)$,
and $S_n(|f|)$ is the one-sided amplitude spectrum of detector noise. The 
definition of $\LL$ in Eq.~\ref{eq:likelihood} assumes that the instrument
noise is colored Gaussian. The marginalized probability distribution for any
single binary parameter (say $\alpha$) can be obtained by integrating
Eq.~\ref{eq:postprob} over all other parameters, i.e. 
\begin{equation}
 p(\alpha | d_n, K) = \int\D \vec{\Theta}_\alpha\, p(\vec{\Theta} | d_n, K),
\end{equation}
where $\vec{\Theta}_\alpha$ is the set of remaining parameters, i.e.
$\vec{\Theta}_\alpha:=\vec{\Theta} - \{\alpha\}$.



We sample the probability distribution of Eq.~\ref{eq:postprob} using
an ensemble sampler Markov-chain Monte-Carlo algorithm based on the {\tt emcee}
package~. A total of $100$ independent walkers were used, each of which
was allowed to collect $100,000$ samples. We measure the correlation time and keep
only $20,000$ independent posterior points to further analyze the calculated 
posterior probability distributions. One simplifying assumption made to mitigate 
computational cost was setting the frequency sampling rate $\Delta f=0.4$~Hz, which
we found to be sufficient to calculate likelihoods robustly~\cite{Purrer:2015nkh}.
\prayush{TODO: MICHAEL}.




%%%%%%%%%%%%%%%%%%%%%%%%%%%%%%%%%%%%%%%%%%%%%%%%%%%%%%%%%%%%%%%%%%%%%%%%%%%%%%%
\section{How is PE affected by not including NS matter effects?}\label{s1:PEwithnoNS}
%%%%%%%%%%%%%%%%%%%%%%%%%%%%%%%%%%%%%%%%%%%%%%%%%%%%%%%%%%%%%%%%%%%%%%%%%%%%%%%

% #################
\begin{figure*}
\centering 
%\includegraphics[trim={11cm 0 0 0},width=2.7\columnwidth]{plots/TNMchirpBiasesOverCIWidths_CI68_3_Lambda_SNR.pdf}\\
% \includegraphics[trim={3.6cm 0 0 0},width=2.25\columnwidth]{plots/TNMchirpBiasesOverCIWidths_CI90_0_Lambda_SNR}
\includegraphics[trim={1.3cm 0 0 0},width=2.\columnwidth]{plots/TNMchirpBiasesOverCIWidths_CI90_0_Lambda_SNR30_70_linear}
\caption{We show here the ratio of systematic and statistical
measurement uncertainties for the binary chirp mass over the NSBH parameter 
space. Each panel shows the same as a function of BH mass and spin. Across
each row, we can see the effect of increasing the signal strength (i.e. SNR)
with the NS's deformability fixed. Down each column, we can see 
the effect of the increasing NS's tidal deformability at fixed SNR. We also
show dashed contours for where the ratio equals $10\%$ (labelled ``-1'')
or $100\%$ (labelled ``0'').
% 
Generally, these two sources of uncertainty are different and for BBHs, the
statistical errors dominate systematic ones~\cite{Kumar:2016dhh}. We find
that for BHNS binaries, its not much different until we get to high SNRs
$\rho\gtrsim 70$ for which if the BH is spinning rapidly enough, the 
measurement uncertainty for $\mchirp$ could become dominated by the exclusion
of tidal effects in filtering templates.
}
\label{fig:TN_chirpMassBias_vs_Lambda_SNR}
\end{figure*}
%
\begin{figure*}[!t]
\centering    
%\includegraphics[trim={11cm 0 0 0},width=2.7\columnwidth]{plots/TNEtaBiasesOverCIWidths_CI68_3_Lambda_SNR.pdf}\\
% \includegraphics[trim={3.6cm 0 0 0},width=2.25\columnwidth]{plots/TNEtaBiasesOverCIWidths_CI90_0_Lambda_SNR.pdf}\\
\includegraphics[trim={1.3cm 0 0 0},width=2.\columnwidth]{plots/TNEtaBiasesOverCIWidths_CI90_0_Lambda_SNR30_70_linear002}
\caption{This figure is similar to Fig.~\ref{fig:TN_chirpMassBias_vs_Lambda_SNR}
with the difference that the quantity in question here is the symmetric 
mass-ratio $\eta$. We find that for fairly loud GW signals, with $\rho\simeq 50$,
not including the effects of NS's tidal deformation on GW emission can become the dominant
source of error for astrophysical searches with Advanced LIGO. However,
for quieter signals with $\rho\leq 30$, it will have minimal effects on the measurement
of $\eta$. We remind the reader that the SNRs here are all single detector values.
}
\label{fig:TN_EtaBias_vs_Lambda_SNR}
\end{figure*}
% 
% 
\begin{figure*}
\centering
\includegraphics[trim={1.3cm 0 0 0},width=2.\columnwidth]{plots/TNChiBHBiasesOverCIWidths_CI90_0_Lambda_SNR_linear}
% \includegraphics[width=\columnwidth]{plots/TNChiBHBiasesOverCIWidthsVsSNR_All_CI68_3.pdf}
% \includegraphics[width=\columnwidth]{plots/TNChiBHBiasesOverCIWidthsVsSNR_All_CI90_0.pdf}
\caption{This figure shows the ratio of the systematic and statistical
measurement errors for BH spins, with other attributed identical to 
Fig.~\ref{fig:TN_chirpMassBias_vs_Lambda_SNR}, and~\ref{fig:TN_EtaBias_vs_Lambda_SNR}.
Similar to the case of mass parameters, we find that below
$\rho\approx 30$, ignoring tidal effects in templates introduces minor systematic effects,
which remain subdominant to the statistical measurement uncertainties.
}
\label{fig:TN_BHspinBias_vs_Lambda_SNR}
\end{figure*}
% 
% 
Past and future GW search and source characterization efforts with Advanced LIGO
have used (or plan to) BH-BH waveform templates for NS-BH mergers, ignoring
the effect of the tidal deformation of the neutron star due to its companion hole
on the emitted GWs. In this section we present the first fully Bayesian Monte-Carlo
study of the effects of this assumption on the recovery of other physical parameters
of NS-BH sources. We do so by comparing the systematic biases introduced by this 
assumption to the overall statistical uncertainties in the measurement of binary 
mass and spin parameters. We find that unless the single-detector SNR of the signal
exceeds $30$ (for design Advanced LIGO sensitivity), the bias accurued by ignoring
tidal effects in filtering templates will remain smaller than the statistical 
uncertainty of the same measurement.

We first take a set of NSBH binary waveforms, which include the modeling of spin and
tidal effects. Binary mass-ratio is sampled
uniformly over the range $q\in[2,5]$ taking $q=\{2,3,4,5\}$ as our test points. The
mass of the NS is fixed to $\mns=1.35M_\odot$, and therefore knowing $q$ gives $\mbh$.
The spin of the NS is fixed to zero. The spin of the BH is sampled uniformly over the
range $\chibh\in[-0.75, +0.75]$, choosing $\chibh=\{-0.5, 0, +0.5, +0.75\}$ as our
test points. Finally, we vary the deformability parameter uniformly over 
$\lambdans\in[0,2000]$, choosing $\lambdans=\{500, 1000, 1500, 2000\}$ plus an additional
$\lambdans=800$ case. Waveform are generated for all possible combinations of these
parameters, and rescaled to different strengths with $\rho=\{20, 30, 50, 70\}$. For all
of the resulting signals, we perform a fully Bayesian parameter estimation aimed at
measuring binary masses and spins from the signal. Our analysis
templates differ from the signal in only one sense: all tidal terms are set to zero
in the former, i.e. they correspond to BHBH mergers.

In Fig.~\ref{fig:TN_chirpMassBias_vs_Lambda_SNR} we show the ratio of the systematic
bias in the measured binary chirp mass $\mchirp$ versus its true value, to the 
statistical uncertainty of the measurement itself. $\mchirp$ is the mass combination
that occurs at leading order in the perturbative expression for GW strain from 
compact binaries, and is therefore very precisely measurable. Even so, we find that
for $\rho=30$ the signal is not loud enough for the systematic bias in $\mchirp$
to exceed a fraction of the statistical uncertainty. When we increase the SNR to 
$\rho=50$ (middle column), we find that if we have two favorable conditions present:
(a) that the BH spin be $\gtrsim 0.4$, {\it and} (b) the NS's true deformability
be large enough that its $\lambdans \gtrsim 1000$, the systematic biases from
ignoring tidal effects in templates reach between $50-100\%$ of the statistical
uncertainty. Therefore at this signal strength, including terms involving the neutron
star's internal structure in waveform models becomes increasingly gainful. At 
even higher SNRs, say $\rho=70$ (right column), under the same favorable conditions
as listed above, the systematic bias in $\mchirp$ measurement {\it exceeds} its 
statistical measurement uncertainty. If the BH is even mildly spinning, say
$\chibh\gtrsim 0.2$, with the NS mildly deformable, say $\lambdans\gtrsim 800$, 
we still are affected in our $\mchirp$ measurement at the $50\%$ level by the 
use of point-particle waveform templates. Finally, we note that the measured (median)
$\mchirp$ is always higher than its true value. This is reasonable since the tidal 
deformation of NSs close to merger reduces the GW signal power in high frequencies, so
it would look like a BH-BH signal of a higher mass (and therefore lower merger or
peak GW-power frequency).


In the next figure, Fig.~\ref{fig:TN_EtaBias_vs_Lambda_SNR}, we show the same
information (i.e. the ratio of systematic bias to statistical uncertainty in parameter
measurement) for the symmetric mass-ratio $\eta$. Note that $\eta$
is bounded from above and below to within $\eta\in[0, 1/4]$. We immediately observe
that the measured (median) $\eta$ is always lowered below its true value due to the 
lack of $\lambdans$-dependent terms in the filter template model. This bias, as for 
the $\mchirp$ bias, decreases the merger frequency for the BH-BH template, which must
better match the NS-BH merger with its quietened merger and ringdown. At SNR $\rho=30$
the systematic bias in $\eta$ measurement stays mostly around $10-50\%$ of its statistical
measurement uncertainty. For the sole case of a very deformable NS with $\lambdans=2000$,
the missing tidal pieces in template models could cause errors that are comparable to 
that due to noise (i.e. statistical uncertainties). For a still stronger signal with 
$\rho=50$, we find that 
when the following conditions are fulfilled, the measurement of $\eta$ is {\it completely
compromised} due to template biases: (a) BH mass $\mbh\leq 5M_\odot$, (b) BH spin
$\chibh\gtrsim +0.4$, and (c) $\lambdans\gtrsim 1500$. Even under more moderate restrictions
on BH and NS parameters, we find that $\rho=50$ is loud enough to necessitate the 
use of tidal-modeled templates. This observation is further strengthened if we divert 
our attention to the third column in Fig.~\ref{fig:TN_EtaBias_vs_Lambda_SNR} that presents
the case for even higher SNRs ($\rho=70$).

Moving on from mass to spin parameters, we now consider the measurement of spin
angular momentum of the BH $\chibh$. The ratio of the systematic bias introduced into
and the statistical uncertainty associated with $\chibh$ measurement is shown in
Fig.~\ref{fig:TN_BHspinBias_vs_Lambda_SNR}. The presentation of information in this 
figure is identical to that in Fig.~\ref{fig:TN_chirpMassBias_vs_Lambda_SNR}
and~\ref{fig:TN_EtaBias_vs_Lambda_SNR}. A diverging colormap is used because both 
extremes of the colorbar range point to large systematic biases, while its zero (or
small) value lies in the middle. For the lowest considered SNR $\rho=30$, $\chibh$
bias is negligible, being below $50\%$ of its statistical measurement uncertainty.
For the most deformable NS structure {\it and} with a low-mass BH, the statistical
bias in $\chibh$ does become an important error. This is furthered at higher SNR, 
considering $\rho=50$ next (middle column of Fig.~\ref{fig:TN_BHspinBias_vs_Lambda_SNR}).
We find that for low-mass BHs (with masses below $4.5M_\odot$), the systematic 
measurement bias for BH spin exceeds its statistical uncertainty to become the 
dominant source of error. Not surprisingly, the systematic bias at both spin 
extremes pushes towards milder spins, since the prior excludes any more extreme
spins than $[-0.75, +0.75]$ and that pushes the bulk of the posterior inwards.
However as the signal's $\lambdans$ is increased from $500\rightarrow 2000$,
low-mass BH-NS binaries with large spins on the BH show very large systematic biases
in spin values, that become larger than the statistical uncertainty for $\rho\geq 30$
and reach up to $4\times$ the same quantity as $\rho\rightarrow 70$.


Summarizing these results, we find that irrespective of system parameters, below a
signal-to-noise ratio of $30$, our measurements of mass and spin parameters of 
astrophysical BHNS binaries will remain limited by the intrinsic uncertainty due to 
instrument noise, and do not depend on whether we include tidal effects in template
models. However, when the signal-to-noise ratio exceeds $30$ the systematic bias in
binary mass and spin measurements become comparable to and exceed the uncertainty
due to noise. Of the different parameters considered, we find that the measurement 
of $\eta$ degrades worst (in a relative-error sense) amongst all other parameters
of BHNS binaries, due to the use of BH-BH waveforms in deciphering an NSBH signal.






%%%%%%%%%%%%%%%%%%%%%%%%%%%%%%%%%%%%%%%%%%%%%%%%%%%%%%%%%%%%%%%%%%%%%%%%%%%%%%%
\section{What do we gain by using templates that include NS matter effects?}\label{s1:PEwithNS}
%%%%%%%%%%%%%%%%%%%%%%%%%%%%%%%%%%%%%%%%%%%%%%%%%%%%%%%%%%%%%%%%%%%%%%%%%%%%%%%
% 
\begin{figure*}
\centering    
\includegraphics[trim={3cm 0 0 0},width=2.\columnwidth]{plots/TTLambdaCIWidths90_0_Lambda_SNR.pdf}
%\includegraphics[trim={3cm 0 0 0},width=2.3\columnwidth]{plots/TTLambdaCIWidths90_0_Lambda_SNR.pdf}
\caption{This figure shows the statistical uncertainty in the measurement of
$\lambdans$, as a percentage of the injected signal's $\lambdans$ value. In each panel,
the same is shown as a function of the BH mass and spin, keeping the 
$\lambdans$ and injection's SNR $\rho$ fixed (given in the panel). Each row contains
panels with the same value of $\lambdans$, with $\rho$ increasing from left to right.
Each column contains panels with the same value of $\rho$, with $\lambdans$ 
increasing from top to bottom.
% 
Contours are drawn in each panel demarkating regions where we can constrain the
$\lambdans$ parameter well (within a factor of two of the injected value).
% 
We note that, as expected, the measurement accuracy for $\lambdans$ improves with (i) increasing
SNR, (ii) decreasing BH mass, (iii) increasing BH spin, and 
(iv) increasing $\lambdans$, i.e. the tidal deformability of the neutron star.
}
\label{fig:TT_LambdaCIWidths90_0_Lambda_SNR}
\end{figure*}
% 
\begin{figure*}
\centering    
\includegraphics[width=0.9025\columnwidth]{plots/TTSNRThresholdFor200LambdaMeasurement_BHspin_BHmass_Lambda1000_0_CI90_0}
\includegraphics[width=0.9025\columnwidth]{plots/TTSNRThresholdFor100LambdaMeasurement_BHspin_BHmass_Lambda1500_0_CI90_0}
% \includegraphics[width=1.025\columnwidth]{plots/TTSNRThresholdFor100LambdaMeasurement_BHspin_BHmass_Lambda2000_0_CI90_0}\\
\caption{In these figures we highlight the regions of BH mass and spin plane where we can 
constrain the $\lambdans$ of its companion NS within a factor of $1-2$ of its true value.
Panels correspond to different signal strength, with $\rho$ increasing from left to right.
Within each panel, $\lambdans$ measurement uncertainty contours are shown as functions of BH 
mass and spin, with line-types corresponding to different uncertainty levels, and
colors showing the injected $\lambdans$.
% We note that this information is, in principle, contained in 
% Fig.~\ref{fig:TT_LambdaCIWidths90_0_Lambda_SNR}, which we gather here to better understand
% the effect of NS's deformability itself on its measurement. 
As hinted at in Fig.~\ref{fig:TT_LambdaCIWidths90_0_Lambda_SNR}, we note that the measurability
of NS matter
effects improves with all factors that enhances the signature of the NS's disruption on
the \textit{detectable} portion of the emitted GW signal (implying, within a frequency band
set by the detectors).
}
\label{fig:TT_SNRThresholds_BHspin_BHmass_CI90_0}
\end{figure*}
%
\begin{figure*}
\centering    
\includegraphics[width=0.9025\columnwidth]{plots/TTLambdaErrorCurves_BHspin_BHmass_SNR20_CI90_0}
\includegraphics[width=0.9025\columnwidth]{plots/TTLambdaErrorCurves_BHspin_BHmass_SNR30_CI90_0}
\caption{In these figures we highlight the regions of BH mass and spin plane 
where we can constrain the $\lambdans$ of its companion NS within a factor of 
$1-2$ of its true value. Panels correspond to different signal strength, with 
$\rho$ increasing from left to right. Within each panel, $\lambdans$ 
measurement uncertainty contours are shown as functions of BH mass and spin,
with line-types corresponding to different uncertainty levels, and colors 
showing the injected $\lambdans$. 
% We note that this information is, in principle, contained in 
% Fig.~\ref{fig:TT_LambdaCIWidths90_0_Lambda_SNR}, which we gather here to better understand
% the effect of NS's deformability itself on its measurement. 
As hinted at in Fig.~\ref{fig:TT_LambdaCIWidths90_0_Lambda_SNR}, the 
measurability of NS tidal deformability improves with all factors that 
enhances the signature of the NS's disruption on the \textit{detectable} 
portion of the emitted GW signal (implying, within a frequency band set by the
detectors).
}
\label{fig:TT_LambdaErrorCurves_BHspin_BHmass_CI90_0}
\end{figure*}
%

In the previous section we showed that we begin to care for the effects of NS's
deformation due to its companion hole's tidal field, if (a) the BH is sufficiently
small, (b) BH spin is aligned and $\gtrsim +0.4$, and (c) the NS is
not very compact, with $\lambdans\gtrsim 1000$. The first two conditions
enhance tidal effects on emitted GWs by increasing the number of orbits at small
separation, where the deviation of NSBH and BHBH dynamics is the largest. The third,
i.e. (c), simply makes the NS more easily deformable. All of them reduce the onset 
frequency for NS's disruption close to merger. Therefore, systems that fulfil the 
above conditions have considerably (or, at least, the most) different GW morphology
when compared to equivalent BHBH coalescences. Ultimately, the extent of 
tidal modulations of emitted GWs depends on the equation of state of neutron star
matter, which becomes an indirect measurable in GW astrophysics.
% 
However, if we do indeed receive GWs from such a binary source, what information can we
possibly extract from it about the internal structure of neutron stars?
In this section we consider this question in perspective of measuring the tidal 
deformability parameter of neutron stars purely from gravitational waves they emits 
while orbiting spinning black holes.
% 
We find that when the BH is sufficiently spinning in the prograde sense,
with $\chibh\gtrsim +0.7$, even at a relatively possible signal strengths ($\rho=20$),
we can begin to constraint $\lambdans$ at the level of $\pm 75\%$ errors with a $90\%$
confidence level. If the signal is stronger, say $\rho=30$, we can contraint
$\lambdans$ better with only $\pm 50\%$ errors, for binaries with large BH spins.
This trend continues as we vary the SNR from $\rho=30-50$ upwards. In other words,
with a single but moderately loud NSBH signal observation, Advanced LIGO will begin
to put constraints on NS matter properties. These constraints can consequently
be used to assess the likelihood of different equations of state for nuclear
matter, and possibly narrow the range they span.

  `

We use the same set of systems as in the previous section, with $q=\{2,3,4,5\}$,
$\chibh=\{-0.5,0,+0.5,+0.75\}$, $\lambdans=\{500, 800, 1000, 1500, 2000\}$ and SNRs
$\rho=\{20,30,50,70\}$. For each unique combination of these parameters, we obtain a
signal waveform using the tidal model of Lackey et al~\cite{Lackey:2013axa}.
We use the same waveform model as input to our Bayesian parameter estimation
method in order to obtain estimates for different binary parameters alongwith 
their full probability distributions, including for the combined NS tidal
deformability parameter $\lambdans$.


The precision with which $\lambdans$ can be measured is given by the width of 
its $90\%$ confidence interval, as a fraction of the true value. 
In Fig.~\ref{fig:TT_LambdaCIWidths90_0_Lambda_SNR} we show the same as a function
of binary parameters and signal strength. Each panel shows the percent statistical
uncertainty $\dlambda$ in the measurement of $\lambdans$ as a function of BH 
masses and spins, with NS parameters and signal strength being held fixed. 
In each row, the signal strength is increased from
left to right with $\rho = 20\rightarrow 70$. In each column, the deformability
of the NS is increased from top to bottom, with $\lambdans=500\rightarrow 2000$.
% 
We find that at SNRs around $\rho=30$, we begin to constrain $\lambdans$ 
meaningfully with $\dlambda\approx 100-150\%$. Amongst other parameters, 
the BH spin and mass play a big role. A smaller BH with a larger spin always
allows for a more precise measurement on $\lambdans$. Visually we can see that
in each panel the bottom right corner, which corresponds to low-mass BHs
with large spins, is the region of smallest measurement errors on $\lambdans$.
The actual deformability of the NS also plays an important role on its own
measurability. For e.g., when $\lambdans\leq 1000$, it is fairly difficult
to meaningfully constrain $\lambdans$ without getting remarkably lucky and
finding such a binary with $\rho\gtrsim 50$.



In Fig.~\ref{fig:TT_SNRThresholds_BHspin_BHmass_CI90_0} we take the case of
moderately deformable neutron stars, with actual $\lambdans = 1000$ (left 
panel) and $1500$ (right panel), and ask how loud such a signal will have
to be for us to put meaningful constraints on the tidal deformability of NSs.
In the left panel, as a function BH mass and spin, 
we show the lowest SNR value $\rho_c$ in order for the signal to have enough
information that we can constrain $\dlambda$ below $200\%$. Note that 
being able to constrain $\lambdans$ within a factor of $2$ interval 
(i.e. within $\pm 100\%$) is a fairly modest requirement. We find promising 
results, which suggest that there is a fairly wide range of BH spins, less
wide (but non-trivial) range of BH masses, within
which if we get a moderately loud NSBH signal with $\rho\approx 20$, we constrain
$\lambdans$ within a factor of $2$ interval (i.e. within $\pm 100\%$) around
the true value. While this is not very precise, it can possibly allow us to
rule out candidates for the equation of state of neutron star matter.
% 
If we want to constrain $\lambdans$ better with a single observation, however,
we need SNRs of $30$ or higher in each detector. This can be seen from the right
panel of Fig.~\ref{fig:TT_SNRThresholds_BHspin_BHmass_CI90_0}, which is similar
to the left but for a higher $\lambdans (=1500)$ system, with a more stringent
constraint requirement that $\lambdans$ be measured within a factor of $1$ of 
its true value (i.e. with $\pm 50\%$ errors, instead of $\pm 100\%$). Even for
the most conducive cases, we {\it need} the SNR to be above $30$ to reach our 
accuracy threshold (which is more than $3\times$ less likely to happen than
getting a signal with $\rho=20$).


Lastly, in Fig.~\ref{fig:TT_LambdaErrorCurves_BHspin_BHmass_CI90_0} we show the
conditions that BH parameters must fulfil beside having an SNR of $20$ (left 
panel) or $30$ (right panel), to obtain meaningful contraints on $\lambdans$.
At $\rho=20$, the left panel shows that we need high BH spins 
$\chibh\gtrsim +0.6$ and a very deformable NS with $\lambdans\geq 1500$, in 
order to put a factor of $2$ bound on $\lambdans$. At $\rho=30$, with
$\lambdans\gtrsim 1000$ and $\chibh\geq+0.6$ we can put a similar factor of
$2$ bound on $\lambdans$, but with $\chibh\sim+0.75$ and an extremely deformable
NS with $\lambdans=200$ one could even put a $\pm 50\%$ errorbar on 
$\lambdans$ measurement.




In summary, with a single moderately loud BHNS observation, we could constrain
the NS compactness parameter $\lambdans$ within $\pm 100\%$ of its true value.
To measure better with one observation, we will need to get lucky with signals
with $\rho\geq 30$ and high BH spins, or $\rho\geq 50$.



%%%%%%%%%%%%%%%%%%%%%%%%%%%%%%%%%%%%%%%%%%%%%%%%%%%%%%%%%%%%%%%%%%%%%%%%%%%%%%%
\section{Combining observations: looking forward with Advanced LIGO}\label{s1:multiple_observations}
%%%%%%%%%%%%%%%%%%%%%%%%%%%%%%%%%%%%%%%%%%%%%%%%%%%%%%%%%%%%%%%%%%%%%%%%%%%%%%%
% 
\begin{figure}
\centering    
\includegraphics[width=1.05\columnwidth]{plots/pdfLambda_vs_N_L800.pdf}
% \includegraphics[width=1.\columnwidth]{plots/FillBetweenNormErrorBarsLambda_vs_N_L800.pdf}
\caption{%
Posterior probability distributions for $\lambdans$ (colored curves), and 
associated $90\%$ confidence intervals (grey vertical lines), shown for 
different number of accumulated observations $N$.
}
\label{fig:TT_Lambda_vs_N_L800_CI90_0}
\end{figure}
%
% 
\begin{figure*}
\centering    
\includegraphics[width=.9\columnwidth]{plots/FillBetweenErrorBarsLambda_vs_N_L2000.pdf}
\includegraphics[width=.9\columnwidth]{plots/FillBetweenErrorBarsLambda_vs_N_L1500.pdf}\\
\includegraphics[width=.67\columnwidth]{plots/FillBetweenErrorBarsLambda_vs_N_L1000.pdf}
\includegraphics[width=.67\columnwidth]{plots/FillBetweenErrorBarsLambda_vs_N_L800.pdf}
\includegraphics[width=.67\columnwidth]{plots/FillBetweenErrorBarsLambda_vs_N_L500.pdf}
\caption{Filled-region plots showing the median and $90\%$ confidence intervals
for $\lambdans$ measurement, as a function of the number of observed events $N$.
Different panels correspond to different injected $\lambdans$ values, which is shown
in their title. In each panel are shown two independent filled regions. 
%  
 One in green shows the recovered confidence interval
 for $\lambdans$ (on the left y-axis), normalized by its true value,
 as a function of the 
 number of observed events $N$. The recovered median is shown by the
 green line-circled curve. The pair of horizontal green dashed (dotted)
 lines show the $\pm 50\%$ ($\pm 25\%$) symmetric error bounds around
 the true $\lambdans$ value, which is shown at y-value $=1$ by definition.
% 
 The second filled region, in grey, shows the recovered
 confidence interval for neutron star compactness, without normalizing.
 The corresponding y-values are shown on the *right* y-axis. The horizontal grey
 dashed line traces the true value of $\lambdans$, and is what we expect
 the filled region to approach with increasing number of observations.
%  
 Finally, the grey dotted lines are contours of $1/\sqrt{N}$, drawn to aid the eye.
}
\label{fig:TT_Lambda_vs_N_L500_2000_CI90_0}
\end{figure*}
%
\begin{figure}
\centering    
\includegraphics[width=1.05\columnwidth]{plots/FillBetweenRelErrorBarsLambda_vs_NShifted_AllLambda.pdf}
\caption{%
Posterior probability distributions for $\lambdans$ (colored curves), and 
associated $90\%$ confidence intervals (grey vertical lines), shown for 
different number of accumulated observations $N$.
}
\label{fig:TT_Lambda_vs_N_CI90_0}
\end{figure}
% 
\begin{figure}
\centering    
\includegraphics[trim=20 0 0 0, width=1.05\columnwidth]{plots/LambdaMedian_vs_N_AllPopulation}
\caption{This figure shows the fractional difference between the median
 recovered value from the posterior probability distribution for $\lambdans$
 as a function of $N$. Different curves correspond to different injected values
 of $\lambdans$, and are shaded from light to dark in direct proportion. 
 The horizontal dashed brown line marks $10\%$ deviation of the median from the 
 true value.
%  
We find that within $\approx 10$ detections, the median measured value for
$\lambdans$ would estimate the true value for the star with the remarkable 
accuracy of $10\%$.
}
\label{fig:TT_Lambda_vs_N_L500_2000_CI90_0_AllInOne}
\end{figure}
%
% 
% \begin{figure*}
% \centering    
% \includegraphics[width=1.03\columnwidth]{plots/RelErrorLambdaMedian_vs_N.pdf}
% \includegraphics[width=1.03\columnwidth]{plots/LambdaCIWidths_vs_N.pdf}
% \caption{{\it Left}: In the main panel is shown the width of the measured $90\%$ 
% confidence interval for $\lambdans$, normalized by its true value,
%  as a function of the number of events observed $N$. In the inset, we
%  show the same measurement uncertainty, but not normalized, and on a 
%  logarithmic axis for $N$ as well.
% % 
%  From both the inset and the main plot, we find that for the first $3-4$ events
%  our measurement is prior limited. As $N$ increases, the uncertainty falls 
%  roughly as $1/\sqrt{N}$ with intermittent loud signals causing more rapid 
%  (than $1/\sqrt{N}$) narrowing of the confidence interval.
% %  
%  {\it Right}: This figure shows the fractional difference between the median
%  recovered value from the posterior probability distribution for $\lambdans$
%  as a function of $N$. Different curves correspond to different injected values
%  of $\lambdans$, and are shaded from light to dark in direct proportion. 
%  The horizontal dashed brown line marks $10\%$ deviation of the median from the 
%  true value.
% %  
% We find that within $\approx 10$ detections, the median measured value for
% $\lambdans$ would estimate the true value for the star with the remarkable 
% accuracy of $10\%$.
% }
% \label{fig:TT_LambdaError_vs_N_L500_2000_CI90_0}
% \end{figure*}
%LambdaCIWidths_vs_N_AllPopulations_Log_L
% 
%
% 
\begin{figure*}
\centering    
\includegraphics[width=.9\columnwidth]{plots/LambdaCIWidths_vs_N_AllPopulations_Log_L2000.pdf}
\includegraphics[width=.9\columnwidth]{plots/LambdaCIWidths_vs_N_AllPopulations_Log_L1500.pdf}\\
\includegraphics[width=.67\columnwidth]{plots/LambdaCIWidths_vs_N_AllPopulations_Log_L1000.pdf}
\includegraphics[width=.67\columnwidth]{plots/LambdaCIWidths_vs_N_AllPopulations_Log_L800.pdf}
\includegraphics[width=.67\columnwidth]{plots/LambdaCIWidths_vs_N_AllPopulations_Log_L500.pdf}
\caption{ In the main panel is shown the width of the measured $90\%$ 
confidence interval for $\lambdans$, normalized by its true value,
 as a function of the number of events observed $N$. In the inset, we
 show the same measurement uncertainty, but not normalized, and on a 
 logarithmic axis for $N$ as well.
% 
 From both the inset and the main plot, we find that for the first $3-4$ events
 our measurement is prior limited. As $N$ increases, the uncertainty falls 
 roughly as $1/\sqrt{N}$ with intermittent loud signals causing more rapid 
 (than $1/\sqrt{N}$) narrowing of the confidence interval.
}
\label{fig:TT_LambdaError_vs_N_L500_2000_CI90_0}
\end{figure*}
% 
\begin{figure}
\centering    
\includegraphics[width=1.05\columnwidth]{plots/PowerLawCoefficient_LambdaErrorvsN_vs_N.pdf}
\caption{This figure shows the relative uncertainty in the measurement of $\lambdans$,
as a function of $\lambdans$ itself. The drawn curves are cumulative, in the sense 
that the $N=n$ curve includes information from all of the first $n$ observed events.
Those corresponding to the first $N\leq 5$ events are shown in fading shades of red,
with the ones for higher value of $N$ being shown in grey (going from light to dark in
proportion to $N$).
Green dash-dotted lines are contours of $y=x^{-(1+1/5)}$.
% 
\prayush{We note that the measurement uncertainty from a population follows the
equation for the green lines.}
% 
}
\label{fig:TT_PowerLawLambdaErrorVsN}
\end{figure}
%
% 
\begin{figure}
\centering    
% \includegraphics[width=1.03\columnwidth]{plots/LambdaRelErrorBars_vs_Lambda_AllPopulations_N80_Log.pdf}
\includegraphics[width=\columnwidth]{plots/PowerLawCoefficient_LambdaErrorvsLambda_vs_N_AllPopulations.pdf}
\caption{This figure shows the relative uncertainty in the measurement of $\lambdans$,
as a function of $\lambdans$ itself. The drawn curves are cumulative, in the sense 
that the $N=n$ curve includes information from all of the first $n$ observed events.
Those corresponding to the first $N\leq 5$ events are shown in fading shades of red,
with the ones for higher value of $N$ being shown in grey (going from light to dark in
proportion to $N$).
Green dash-dotted lines are contours of $y=x^{-(1+1/5)}$.
% 
\prayush{We note that the measurement uncertainty from a population follows the
equation for the green lines.}
% 
}
\label{fig:TT_Lambda_vs_Lambda_L500_2000_CI90_0_AllInOne}
\end{figure}



\textbf{Multiple identical sources at low SNR: }\label{s2:identical_multiple}
% 
First we show a crude calculation which tells us that approximately $8$ (or $16$) 
observations
of NSBH signals with $\rho\geq 10$ distributed uniformly in effective volume, would
give us similarly accurate a measurement of $\lambdans$ as would a single detection
with $\rho=50$ (or $\rho=70$). 


We want to distribute signals in effective distance $\deff$, which is a combination of 
luminosity distance, inclination angle of the binary's orbital angular momentum
with respect to the detector, and its various sky angles. The signal strength 
scales inversely with $\deff$, if it comprises only of the dominant $(2,2)$ mode, allowing
for various angle-dependent factors to scale out beside the luminosity distance.
% 
In order to consider multiple events, lets imagine a population uniformly distributed
in effective volume, i.e. within a sphere of radius $\deff^\mathrm{max}$. The 
radius of this sphere is set by the lowest SNR that is distinguishable from noise
by LIGO searches. Now, divide the sphere into $I$ shells of equal width. The radius
of the $i$-th sphere would then be $D_i = \deff^\mathrm{max} (i - 1/2)/I$. The 
measurement uncertainty in $\lambdans$ scales inversely with the SNR, and hence
directly with the effective distance to the source. Therefore, if we have a 
measurement error $\sigma_0$ for a source located at $\deff = D_0$, the same error
for the same source located within the $i$-th shell would be 
$\sigma_i = \sigma_0 \dfrac{D_i}{D_0}.$
Independent measurements of $\lambdans$ for identical sources at different distances
would have their combined error given by
\begin{equation}\label{eq:1oversigma}
\frac{1}{\sigma^2} = \sum_{i=1}^I \frac{N_i}{\sigma_i^2} = \left(\frac{D_0}{\sigma_0}\right)^2 \sum_{i=1}^I\frac{N_i}{D_i^2},
\end{equation}
where $N_i$ is the number of sources detected in the $i$-th shell.
% , and is a random
% variable, with its probability proportional to the volume of the shell, i.e.
% $$
% p(N_i) \propto \frac{V_i}{V_\mathrm{total}} \propto \dfrac{\left(\deff^\mathrm{max} \frac{i}{I}\right)^3 - \left(\deff^\mathrm{max} \frac{i-1}{I}\right)^3}{(\deff^\mathrm{max})^3} \propto \frac{1}{I^3} [i^3 - (i-1)^3].
% $$
% Therefore, the expected number of detections in the $i$-th shell would be
% $$
% \langle N_i\rangle = \frac{N}{I^3} [i^3 - (i-1)^3],
% $$
% where $N=\sum_{i=1}^I N_i$ is the total number of NSBH detections. $N$ is expected
% to be Poisson distributed around the mean detection rate
% $\mathcal{R}\equiv\langle N\rangle$, where $\mathcal{R}$ can vary from $0.6-1000$ per
% $\mathrm{Gpc}^3$ per year~\cite{Abadie:2010cfa}. If there are $n$ resulting detections
% a year per unit effective volume, then
% \begin{equation}
% \langle N\rangle = \int_0^{\deff^\mathrm{max}} 4\pi n D^2 \D D = \frac{4\pi}{3} n (\deff^\mathrm{max})^3,
% \end{equation}
% and
% \begin{eqnarray}
%  \langle \frac{1}{\sigma^2}\rangle &=& \left(\frac{D_0}{\sigma_0}\right)^2 \int_0^{\deff^\mathrm{max}} \frac{4\pi n D^2 }{D^2}\D D\\
%  &=& \left(\frac{D_0}{\sigma_0}\right)^2 4\pi n \deff^\mathrm{max},
% \end{eqnarray}
% where in the previous equation we have converted the summation in Eq.~\ref{eq:1oversigma} 
% to an integral. 
Continuing on this line of inquiry, Ref.~\cite{Markakis:2010mp} calculates the 
root-mean-square averaged measurement error from 
$\langle N\rangle:=\langle\sum N_i\rangle$ sources distributed uniformly in
effective volume within a sphere of radius $\deff^\mathrm{max}$ to be
\begin{equation}\label{eq:rmsSigmaIdenticalSources}
 \sigma_{avg} := \frac{1}{\sqrt{\langle1/\sigma^{2}\rangle}} = \frac{\sigma_0}{D_0} \deff^\mathrm{max} \frac{1}{\sqrt{3\langle N\rangle}},
\end{equation}
given a fiducial pair $(\sigma_0, D_0)$.
Further, it is straightforward to deduce from Eq.~\ref{eq:rmsSigmaIdenticalSources}
that the same measurement certainty as afforded by a single
observation with a high SNR $\rho_c$ can be obtained from $N = \rho_c^2/300$ observations
uniformly distributed in effective volume with SNRs $\geq 10$. E.g., to get to the level 
of certainty afforded at SNR$=70$, we would need $49/3\approx 16-17$ realistic detections.

There are two main caveats to this rudimentary calculation, which was outlined in 
Ref.~\cite{Markakis:2010mp}, (i) it applies to identical sources,
which is astrophysically next to impossible to achieve exactly, but could perhaps happen 
approximately; (ii) the combined average measurement error $\sigma_{avg}$ has been defined
as in the first equality in Eq.~\ref{eq:rmsSigmaIdenticalSources}. To overcome both,
we next perform a Bayesian population analysis.


\textbf{Astrophysical source population: }\label{s2:astro_multiple}
% 
Imagine that we have $N$ stretches of data, $d_1, d_2, \cdots, d_N$, each 
containing a single signal emitted by an NSBH binary. Each of these signals can
be characterized by the following binary parameters:
$\,\vec{\theta} = \{\mbh, \mns, \chibh, \chins, \rho\}\cup\{\lambdans\}$. Note
that we have folded in the luminosity distance, inclination angle of the 
binary's orbital angular momentum with respect to the plane of the detector,
and source's sky location angles into a single parameter $\rho$, which is the 
SNR of the received signal. This is possible because of our simplification of 
using only the dominant $l=|m|=2$ modes of the GW strain~\footnote{The 
$l=|m|=2$ multipoles of the GW strain, when decomposed in a spin $-2$ weighted
spherical harmonic basis, contain more than $99\%$ of the total signal power}
that makes various angle parameters completely degenerate with distance. 
Further, let $K$ denote all of our collective prior knowledge, except for 
expectations on binary parameters themselves, which enter the following 
calculation explicitly. For instance, $K$ includes our assumption that all
NS's in a single population have the same deformability parameter $\lambdans$, 
and its cumulative measurement is therefore possible.
% 
Using Bayes' theorem, the measured probability distribution function for 
$\lambdans$ from $N$ observations is given by~\footnote{%
Eq.~\ref{eq:lambdaMultiple} includes the following implicit factor to normalize
its right-hand side: $\dfrac{\prod_i p(d_i|K)}{\int p(\lambdans)\,p(d_1, d_2, \cdots, d_N | \lambdans; K)\,\D\lambdans}$}
\begin{equation}\label{eq:lambdaMultiple}
 p(\lambdans | d_1, d_2, \cdots, d_N; K) = p(\lambdans | K)^{1-N}\prod_{i=1}^N p(\lambdans | d_i, K),
\end{equation}
where $p(\lambdans|K)$ is the prior expectation for $\lambdans$, and all $N$ 
observations are taken to be mutually independent. A priori, we assume
that no particular value of $\lambdans$ is preferred over another, and 
$\lambdans\leq 4000$, i.e.
\begin{equation}\label{eq:lprior}
 p(\lambdans | K) = \dfrac{1}{4000}\,\mathrm{Rect}\left(\frac{\lambdans-2000}{4000}\right).
\end{equation}
% 
In the second set of terms in Eq.~\ref{eq:lambdaMultiple} (of the form 
$p(\lambdans | d_i, K)$), each is the probability distribution for $\lambdans$
inferred {\it a posteriori} from the \textit{i}-th observation in isolation. 
They are calculated by integrating
\begin{equation}\label{eq:margpost}
 p(\lambdans | d_i, K) = \int \D \vec{\theta}\, p(\vec{\theta}, \lambdans | d_i, K),
\end{equation}
where $p(\vec{\theta}, \lambdans | d_n, K)$ is the inferred joint probability 
distribution of all source parameters $\vec{\theta}\cup\{\lambdans\}$ for the 
$i$-th event, as given by Eq.~\ref{eq:postprob}. By substituting
Eq.~\ref{eq:lprior}-\ref{eq:margpost} into Eq.~\ref{eq:lambdaMultiple}, we
calculate the probability distribution for $\lambdans$ as measured using $N$
independent events.




Our goal is to understand the improvement in our measurement of $\lambdans$
with the number of recorded events. To do so, we simulate populations of $N$
observations each, and quantify what we learn from each successive observation 
using Eq.~\ref{eq:lambdaMultiple}. For each population, we determine how
rapidly our median estimate for $\lambdans$ converges to the true value,
and how rapidly our confidence intervals for the same shrink, with increasing
$N$. In order to generate each population, we first fix
the NS properties: (i) NS mass $\mns=1.35M_\odot$, (ii) NS spin $\chins=0$
and (iii) NS tidal deformability $\lambdans=$ fixed value chosen from
$\{500,800,1000,1500,2000\}$. Each event is generated independently, sampling
the remaining source parameters uniformly from the following ranges: (i)
mass-ratio $q\in[2,5]$, (ii) BH spin $\chibh\in[0, 0.75]$, (iii) effective 
distance from within a spherical volume whose radius is given by the lowest 
detectable SNR (which we take as $\rho=10$). We remind ourselves that
once binary parameters are fixed, there is a one-to-one mapping between
effective distance and SNR. Finally, in order to mitigate computational
cost, we make an additional simplification.
We perform full Bayesian parameter estimation for a set of simulated signals
whose parameters are the vertices of the following regular hypercubic grid
(henceforth ``G''):
$q=\{2,3,4,5\}\times\chibh=\{-0.5,0,0.5,0.75\}\times\lambdans=\{500,800,1000,1500,2000\}\times\rho=\{10,20,30,50,70\}$.
The final population is given by substituting all of the $N$ events, generated
as above, with their respective nearest neighbours on the parameter grid G.


In Fig.~\ref{fig:TT_Lambda_vs_N_L800_CI90_0} we show results for a population
of $N=80$ events. The neutron star deformability is fixed at $\lambdans=800$
for all events. Each curve shows the probability distributions for $\lambdans$
as inferred from $N$ events, with $N$ ranging from $1-50$. We also mark the 
$90\%$ confidence intervals for $\lambdans$ as dark grey vertical lines,
one associated with each probability distribution curve. Distributions are 
normalized to unit area. From the figure, we can visualize the information 
information flow. The first few observations do not have enough information 
to bound $\lambdans$ (with $90\%$ confidence) much more than our prior from
Eq.~\ref{eq:lprior} does. However, the median of these cumulative distributions
track the true value of $\lambdans$ rapidly, reaching close in as few as $4$
observations.



In Fig.~\ref{fig:TT_Lambda_vs_N_L500_2000_CI90_0}, we show our measurement 
uncertainty for $\lambdans$ for different populations, with $\lambdans$ alone
changing between populations. Lets consider the top left panel first, which
corresponds to a population of the most deformable neutron stars considered,
with $\lambdans=2000$. The green filled region shows $\delta\Lambda/\Lambda$
as a function of $N$, where we take $\delta\Lambda$ to be the width of the 
$90\%$ confidence interval on $\lambdans$ after $N$ observations.
%
The green filled circles show the median measured value of $\lambdans$
as a function of $N$. The dashed line at y value $=1$ marks zero error, i.e.
$\delta\Lambda/\Lambda=0$. The dotted and dashed lines around this denote
$\pm 25\%$ and $\pm 50\%$ relative errors, respectively. Independently, the
light-grey filled shown the absolute measurement uncertainty $\delta\Lambda$
on the right y-axis , as a function of $N$. A horizontal grey dashed line
marks the true value of $\lambdans$. Contours of $1/\sqrt{N}$ are shown as 
dotted light grey curves to guide the eye while studying the dependence of
$\delta\Lambda$ on $N$.
% 







%%%%%%%%%%%%%%%%%%%%%%%%%%%%%%%%%%%%%%%%%%%%%%%%%%%%%%%%%%%%%%%%%%%%%%%%%%%%%%%
\section{Discussion}\label{s1:discussion}
%%%%%%%%%%%%%%%%%%%%%%%%%%%%%%%%%%%%%%%%%%%%%%%%%%%%%%%%%%%%%%%%%%%%%%%%%%%%%%%


The pioneering terrestrial observation of gravitational waves by Advanced LIGO
harbingers the dawn of an era of gravitational-wave astronomy where 
observations, more than theory, would play the driving role. Binaries of 
compact stellar objects evolve under the radiation reaction force of emitting
gravitational waves.


Conventional astronomical methods have, to date, made observations of ~2500
NSs with masses between $1.25-2.1 M_\odot$~\cite{Lyne:2004cj,
Demorest:2010bx,2013Sci...340..448A,atnfcatalog,mcgillmagnetarcatalog,
stellarcollapsemass}, although
tightly clustered within $1.35\pm0.5M_\odot$~\cite{stellarcollapsemass}.
%
Observations of over 2500 NSs have shown that the spin of NSs
$|\vec{\chi}_\mathrm{NS}|$ have magnitudes that are $<1\%$~\cite{Miller:2014aaa}.
%
On the other hand, indirect observations of stellar-mass BHs place their
masses between $5-35M_\odot$, with their spin angular momenta 
$|\vec{\chi}_\mathrm{BH}|$ ranging from small to nearly extremal (Kerr) values
(see, e.g., Refs.~\cite{McClintockEtAl:2006,Miller:2009cw,Gou:2014una} for 
examples of nearly extremal estimates of BH spins, Refs.~\cite{McClintock:2013vwa,
Reynolds:2013qqa} for recent reviews of astrophysical BH spin measurements,
and Figure 5 of Ref.~\cite{Miller:2014aaa} for a comparison of NS and BH spins).



%%%%%%%%%%%%%%%%%%%%%%%%%%%%%%%%%%%%%%%%%%%%%%%%%%%%%%%%%%%%%%%%%%%%%%%%%%%%%%%
% Acknowledgments
%%%%%%%%%%%%%%%%%%%%%%%%%%%%%%%%%%%%%%%%%%%%%%%%%%%%%%%%%%%%%%%%%%%%%%%%%%%%%%%
\begin{acknowledgments}
Acknowledgments
\end{acknowledgments}

%%%%%%%%%%%%%%%%%%%%%%%%%%%%%%%%%%%%%%%%%%%%%%%%%%%%%%%%%%%%%%%%%%%%%%%%%%%%%%%
\section*{References}
%%%%%%%%%%%%%%%%%%%%%%%%%%%%%%%%%%%%%%%%%%%%%%%%%%%%%%%%%%%%%%%%%%%%%%%%%%%%%%%
\bibliography{References/References}

\end{document}
