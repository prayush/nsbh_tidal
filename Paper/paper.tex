\documentclass[aps,prd,amsmath,floats,floatfix, twocolumn,
superscriptaddress,nofootinbib,showpacs]{revtex4-1}

\usepackage[colorlinks, pdfborder={0 0 0}, plainpages=false]{hyperref}
\usepackage{graphicx}
\usepackage{xspace}
\usepackage[usenames,dvipsnames]{color}
\usepackage{amssymb}

\newcommand{\roughly}{\mathchar"5218\relax} % Different from \sim in spacing

% Macros for text changes
\newcommand{\red}{\textcolor{red}}
\newcommand{\dan}[1]{\textcolor{WildStrawberry}{#1}}
\newcommand{\Mark}[1]{\textcolor{Cerulean}{#1}}
\newcommand{\larry}[1]{\textcolor{OliveGreen}{#1}}
\newcommand{\bela}[1]{\textcolor{Blue}{#1}}
\newcommand{\saul}[1]{\textcolor{Orange}{#1}}

% Macros for text notes and comments
\newcommand{\Note}[1]{\textcolor{blue}{\textbf{[#1]}}}


\newcommand{\Caltech}{\affiliation{Theoretical Astrophysics 350-17,
    California Institute of Technology, Pasadena, CA 91125, USA}}
\newcommand{\Cornell}{\affiliation{Center for Radiophysics and Space
    Research, Cornell University, Ithaca, New York 14853, USA}}
\newcommand{\CITA}{\affiliation{Canadian Institute for Theoretical
    Astrophysics, 60 St.~George Street, University of Toronto,
    Toronto, ON M5S 3H8, Canada}} %
\newcommand{\GWPAC}{\affiliation{Gravitational Wave Physics and
    Astronomy Center, California State University Fullerton,
    Fullerton, California 92834, USA}} %


%%%%%%%%%%%%%%%%%%%%%%%%%%%%%%%%%%%%%%%%%%%%%%%%%%%%%%%%%%%%%%%

\begin{document}

\title{
Measuring matter effects in Neutron star - Black hole binaries with Advanced LIGO
}

\author{People across the Atlantic}

\date{\today}

\begin{abstract}
Abstract
\end{abstract}

\pacs{}
% 04.25.D- Numerical relativity
% 04.25.dg Numerical studies of black holes and black-hole binaries
% 04.25.Nx Post-Newtonian approximation; perturbation theory; related approximations 
% 04.30.-w Gravitational waves (see also 04.80.Nn Gravitational wave detectors and experiments)
% 04.30.Db Wave generation and sources 
% 02.70.Hm Spectral methods

\maketitle

%%%%%%%%%%%%%%%%%%%%%%%%%%%%%%%%%%%%%%%%%%%%%%%%%%%%%%%%%%%%%%%%%%%%%%%%%%%%%%%
\section{Introduction}
%%%%%%%%%%%%%%%%%%%%%%%%%%%%%%%%%%%%%%%%%%%%%%%%%%%%%%%%%%%%%%%%%%%%%%%%%%%%%%%
Introduction

%%%%%%%%%%%%%%%%%%%%%%%%%%%%%%%%%%%%%%%%%%%%%%%%%%%%%%%%%%%%%%%%%%%%%%%%%%%%%%%
\section{Techniques}
%%%%%%%%%%%%%%%%%%%%%%%%%%%%%%%%%%%%%%%%%%%%%%%%%%%%%%%%%%%%%%%%%%%%%%%%%%%%%%%
\subsection{Modeling tidal deformation during inspiral}
Describe the PN terms used here

\subsection{Modeling tidal disruption near merger}
Describe the Lackey et al model

\subsection{MCMC methods}
Describe the emcee-based PE code

%%%%%%%%%%%%%%%%%%%%%%%%%%%%%%%%%%%%%%%%%%%%%%%%%%%%%%%%%%%%%%%%%%%%%%%%%%%%%%%
\section{SNR loss from ignoring NS tidal effects}
%%%%%%%%%%%%%%%%%%%%%%%%%%%%%%%%%%%%%%%%%%%%%%%%%%%%%%%%%%%%%%%%%%%%%%%%%%%%%%%
See what is the loss in SNR if we recover NSBH signals with a template bank of
BH-BH waveforms? It is very likely that the answer would be: $<1\%$. But it
is very interesting to show that for spinning BHs!

%%%%%%%%%%%%%%%%%%%%%%%%%%%%%%%%%%%%%%%%%%%%%%%%%%%%%%%%%%%%%%%%%%%%%%%%%%%%%%%
\section{Systematic bias in recovered parameters}
%%%%%%%%%%%%%%%%%%%%%%%%%%%%%%%%%%%%%%%%%%%%%%%%%%%%%%%%%%%%%%%%%%%%%%%%%%%%%%%
\begin{figure*}[h]
\centering 
\textbf{With BH-BH Templates}
\includegraphics[width=2\columnwidth]{plots/TN_chirpMassBias_vs_SNR_q23.pdf}\\ 
\textbf{With NS-BH Templates}
\includegraphics[width=2\columnwidth]{plots/TT_chirpMassBias_vs_SNR_q23.pdf}%\quad
\caption{These figures show the (fractional) difference between the median value of
recovered posterior distribution for binary chirp mass and the actual injected chirp mass,
as a function of the signal-to-noise-ratio (SNR) of the injected signal.
The top and bottom panels show results for parameter estimation performed using
waveform templates without and with NS matter effects included, respectively.
In each row, the left and right panel correspond to 
$q=m_\mathrm{BH}/m_\mathrm{NS}=\{2,3\}$, respectively. 
In each panel, the color of the curves correspond to the value of NS deformability 
parameter, and line style correspond to the dimensionless spin on the BH 
($\chi_\mathrm{BH}\equiv \dfrac{a}{m_\mathrm{BH}^2}$).
}
\label{fig:TNT_chirpMassBias_vs_SNR_q23}
\end{figure*}
% 
\begin{figure*}[h]
	\centering 
	\textbf{With BH-BH Templates}
	\includegraphics[width=2\columnwidth]{plots/TN_chirpMassCIWidth90_vs_SNR_q23.pdf}\\ 
	\textbf{With NS-BH Templates}
	\includegraphics[width=2\columnwidth]{plots/TT_chirpMassCIWidth90_vs_SNR_q23.pdf}%\quad
	\caption{These figures show the narrowing of the $90\%$ confidence 
		interval (recovered) for chirp mass, as a function of the 
		signal-to-noise-ratio (SNR). Note that the confidence interval
                width is normalized by the true parameter value.
                The binary parameters, i.e. black hole spin,
                mass-ratio, NS tidal deformability, are all depicted in the 
                same way as in Fig.~\ref{fig:TNT_chirpMassBias_vs_SNR_q23}.
	}
	\label{fig:TNT_chirpMassCIWidth90_vs_SNR_q23}
\end{figure*}
% 
\begin{figure*}[h]
\centering    
\textbf{With BH-BH Templates}
\includegraphics[width=2\columnwidth]{plots/TN_EtaBias_vs_SNR_q23.pdf}\\ 
\textbf{With NS-BH Templates}
\includegraphics[width=2\columnwidth]{plots/TT_EtaBias_vs_SNR_q23.pdf}%\quad
\caption{This figure is similar to Fig.~\ref{fig:TNT_chirpMassBias_vs_SNR_q23},
with the difference that we consider the systematic bias in the dimensionless
mass-ratio $\eta$ here.}
\label{fig:TNT_EtaBias_vs_SNR_q23}
\end{figure*}
% 
\begin{figure*}[h]
	\centering    
	\textbf{With BH-BH Templates}
	\includegraphics[width=2\columnwidth]{plots/TN_EtaCIWidth90_vs_SNR_q23.pdf}\\ 
	\textbf{With NS-BH Templates}
	\includegraphics[width=2\columnwidth]{plots/TT_EtaCIWidth90_vs_SNR_q23.pdf}%\quad
	\caption{This figure is similar to Fig.~\ref{fig:TNT_chirpMassCIWidth90_vs_SNR_q23},
		with the difference that we consider the confidence intervals in the recovery
		of the dimensionless
		mass-ratio $\eta$ here.}
	\label{fig:TNT_EtaCIWidth90_vs_SNR_q23}
\end{figure*}
% 
\begin{figure*}[h]
\centering    
\textbf{With BH-BH Templates}
\includegraphics[width=2\columnwidth]{plots/TN_BHspinBias_vs_SNR_q23.pdf}\\ 
\textbf{With NS-BH Templates}
\includegraphics[width=2\columnwidth]{plots/TT_BHspinBias_vs_SNR_q23.pdf}%\quad
\caption{This figure is similar to Fig.~\ref{fig:TNT_chirpMassBias_vs_SNR_q23},
with the differences that we consider the systematic bias in the dimensionless
spin on the BH here, and that the differences shown are absolute and not
fractions of the injected spin.}
\label{fig:TNT_BHspinBias_vs_SNR_q23}
\end{figure*}
% 
\begin{figure*}[h]
	\centering    
	\textbf{With BH-BH Templates}
	\includegraphics[width=2\columnwidth]{plots/TN_BHspinCIWidth90_vs_SNR_q23.pdf}\\ 
	\textbf{With NS-BH Templates}
	\includegraphics[width=2\columnwidth]{plots/TT_BHspinCIWidth90_vs_SNR_q23.pdf}%\quad
	\caption{This figure is similar to Fig.~\ref{fig:TNT_chirpMassCIWidth90_vs_SNR_q23},
		with the difference that we consider the confidence intervals in the recovery
		of the dimensionless spin on the BH $\chi_\mathrm{BH}$ here.}
	\label{fig:TNT_BHspinCIWidth90_vs_SNR_q23}
\end{figure*}
% 
\begin{figure*}
\centering
\textbf{With NS-BH Templates}
\includegraphics[width=2\columnwidth]{plots/TT_NSLambdaBias_vs_SNR_q23.pdf}
\caption{This figure is similar to the bottom row of
Fig.~\ref{fig:TNT_chirpMassBias_vs_SNR_q23}, with the difference that  we
consider the NS tidal parameter $\Lambda$ here.}
\label{fig:TT_NSLambdaBias_vs_SNR_q23}
\end{figure*}
% 
\begin{figure*}
	\centering
	\textbf{With NS-BH Templates}
	\includegraphics[width=2\columnwidth]{plots/TT_NSLambdaCIWidth90_vs_SNR_q23.pdf}
	\caption{This figure is similar to the bottom row of
		Fig.~\ref{fig:TNT_chirpMassCIWidth90_vs_SNR_q23}, with the difference that  we
		consider the NS tidal parameter $\Lambda$ here.}
	\label{fig:TT_NSLambdaCIWidth90_vs_SNR_q23}
\end{figure*}
% 
% 
Advanced LIGO searches and parameter estimation efforts aimed at binaries 
containing NS and stellar-mass BHs are poised to use waveform models that do not
include the effects of the NS tidal deformability. While this is not expected to
be the dominant source of error at low SNRs, it will likely introduce a
systematic bias in the physical parameters that we recover from GW observations.
In this section, we study the impact of the same for binaries where the BH spins
are aligned to the orbital angular momentum.

\begin{enumerate}
\item Above what SNR values, below what mass-ratios, above what BH spins, etc, 
do we begin to care about NS deformability?\newline
``one interesting plot to make would be the parameter bias (max-Likelihood
parameters minus true parameters) vs SNR, as functionals of q/sBH/Lambda, when
using T signals recovered with N templates.''
\end{enumerate}


Could have subsections for aLIGO and ET ?

%%%%%%%%%%%%%%%%%%%%%%%%%%%%%%%%%%%%%%%%%%%%%%%%%%%%%%%%%%%%%%%%%%%%%%%%%%%%%%%
\section{Using templates with tidal effects}
%%%%%%%%%%%%%%%%%%%%%%%%%%%%%%%%%%%%%%%%%%%%%%%%%%%%%%%%%%%%%%%%%%%%%%%%%%%%%%%
Having shown in the previous section that we begin to care for NS tidal effects
for signals in the XXX corner of the paraemter space, with SNRs above YY, here
we investigate the effect of the improvement in the accuracy of the recovered
parameters when tidal effects are included in the templates.

\begin{enumerate}
\item What is the reduction in the bias of the maximum likelihood parameters
when using tidal templates?\newline
``Plot the ratio of the bias between N templates and T templates (against T
signals), as a function of SNR.''
\item What is the reduction in the uncertainty in binary mass and spin, if
any?\newline
``show how the the $95\%$ confidence intervals shrink, as a function of SNR, 
when we go from using N templates to T templates.''
\end{enumerate}



Could have subsections for aLIGO and ET ?


%%%%%%%%%%%%%%%%%%%%%%%%%%%%%%%%%%%%%%%%%%%%%%%%%%%%%%%%%%%%%%%%%%%%%%%%%%%%%%%
\section{Discussion}
%%%%%%%%%%%%%%%%%%%%%%%%%%%%%%%%%%%%%%%%%%%%%%%%%%%%%%%%%%%%%%%%%%%%%%%%%%%%%%%
Discussion

%%%%%%%%%%%%%%%%%%%%%%%%%%%%%%%%%%%%%%%%%%%%%%%%%%%%%%%%%%%%%%%%%%%%%%%%%%%%%%%
% Acknowledgments
%%%%%%%%%%%%%%%%%%%%%%%%%%%%%%%%%%%%%%%%%%%%%%%%%%%%%%%%%%%%%%%%%%%%%%%%%%%%%%%
\begin{acknowledgments}
Acknowledgments
\end{acknowledgments}

%%%%%%%%%%%%%%%%%%%%%%%%%%%%%%%%%%%%%%%%%%%%%%%%%%%%%%%%%%%%%%%%%%%%%%%%%%%%%%%
\section*{References}
%%%%%%%%%%%%%%%%%%%%%%%%%%%%%%%%%%%%%%%%%%%%%%%%%%%%%%%%%%%%%%%%%%%%%%%%%%%%%%%
\bibliography{References/References}

\end{document}
