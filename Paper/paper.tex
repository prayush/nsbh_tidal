\documentclass[aps,prd,amsmath,floats,floatfix, twocolumn,
superscriptaddress,nofootinbib,showpacs]{revtex4-1}

\usepackage[colorlinks, pdfborder={0 0 0}, plainpages=false]{hyperref}
\usepackage{graphicx}
\usepackage{xspace}
\usepackage[usenames,dvipsnames]{color}
\usepackage{amssymb}

\newcommand{\roughly}{\mathchar"5218\relax} % Different from \sim in spacing

% Macros for text changes
\newcommand{\red}{\textcolor{red}}
\newcommand{\dan}[1]{\textcolor{WildStrawberry}{#1}}
\newcommand{\Mark}[1]{\textcolor{Cerulean}{#1}}
\newcommand{\larry}[1]{\textcolor{OliveGreen}{#1}}
\newcommand{\bela}[1]{\textcolor{Blue}{#1}}
\newcommand{\saul}[1]{\textcolor{Orange}{#1}}

% Macros for text notes and comments
\newcommand{\Note}[1]{\textcolor{blue}{\textbf{[#1]}}}


\newcommand{\Caltech}{\affiliation{Theoretical Astrophysics 350-17,
    California Institute of Technology, Pasadena, CA 91125, USA}}
\newcommand{\Cornell}{\affiliation{Center for Radiophysics and Space
    Research, Cornell University, Ithaca, New York 14853, USA}}
\newcommand{\CITA}{\affiliation{Canadian Institute for Theoretical
    Astrophysics, 60 St.~George Street, University of Toronto,
    Toronto, ON M5S 3H8, Canada}} %
\newcommand{\GWPAC}{\affiliation{Gravitational Wave Physics and
    Astronomy Center, California State University Fullerton,
    Fullerton, California 92834, USA}} %


%%%%%%%%%%%%%%%%%%%%%%%%%%%%%%%%%%%%%%%%%%%%%%%%%%%%%%%%%%%%%%%

\begin{document}

\title{
Measuring matter effects in Neutron star - Black hole binaries with Advanced LIGO
}

\author{People across the Atlantic}

\date{\today}

\begin{abstract}
Abstract
\end{abstract}

\pacs{}
% 04.25.D- Numerical relativity
% 04.25.dg Numerical studies of black holes and black-hole binaries
% 04.25.Nx Post-Newtonian approximation; perturbation theory; related approximations 
% 04.30.-w Gravitational waves (see also 04.80.Nn Gravitational wave detectors and experiments)
% 04.30.Db Wave generation and sources 
% 02.70.Hm Spectral methods

\maketitle

%%%%%%%%%%%%%%%%%%%%%%%%%%%%%%%%%%%%%%%%%%%%%%%%%%%%%%%%%%%%%%%%%%%%%%%%%%%%%%%
\section{Introduction}
%%%%%%%%%%%%%%%%%%%%%%%%%%%%%%%%%%%%%%%%%%%%%%%%%%%%%%%%%%%%%%%%%%%%%%%%%%%%%%%
Introduction

%%%%%%%%%%%%%%%%%%%%%%%%%%%%%%%%%%%%%%%%%%%%%%%%%%%%%%%%%%%%%%%%%%%%%%%%%%%%%%%
\section{Techniques}
%%%%%%%%%%%%%%%%%%%%%%%%%%%%%%%%%%%%%%%%%%%%%%%%%%%%%%%%%%%%%%%%%%%%%%%%%%%%%%%
\subsection{Modeling tidal deformation during inspiral}
Describe the PN terms used here

\subsection{Modeling tidal disruption near merger}
Describe the Lackey et al model

\subsection{MCMC methods}
Describe the emcee-based PE code

%%%%%%%%%%%%%%%%%%%%%%%%%%%%%%%%%%%%%%%%%%%%%%%%%%%%%%%%%%%%%%%%%%%%%%%%%%%%%%%
\section{SNR loss from ignoring NS tidal effects}
%%%%%%%%%%%%%%%%%%%%%%%%%%%%%%%%%%%%%%%%%%%%%%%%%%%%%%%%%%%%%%%%%%%%%%%%%%%%%%%
See what is the loss in SNR if we recover NSBH signals with a template bank of
BH-BH waveforms? It is very likely that the answer would be: $<1\%$. But it
is very interesting to show that for spinning BHs!

%%%%%%%%%%%%%%%%%%%%%%%%%%%%%%%%%%%%%%%%%%%%%%%%%%%%%%%%%%%%%%%%%%%%%%%%%%%%%%%
\section{Systematic bias in recovered parameters}
%%%%%%%%%%%%%%%%%%%%%%%%%%%%%%%%%%%%%%%%%%%%%%%%%%%%%%%%%%%%%%%%%%%%%%%%%%%%%%%
\begin{figure*}[h]
\centering 
\textbf{Without BH-BH Templates}
\includegraphics[width=2\columnwidth]{plots/TN_chirpMassBias_vs_SNR_q23.pdf}\\ 
\textbf{Without NS-BH Templates}
\includegraphics[width=2\columnwidth]{plots/TT_chirpMassBias_vs_SNR_q23.pdf}%\quad
\caption{These figures show the unfaithfulness of the two EOB models we consider
in this paper against NR, with the top panel corresponding to SEOBNRv1 and the bottom panel to
SEOBNRv2. Note that the SEOBNRv1 model does not include component spins
higher than $+0.6$, we cannot compute its faithfulness in that region of binary parameter space.
We find that both models have high agreement with NR for smaller mass-ratios 
and/or small spin magnitudes. The SEOBNRv1 model significantly deviates from the NR 
waveform when the spin of the primay black hole reaches $\sim +0.6$ for $q=3$, while
SEOBNRv2 appears to be unfaithful to NR when \textit{both} black holes have high spin 
magnitudes, especially when both spins are aligned with the orbital angular momentum.
}
\label{fig:SEOB_unfaith_TotalMass_Spin1z_Spin2z}
\end{figure*}
% 
\begin{figure*}[h]
\centering    
\textbf{Without BH-BH Templates}
\includegraphics[width=2\columnwidth]{plots/TN_EtaBias_vs_SNR_q23.pdf}\\ 
\textbf{Without NS-BH Templates}
\includegraphics[width=2\columnwidth]{plots/TT_EtaBias_vs_SNR_q23.pdf}%\quad
\caption{These figures show the unfaithfulness of the two EOB models we consider
in this paper against NR, with the top panel corresponding to SEOBNRv1 and the bottom panel to
SEOBNRv2. Note that the SEOBNRv1 model does not include component spins
higher than $+0.6$, we cannot compute its faithfulness in that region of binary parameter space.
We find that both models have high agreement with NR for smaller mass-ratios 
and/or small spin magnitudes. The SEOBNRv1 model significantly deviates from the NR 
waveform when the spin of the primay black hole reaches $\sim +0.6$ for $q=3$, while
SEOBNRv2 appears to be unfaithful to NR when \textit{both} black holes have high spin 
magnitudes, especially when both spins are aligned with the orbital angular momentum.
}
\label{fig:SEOB_unfaith_TotalMass_Spin1z_Spin2z}
\end{figure*}
% 
\begin{figure*}[h]
\centering    
\textbf{Without BH-BH Templates}
\includegraphics[width=2\columnwidth]{plots/TN_BHspinBias_vs_SNR_q23.pdf}\\ 
\textbf{Without NS-BH Templates}
\includegraphics[width=2\columnwidth]{plots/TT_BHspinBias_vs_SNR_q23.pdf}%\quad
\caption{These figures show the unfaithfulness of the two EOB models we consider
in this paper against NR, with the top panel corresponding to SEOBNRv1 and the bottom panel to
SEOBNRv2. Note that the SEOBNRv1 model does not include component spins
higher than $+0.6$, we cannot compute its faithfulness in that region of binary parameter space.
We find that both models have high agreement with NR for smaller mass-ratios 
and/or small spin magnitudes. The SEOBNRv1 model significantly deviates from the NR 
waveform when the spin of the primay black hole reaches $\sim +0.6$ for $q=3$, while
SEOBNRv2 appears to be unfaithful to NR when \textit{both} black holes have high spin 
magnitudes, especially when both spins are aligned with the orbital angular momentum.
}
\label{fig:SEOB_unfaith_TotalMass_Spin1z_Spin2z}
\end{figure*}
% 
\begin{figure*}[h]
\centering    
\includegraphics[width=2\columnwidth]{plots/TT_NSLambdaBias_vs_SNR_q23.pdf}
\caption{These figures show the unfaithfulness of the two EOB models we consider
in this paper against NR, with the top panel corresponding to SEOBNRv1 and the bottom panel to
SEOBNRv2. Note that the SEOBNRv1 model does not include component spins
higher than $+0.6$, we cannot compute its faithfulness in that region of binary parameter space.
We find that both models have high agreement with NR for smaller mass-ratios 
and/or small spin magnitudes. The SEOBNRv1 model significantly deviates from the NR 
waveform when the spin of the primay black hole reaches $\sim +0.6$ for $q=3$, while
SEOBNRv2 appears to be unfaithful to NR when \textit{both} black holes have high spin 
magnitudes, especially when both spins are aligned with the orbital angular momentum.
}
\label{fig:SEOB_unfaith_TotalMass_Spin1z_Spin2z}
\end{figure*}
% 
% 
Advanced LIGO searches and parameter estimation efforts aimed at binaries 
containing NS and stellar-mass BHs are poised to use waveform models that do not
include the effects of the NS tidal deformability. While this is not expected to
be the dominant source of error at low SNRs, it will likely introduce a
systematic bias in the physical parameters that we recover from GW observations.
In this section, we study the impact of the same for binaries where the BH spins
are aligned to the orbital angular momentum.

\begin{enumerate}
\item Above what SNR values, below what mass-ratios, above what BH spins, etc, 
do we begin to care about NS deformability?\newline
``one interesting plot to make would be the parameter bias (max-Likelihood
parameters minus true parameters) vs SNR, as functionals of q/sBH/Lambda, when
using T signals recovered with N templates.''
\end{enumerate}


Could have subsections for aLIGO and ET ?

%%%%%%%%%%%%%%%%%%%%%%%%%%%%%%%%%%%%%%%%%%%%%%%%%%%%%%%%%%%%%%%%%%%%%%%%%%%%%%%
\section{Using templates with tidal effects}
%%%%%%%%%%%%%%%%%%%%%%%%%%%%%%%%%%%%%%%%%%%%%%%%%%%%%%%%%%%%%%%%%%%%%%%%%%%%%%%
Having shown in the previous section that we begin to care for NS tidal effects
for signals in the XXX corner of the paraemter space, with SNRs above YY, here
we investigate the effect of the improvement in the accuracy of the recovered
parameters when tidal effects are included in the templates.

\begin{enumerate}
\item What is the reduction in the bias of the maximum likelihood parameters
when using tidal templates?\newline
``Plot the ratio of the bias between N templates and T templates (against T
signals), as a function of SNR.''
\item What is the reduction in the uncertainty in binary mass and spin, if
any?\newline
``show how the the $95\%$ confidence intervals shrink, as a function of SNR, 
when we go from using N templates to T templates.''
\end{enumerate}



Could have subsections for aLIGO and ET ?


%%%%%%%%%%%%%%%%%%%%%%%%%%%%%%%%%%%%%%%%%%%%%%%%%%%%%%%%%%%%%%%%%%%%%%%%%%%%%%%
\section{Discussion}
%%%%%%%%%%%%%%%%%%%%%%%%%%%%%%%%%%%%%%%%%%%%%%%%%%%%%%%%%%%%%%%%%%%%%%%%%%%%%%%
Discussion

%%%%%%%%%%%%%%%%%%%%%%%%%%%%%%%%%%%%%%%%%%%%%%%%%%%%%%%%%%%%%%%%%%%%%%%%%%%%%%%
% Acknowledgments
%%%%%%%%%%%%%%%%%%%%%%%%%%%%%%%%%%%%%%%%%%%%%%%%%%%%%%%%%%%%%%%%%%%%%%%%%%%%%%%
\begin{acknowledgments}
Acknowledgments
\end{acknowledgments}

%%%%%%%%%%%%%%%%%%%%%%%%%%%%%%%%%%%%%%%%%%%%%%%%%%%%%%%%%%%%%%%%%%%%%%%%%%%%%%%
\section*{References}
%%%%%%%%%%%%%%%%%%%%%%%%%%%%%%%%%%%%%%%%%%%%%%%%%%%%%%%%%%%%%%%%%%%%%%%%%%%%%%%
\bibliography{References/References}

\end{document}
