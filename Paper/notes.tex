% \documentclass[a4paper,amsmath,floats,floatfix,12pt]{article}
\documentclass[aps,prd,amsmath,floats,floatfix,onecolumn,superscriptaddress,nofootinbib,showpacs]{revtex4-1}

% \usepackage[utf8]{inputenc}

% \usepackage[colorlinks, pdfborder={0 0 0}, plainpages=false]{hyperref}
\usepackage{graphicx}
\usepackage{xspace}
\usepackage[usenames,dvipsnames]{color}
\usepackage{amssymb}
\usepackage[dvipsnames]{xcolor}
\usepackage{placeins}

% Macros for text notes and comments
\newcommand{\Note}[1]{\textcolor{blue}{\textbf{[#1]}}}
\newcommand{\D}{\mathrm{d}}
\newcommand{\lambdans}{\Lambda_\mathrm{NS}}
\newcommand{\dlambda}{\delta\Lambda}

\newcommand{\ii}{\mathrm{i}}
\newcommand{\chibh}{\chi_\mathrm{BH}}
\newcommand{\chins}{\chi_\mathrm{NS}}
\newcommand{\mbh}{m_\mathrm{BH}}
\newcommand{\mns}{m_\mathrm{NS}}
\newcommand{\mchirp}{\mathcal{M}_c}
\newcommand{\LL}{\mathcal{L}}
\newcommand{\deff}{D_\mathrm{eff}}

\begin{document}
%opening
\title{NSBH tidal paper, Q\&A}
\author{Prayush Kumar}

% \begin{abstract}
% Blank
% \end{abstract}
\maketitle


\section{Derivation of Eq.8 and footnote 2}
The probability of $\lambdans$, given the $N$ unique and independent events,
along with all other assumptions about the universe bundled into $K$, is
\begin{eqnarray}
 p(\lambdans |d_1, d_2, \cdots, d_N, K) &= \dfrac{p(d_1,d_2,\cdots,d_N |\lambdans , K)\,p(\lambdans)}{\int\D\lambdans p(\lambdans ) p(d_1,d_2,\cdots,d_N |\lambdans , K)},\label{eq:11}\\
  &= \dfrac{p(\lambdans) \prod_{i} p(d_i|\lambdans, K)}{\int\D\lambdans p(\lambdans ) p(d_1,d_2,\cdots,d_N |\lambdans , K)},\label{eq:12}\\
  &= \dfrac{p(\lambdans) \prod_{i} \left( p(\lambdans |d_i, K)\dfrac{p(d_i)}{p(\lambdans)} \right)}{\int\D\lambdans p(\lambdans ) p(d_1,d_2,\cdots,d_N |\lambdans , K)}\label{eq:13};
\end{eqnarray}
where Eq.~\ref{eq:11} and Eq.~\ref{eq:13} are application of Bayes' theorem,
while Eq.~\ref{eq:12} comes from the mutual independence of all $N$ events. 
Assuming in addition that all events are {\it equally likely}: 
$p(d_i) = p(d_j) = p(d)$, we get
\begin{eqnarray}
 p(\lambdans |d_1, d_2, \cdots, d_N, K) &= \dfrac{p(\lambdans) \left(\dfrac{p(d)}{p(\lambdans)}\right)^N\prod_{i} p(\lambdans |d_i, K) }{\int\D\lambdans p(\lambdans ) p(d_1,d_2,\cdots,d_N |\lambdans , K)}\label{eq:21},\\
  &= p(\lambdans)^{1-N}\times \dfrac{p(d)^N}{\int\D\lambdans p(\lambdans) p(d_1, d_2, \cdots, d_N |\lambdans, K)} \times\prod_i p(\lambdans |d_i, K)\label{eq:22}.
\end{eqnarray}
The first factor in Eq.~\ref{eq:22} is the prefactor appearing in Eq.(8), and
the second factor in Eq.~\ref{eq:22} is the normalization factor I hint at
in footnote 2.










\end{document}
